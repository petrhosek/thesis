\section{Summary}
\label{multi-version:summary}

Software updates are an integral part of the software development and
maintenance process, but unfortunately they present a high risk, as new
releases often introduce new bugs and security vulnerabilities.

In this chapter, we have argued for a new way of performing software updates,
in which the new version of an application is run in parallel with old
application versions, in order to increase the reliability and security of the
overall system. We believe that multi-version software updates can have a
significant impact on current software engineering practices, by allowing
frequent software updates without sacrificing the stability and security of
older versions.

We proposed two different multi-version execution schemes: failure recovery and
transparent failover, targeting different scenarios with different trade-offs.
We also presented two different designs for building monitors suitable for
multi-version execution implementing these schemes. The first one, called \mx
described in Chapter~\ref{chap:safe-updates}, is focused on surviving crashes
caused by bugs introduced in software updates, with the prototype
implementation built using the \ptrace mechanism. The second one, called \varan
described in Chapter~\ref{chap:efficient-execution}, is aimed towards running a
large number of versions side-by-side with low performance overhead, and uses
selective binary rewriting to achieve this goal. 

\section{Real-world scenarios}
\label{multi-version:scenarios}

To motivate the multi-version execution technique, we present three existing
scenarios involving the \vim editor, as well as \lighttpd, and \redis servers.
These correspond to two categories of applications that could benefit from our
approach: user applications such as text editors for which reliability is a key
concern; and network servers, with stringent security and availability
requirements.

% To motivate the multi-version execution technique, we present several existing
% scenarios involving the \chrome browser and the \vim editor, as well as the
% \lighttpd, \redis and \vsftpd servers. These correspond to two categories of
% applications that could benefit from our approach: desktop applications such as
% web browsers and office tools for which reliability is a key concern; and
% network servers, with stringent security and dependability requirements.

% \gchrome\footnote{\url{http://www.google.com/chrome}} is one of the most widely
% used web browsers.  Even though \chrome releases are tested extensively before
% deployment, they sometimes introduce new bugs that affect the stability of the
% browser.  A concrete example is version $6.0.466.0$, which introduced a bug
% that caused \chrome to crash when trying to load certain web pages over
% SSL.\footnote{\url{http://code.google.com/p/chromium/issues/detail?id=49197}}
% One might argue that in this case the user should downgrade to an older version
% and wait until the bug is fixed. However, versions immediately preceding
% $6.0.466.0$ suffer from a different
% bug,\footnote{\url{http://code.google.com/p/chromium/issues/detail?id=49721}}
% which was introduced in version $6.0.438.0$ and which crashes Chrome during
% certain sequences of repeated back and forward navigation.

% \vim\footnote{\url{http://www.vim.org/}} is arguably one of the most popular
% text editors.  In version $7.1.127$, while trying to fix a memory leak,
% \vim developers introduced a double \textstt{free} bug that caused \vim to
% crash whenever the user tried to use a path completion feature.  This bug made
% its way into \textit{Ubuntu} $8.04$, affecting a large number of
% users.\footnote{\url{https://bugs.launchpad.net/ubuntu/+source/vim/+bug/219546}}

% \lighttpd\footnote{\url{http://www.lighttpd.net/}} is a popular web-server used
% by several high-traffic websites such as Wikipedia and Xkcd.
% Despite its popularity, crash bugs are still a common
% occurrence in \lighttpd, as evident from its bug tracking
% database.\footnote{\url{http://redmine.lighttpd.net/issues/}}
% As one example, a patch introduced in April
% 2009\footnote{\url{http://redmine.lighttpd.net/projects/lighttpd/repository/revisions/2438}}
% (correctly) fixed the way HTTP ETags are computed.
% Unfortunately, this fix broke the support for compression, which relied on the
% previous way in which ETags were computed and resulted in a segmentation fault
% whenever a client requested HTTP compression.  This issue was only diagnosed
% and reported in March
% 2010\footnote{\url{http://redmine.lighttpd.net/issues/2169}} and fixed at the
% end of April
% 2010,\footnote{\url{http://redmine.lighttpd.net/projects/lighttpd/repository/revisions/2723}}
% more than one year after it was introduced, leaving the server vulnerable to
% possible attacks in between.

% \redis is an advanced key-value data structure
% server,\footnote{\url{http://redis.io/}} often referred to as one the most
% popular NoSQL databases.  Due to its performance and low-resource requirements,
% \redis is being used by many well-known services such as GitHub.  Because the
% whole dataset is held in memory, reliability is critically important, as a
% crash could result in total data loss.  However, like any other large software
% system, \redis is subject to crash bugs. Issue
% 344\footnote{\url{http://code.google.com/p/redis/issues/detail?id=344}} is one
% such example.  This issue causes \redis to crash when the \textstt{HMGET}
% command is used with the wrong type. The bug was introduced in April 2010,
% diagnosed and reported only half a year later in October 2010 and then fixed
% after fifteen days.

% \vsftpd\footnote{\url{https://security.appspot.com/vsftpd.html}} is a fast and 
% secure FTP server for UNIX systems.  Version $2.2.0$ added several new
% features such as network isolation, but unfortunately also introduced
% a bug\footnote{\url{https://bugs.launchpad.net/ubuntu/+source/vsftpd/+bug/462749}}
% which triggered a segmentation fault whenever a client used the
% passive FTP mode.  This bug made \vsftpd practically unusable since
% the passive mode is being frequently used by clients behind firewalls.
% Despite being reported several times, this bug was only fixed in
% version $2.2.1$, released more than two months after the bug was
% introduced.

\subsection{\vim double free on home path completion}

\vim\footnote{\url{http://www.vim.org/}} is arguably one of the most popular
text editors. While trying to fix a memory leak, \vim developers introduced a
double \lstinline`free` bug that caused \vim to crash whenever the user tried to
use a path completion feature. This bug was introduced in version 7.1.127,
released in September 2007. The incorrectly fixed version of the
\lstinline`ExpandOne` function is shown in Listing~\ref{lst:vim-bug}. The newly
introduced code frees up the memory used to store the expanded path.
Unfortunately, when path contains the \lstinline`~` symbol used as an alias for
user's home directory, this code path will be invoked twice leading to a crash.

\begin{lstlisting}[alsolanguage=diff,numbers=none,label=lst:vim-bug,caption={First (incorrect) fix of \lstinline`ExpandOne` function in \vim.}]
@@ -3543,0 +3548,4 @@ ExpandOne(xp, str, orig, options, mode)
+    /* Free "orig" if it wasn't stored in "orig_save". */
+    if (orig != orig_save)
+       vim_free(orig);
+
\end{lstlisting}

The double \lstinline`free` bug was detected and fixed in version 7.1.147, released
one month later in October 2007. The correct version of the \lstinline`ExpandOne`
function is shown in Listing~\ref{lst:vim-fixed}. In this version, the
\lstinline`orig_saved` variable type was changed to \lstinline`int` and is
being used as a flag preventing the second \lstinline`free` invocation.

\begin{lstlisting}[alsolanguage=diff,numbers=none,label=lst:vim-fixed,caption={Second (correct) fix \lstinline`ExpandOne` function in \vim.}]
@@ -3355,0 +3356 @@ ExpandOne(xp, str, orig, options, mode)
+    int                orig_saved = FALSE;
@@ -3423,0 +3425 @@ ExpandOne(xp, str, orig, options, mode)
+       orig_saved = TRUE;
@@ -3549 +3551 @@ ExpandOne(xp, str, orig, options, mode)
-    if (orig != orig_save)
+    if (!orig_saved)
        vim_free(orig);
\end{lstlisting}

Despite the bug being fixed promptly, it made its way into Ubuntu 8.04,
affecting a large number of
users.\footnote{\url{https://bugs.launchpad.net/ubuntu/+source/vim/+bug/219546}}
Since the 8.04 is a long-term support version, the Ubuntu maintainers avoid
incorporating new changes except for critical security bug fixes. This meant
that the \vim users affected by this bug had to build and install their own
version of \vim from the source.

\subsection{\lighttpd mod\_compress segfault with disabled ETag support}

\lighttpd\footnote{\url{http://www.lighttpd.net/}} is a popular open-source 
web-server used either alone or in conjunction with other web-servers
by several high-traffic websites such as Wikipedia and Xkcd.
Despite its popularity, crash bugs are still a common
occurrence in \lighttpd, as evident from its bug tracking
database.\footnote{\url{http://redmine.lighttpd.net/issues/}}  Below
we discuss one such bug, which our approach could successfully
eliminate.

In April 2009, a patch was
applied\footnote{\url{http://redmine.lighttpd.net/projects/lighttpd/repository/revisions/2438}}
to \lighttpd's code related to the HTTP ETag functionality.  An ETag
is a unique string assigned by a web server to a specific version of a
web resource, which can be used to quickly determine if the resource
has changed.  The patch was a one-line change, which discarded the
terminating zero when computing a hash representing the ETag.  More
exactly, line 47 in \textstt{etag.c}:

\begin{lstlisting}[numbers=none,breaklines=true,xleftmargin=0pt]
for (h=0, i=0; i < etag->used; ++i) h = (h<<5)^(h>>27)^(etag->ptr[i]);
\end{lstlisting}
\noindent was changed to:
\begin{lstlisting}[numbers=none,breaklines=true,xleftmargin=0pt]
for (h=0, i=0; i < etag->used@-1@; ++i) h = (h<<5)^(h>>27)^(etag->ptr[i]);
\end{lstlisting}

This correctly changed the way ETags are computed, but unfortunately,
it broke the support for compression, whose implementation depended on
the previous computation.  More precisely, \lighttpd's support for HTTP
compression uses caching to avoid re-compressing files which have not
changed since the last access.  To determine whether the cached
file is still valid, \lighttpd internally uses ETags.  Unfortunately,
the code implementing HTTP compression did not consider the case when
ETags are disabled.  In this case, \textstt{etags->used}
is \textstt{0}, and when the line above is
executed, \textstt{etag->used-1} underflows to a very large value, and
the code crashes while accessing \textstt{etag->ptr[i]}.
Interestingly enough, the original code was still buggy (it always
returns zero as the hash value, and thus it would never re-compress
the files), but it was not vulnerable to a crash.

The segfault was diagnosed and reported in March
2010\footnote{\url{http://redmine.lighttpd.net/issues/2169}} and fixed
at the end of April
2010,\footnote{\url{http://redmine.lighttpd.net/projects/lighttpd/repository/revisions/2723}}
more than one year after it was introduced.  
%The history is depicted graphically in Figure~\ref{fig:lighttpd-history}.  
% The bottom line is
% that for about one year, users affected by this buggy patch
% essentially had to decide between%
% \begin{inparaenum}[(1)]
% \item incorporating security and bug fixes added to the code, but being
%   vulnerable to this crash bug, and
% \item giving up on these security and bug fixes and using an old version of
%   \lighttpd, which is not vulnerable to this bug.
% \end{inparaenum}
% Note that this is particularly true for the eleven-month period
% between the time when the bug was introduced and the time it was
% diagnosed, since during this period most users would not know how to
% change the server's configuration to avoid the crash.

%% The original code, which can be seen in Listing~\ref{lst:2437}, makes the
%% \texttt{i < etag->used} comparison (line 4). Because both \texttt{i} and
%% \texttt{etag->used} are $0$, the condition \texttt{0 < 0} does not hold and
%% the loop body will never be executed.

%% The affected code can be seen in Listing~\ref{lst:2438}. Here, the condition
%% has been changed and comparison has now the form \texttt{i < etag->used-1}.
%% When executing, the \texttt{etag->used} variable will underflow and condition
%% \texttt{0 < 4294967295} will be true. Therefore, the loop body will be
%% executed and access to \texttt{etag->ptr[0]} will result in segmentation
%% fault.

\subsection{\redis crash on wrong \textstt{HMGET} type}

\begin{figure}[t]
\begin{minipage}[b]{0.90\columnwidth}
\begin{lstlisting}[label=lst:original, caption={Original (correct) version of the {\footnotesize \texttt{hmgetCommand}} function in \redis.}]
robj *o = lookupKeyRead(c->db, c->argv[1]); /*@\label{line:key-found}@*/
if (o == NULL) {
  addReplySds(c,sdscatprintf(sdsempty(),"*%d\r\n",c->argc-2));
  for (i = 2; i < c->argc; i++) {
    addReply(c,shared.nullbulk);
  }
  return;
} else {
  if (o->type != REDIS_HASH) { /*@\label{line:type-found}@*/
    addReply(c,shared.wrongtypeerr); /*@\label{line:report-error}@*/
    return; /*@\label{line:return}@*/
  }
}
addReplySds(c,sdscatprintf(sdsempty(),"*%d\r\n",c->argc-2));
\end{lstlisting}
\end{minipage}
\hspace{2.2\columnsep}
\begin{minipage}[b]{0.9\columnwidth}
\begin{lstlisting}[label=lst:refactored, caption={Refactored (buggy) version of the {\footnotesize \texttt{hmgetCommand}} function in \redis.}]
robj *o, *value;
o = lookupKeyRead(c->db,c->argv[1]);
if (o != NULL && o->type != REDIS_HASH) {
  addReply(c,shared.wrongtypeerr); /*@\label{line:report-error2}@*/
}
addReplySds(c,sdscatprintf(sdsempty(),"*%d\r\n",c->argc-2));
for (i = 2; i < c->argc; i++) {
  if (o != NULL && (value = hashGet(o,c->argv[i])) != NULL) { /*@\label{line:hashGet}@*/
    addReplyBulk(c,value);
    decrRefCount(value);
  } else {
    addReply(c,shared.nullbulk);
  }
}
\end{lstlisting}
\end{minipage}
\end{figure}

\redis is an advanced key-value data structure
server,\footnote{\url{http://redis.io/}} often referred to as one the most
popular NoSQL databases.  Due to its performance and low-resource requirements,
\redis is being used by many well-known services such as GitHub.  Because the
whole dataset is held in memory, reliability is critically important, as a
crash could result in total data loss.  However, like any other large software
system, \redis is subject to crash bugs. Issue
344\footnote{\url{http://code.google.com/p/redis/issues/detail?id=344}} is one
such example.  This issue causes \redis to crash when the \textstt{HMGET}
command is used with the wrong type.  The bug was introduced during a code
refactoring applied in revision \textstt{7fb16bac}.  The original code of the
problematic \textstt{hmgetCommand} function is shown in
Listing~\ref{lst:original}, while the (buggy) refactored version is shown in
Listing~\ref{lst:refactored}.

In the original code, if the lookup on line~\ref{line:key-found} is successful,
but the type is not \textstt{REDIS\_HASH} (line~\ref{line:type-found}), the
function returns after reporting an incorrect type
(lines~\ref{line:report-error}--\ref{line:return}). However, in the refactored
version (Listing~\ref{lst:refactored}), the \textstt{return} statement is
missing, and after reporting an incorrect type (line~\ref{line:report-error2}),
the function continues execution and causes segmentation fault inside the
\textstt{hashGet} function invoked on line~\ref{line:hashGet}. This is a
critical bug, which may result in losing some or even all of the stored data
in case when Append Only File (AOF) persistence log is corrupted due to issue
620\footnote{\url{http://code.google.com/p/redis/issues/detail?id=620}}.

The bug was introduced in April 2010, diagnosed and reported only half a year
later in October 2010 and then fixed after fifteen days.  As in the case of the
\lighttpd bug, this means that for several months, any \redis instance has been
susceptible to a crash and vulnerable to a possible attack.

\section{Future work}
\label{sec:future-work}

There are a number of other ways in which both \mx and \varan could be extended
in the future. In general, we would like to increase their flexibility, in
order to expand the range of versions we could run in parallel. This means both
tolerance to a wider set of changes as well as ability to detect different types
of divergences.

The recovery mechanism in \mx is based on the ability of restarting the crashed
version using a lightweight checkpointing mechanism. In our current
implementation, we only keep the latest checkpoint for performance reasons.
However, this means that \mx cannot recover a fault caused by a change prior to
the last checkpoint. We could raise this limitation by keeping more checkpoints
and iteratively retrying them if we fail to recover using a more recent one.
There is a trade-off involved: keeping more checkpoints incurs a higher
performance overhead and we would also need to keep the log of all the system
calls between the checkpointed state and the point of failure, akin to \varan's
record \& replay mechanism. Other option is to improve our checkpointing
mechanism by implementing it as a loadable kernel module that only stores the
part of the state needed for recovery~\cite{flashback}.

%To improve the performance overhead incurred by \mx, we could combine \ptrace
%with \seccompbpf to allow the system calls without side-effects to be executed
%directly without \mxm involvement~\cite{mbox}. 

While \varan allows executing a large number of versions in parallel, the
number of versions is currently set in the beginning. We would like enhance our
prototype implementation with the ability to dynamically adjust the number of
versions that are run concurrently. This will ensure that multi-version
applications will be able to utilize all available resources (\ie idle
processor time) without affecting the overall system performance during peak
load.

While scaling down the number of versions is straightforward, to scale up we
would need the ability to start new versions and allow them to catch up with the
leader. There are different ways in which such support could be implemented. For
applications structured around a central dispatch loop (\eg network servers or
applications with graphical user interface), we could infer the loop headers
(either statically~\cite{DJgraphs,havlak} or dynamically~\cite{sato11}) and
allow the new versions to ``catch up'' with the leader at the beginning of the
dispatch loop. Some DSU systems~\cite{kitsune} use a similar approach, but
rather than inferring the loop headers, they require the programmer to annotate
the application. We could combine \varan with one of these DSU systems to take
advantage of these annotations.

% Finally, we would like to be able to transparently run multi-version
% applications on multiple underlying platforms, ranging from
% multicore processors to large-scale data centers.  This requires the
% ability to span our virtualized environment across multiple logical as
% well as physical nodes.  In particular, we aim to include the
% possibility of executing certain versions of an application remotely,
% to enable scenarios with hundreds or even thousands of application
% versions.

% During execution, we perform regular checkpoints (\eg based on
% \textstt{fork} which uses copy-on-write internally) and we also record
% all the system calls since the last checkpoint.  On recovery, we first
% restore the failing version from the latest checkpoint and then replay
% all recorded system calls to bring the application to a state just
% before the crash.  The frequency of checkpointing involves a
% trade-off: frequent checkpointing incurs a performance overhead, but
% infrequent checkpointing requires more space to store the system call
% logs.  The two variants of MV execution discussed in this proposal
% would correspond to the two extremes: running the two versions in
% parallel and synchronising the execution at every system call
% (parallel MV execution), and executing the two versions sequentially
% with the second version only being run when the first version crashes
% (sequential MV execution).

% With checkpointing, we can explore points in-between the two extremes,
% and control the different trade-offs more precisely by adaptively
% changing the frequency with which the checkpoints are taken.  One
% strategy would be to monitor the rate of system calls issued by an
% application and adapt the checkpointing rate accordingly.  Ideally, one
% would checkpoint frequently enough to have the log fit in memory, which
% could significantly reduce the recording overhead by eliminating
% expensive I/O operations.

% Furthermore, the way we employ the record and replay mechanism gives us
% additional control over the recovery process.  We can only start replaying the
% execution after the version being recorded crashes, or we can reduce the
% recovery time by running both versions in parallel, recording system calls to a
% log in one version and immediately replaying them in the second version (which
% potentially could be run at lower priority).  While the former variant has been
% used in other contexts, the latter one is novel and more suitable for
% multi-version execution.  Furthermore, by restricting the maximal/minimal
% length of the system call log we can explore the design space in-between these
% two variants.

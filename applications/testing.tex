\section{Regression testing}
\label{sec:testing}

By default, whenever \mx detects a divergence between the two versions which
are being run in parallel, it either discard one of the versions, or, if the
divergence is caused by the segmentation fault, it tries to recover the failing
version (\S\ref{sec:rem}). This mode of operation is suitable as a way for
improving reliability of updated software, as shown in
Section~\ref{sec:reliability-evaluation}.

However, \mx can be also used in a different mode, in which the application is
fail-stopped whenever a divergence is detected. This mode of operation is
suitable for regression testing. When working on the new version of the
application developers would typically use the regression suite to ensure that
the new version does not break the existing functionality. When such a situation
occurs, the first step is typically to determine where the bug was introduced.
While this may be easy for simple sequential applications, such task could be
challenging for applications which use non-deterministic inputs such as network
events or random number generators.

One of the main goals of \mx is to intercept all sources of non-determinism and
provide both versions with the same input. This allows developers to run both
versions in parallel, and see whether the patch introduces any divergence by
running both applications in lockstep. When a divergence is detected, \mx can
also provide developers with additional information such as the stack trace
obtained by the \rem component.

While conceptually simple, we believe this mechanism could be helpful in many
development scenarios. Furthermore, we believe it could be integrated with
other mechanisms such as \lstinline`git bisect`, which normally relies on the
regression testing suite, which might be too coarse grained in some scenarios,
but when combined with \mx it can also detect more subtle differences (\eg due
to non-determinism).

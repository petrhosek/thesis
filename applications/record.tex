\section{Record-Replay}
\label{sec:record_replay}

Although \varan shares similarities with record-replay systems, there are
significant differences; in particular, the log is of fixed size and
only kept in-memory.  However, it is possible to easily extend \varan to
provide full record-replay capabilities by implementing two artificial
clients:
\begin{inparaenum}[(i)]
\item one acting as a \emph{follower} whose only goal is to write the
  content of the ring buffer to persistent storage, and
\item one acting as a \emph{leader}, reading the content of the log
  from the persistent storage and publishing events into the ring
  buffer for consumption by other clients.
\end{inparaenum}

Compared to some of the previous record-replay systems, \varan has a
number of advantages. First, decoupling the logic responsible for
reading/writing the log from the actual application into a separate
process allows the application to run at nearly full speed and utilize
the multiple cores available in modern CPUs.  Second, since \varan was
designed to run multiple instances at the same time, we can replay
multiple versions at once, \eg to determine which versions of the
application from a given range are susceptible to a crash reported by
the user.

We have implemented a simple prototype of the two aforementioned
clients on top of \varan and compared its performance against
\scribe~\cite{scribe}, a state-of-the-art record-replay system
implemented in the kernel.  Unfortunately, because \scribe is
implemented in the kernel and is only maintained for an old 32-bit
Linux kernel (2.6.35), we had to run our experiments inside a virtual
machine (kindly provided to us by \scribe's authors, as the source tree
was broken at the time of our experiments). 
%% This experience clearly shows one of the main disadvantages of
%% kernel-level frameworks---the difficulty of maintaining the code base.
To allow for a more faithful comparison, we ran \varan inside the same
virtual machine.

We used \redis as a benchmark, running the same workload as before,
and configured both systems to record the execution to persistent
storage.  We recorded an overhead of \redisRROvhScribe for
\scribe,\footnote{The overhead we measured for \scribe is higher than
  the overhead reported in~\cite{scribe}; however, note that the original
  work used less aggressive benchmarks such as \httpd.  The use of a
  virtual machine also affected the result.}  compared to
\redisRROvhNx for \varan.
%(but remember that we ran it inside a virtual machine)

% Despite being implemented in the kernel, \scribe
% performed worse than \varan on our \redis experiments: the overhead
% introduced by \scribe was \redisRROvhScribe, compared to only
% \redisRROvhNx for \varan.



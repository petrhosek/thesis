To evaluate \mx on \lighttpd, we have used two different crash bugs.
The first bug is the one described in detail in
\sref{sec:example}, related to the ETag and compression
functionalities.  As previously discussed, the crash is triggered by a
very small change, which decrements the upper bound of a \textstt{for}
loop by one.  \mx successfully protects the application against this
crash, and allows the new version to survive it by using the code of
the old version.

%As we discuss in \sref{sec:bounds}, \mx
%allows users to incorporate all the changes in the
%next \maxDistLighttpdOne revisions following the buggy patch, while
%still protecting the overall application against this crash.

%% the behaviour of the new version, except in the case when the new
%% version crashes, when the buggy line is effectively replaced by the
%% old working code.


%To evaluate \mx on \lighttpd, we have focused on two critical bugs,
%\#1601 and \#2140. Both of these bugs result in segmentation fault
%causing \lighttpd to crash.

%The bug \#1601 affects the HTTP redirection functionality, in particular
%the \texttt{\%n} substitution with condition substring. This functionality has
%been introduced in revision \texttt{510}. However, there is an incorrect
%comparison in one of the conditions which causes segmentation fault when
%appending matched parts to buffer if there was no matching regular expression.
%The affected code can be seen in Listing~\ref{lst:510}.

%\begin{lstlisting}[label=lst:510, caption={Original correct version of the function}]
%cond_cache_t *cache = &con->cond_cache[dc->context_ndx];
%if (n > cache->patterncount) {
  %return 0;
%}
%\end{lstlisting}

%The fix to this bug consists of a single changed line as can be seen in
%Listing~\ref{lst:2138} and has been incorporated in revision \texttt{2138}, yet
%this bug remained undetected for nearly three years (August 8, 2005 --- March
%26, 2008) rendering \lighttpd webserver vulnerable to attack.

%\begin{lstlisting}[label=lst:2138, caption={Refactored failing version of the function}]
%cond_cache_t *cache = &con->cond_cache[dc->context_ndx];
%if (n >= cache->patterncount) {
  %return 0;
%}
%\end{lstlisting}

The other crash bug we
reproduced
affects the URL rewrite functionality.\footnote{\url{http://redmine.lighttpd.net/projects/lighttpd/issues/2140}}
%This seems to have been to have been present in \lighttpd since the
%first revision, we
This is also caused by an incorrect bound in a \textstt{for} loop.
More precisely, the loop: 

\begin{lstlisting}[numbers=none,breaklines=true,xleftmargin=0pt]
for (k=0; k < pattern_len; k++)
\end{lstlisting}

\noindent should have been:

\begin{lstlisting}[numbers=none,breaklines=true,xleftmargin=0pt]
for (k=0; k@+1@ < pattern_len; k++)
\end{lstlisting}

The bug seems to have been present since the very first version
added to the repository.  It was reported in December 2009, and
fixed one month later.  As a result, we are running \mx using the last
version containing the bug together with the one that fixed it.  While
this bug does not fit within the pattern targeted by \mx (where a
newer revision introduces the bug), from a technical perspective it is
equally challenging.  \mx is able to successfully run the two versions
in parallel, and help the old version survive the crash bug.

%% Both \lighttpd bugs \#1601 and \#2140 are very simple - their fix
%% consist of a single character yet still they made \lighttpd server
%% vulnerable to a potential attack.

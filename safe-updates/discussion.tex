\section{Discussion}
\label{safe-updates:discussion}

This section discusses in more detail the scope of our approach, its limitation
and the different trade-offs involved.

%% This section discusses in more detail the scope of our approach
%% with regard to (1) the type of applications and code changes that
%% could benefit most from our approach, and (2) the trade-off between
%% availability/reliability/availability and strict
%% correctness/performance/energy consumption.

\paragraph{CPU utilisation and memory consumption} \mx incurs a performance
overhead, as discussed in \sref{sec:performance}. In our experience, \mx is
applicable to interactive applications such as command-line utilities and
interactive applications, where the performance degradation is not noticeable
to the user. \mx is not applicable to servers requiring high-throughput or to
patches that fix performance bugs, as the system runs no faster than the
slowest version.  Both of of these issues are addressed by
\varan.

Our approach of using idle CPU time to run additional versions also increases
energy consumption and therefore might not be applicable to energy-constrained
devices such as smartphones. However, it is interesting to note that idle CPUs
are not ``free'' either: even without considering the initial cost of
purchasing the, an energy-efficient server consumes half its full power when
doing virtually no work---and for other servers, this ratio is usually much
worse~\cite{barroso2007}.

Similarly, \mx also doubles the memory consumption, which in practice is less
of an issue but could still lead to a performance degradation due to increased
memory pressure for some applications.

\paragraph{Memory-based communication.} The current \mx prototype does not
support shared memory, which is often used for efficient communication between
processes, as its use would be invisible to our system call interposition
mechanism. One way we could handle this scenario is by serialising all accesses
to shared memory using using a repeatable deterministic task
scheduler~\cite{russinovich:pldi96}. This approach has been used by Paranoid
Android~\cite{paranoid-android} in the same context.

\paragraph{Deployment strategy} While our approach eases the decision of
applying a software update---as incorporating a new version would never
decrease the reliability of the multi-version application---the number of
versions that can be run in parallel is limited, being dictated by the number
of available resources (\eg the number of available CPU cores).  As a result,
we need a deployment strategy to decide what versions to use.  For example, we
could always run the last $N$ released versions (where $N$ is the number of
available resources), or we could always keep a one-year old version, etc.
This thesis focuses on designing and implementing multi-version execution
techniques, but in future work we plan to explore deployment strategies in more
detail.

\section{Discussion}
\label{safe-updates:discussion}

This section discusses in more detail the scope of our approach with
regard to the type of software updates suitable to multi-version
execution and the different trade-offs involved.

%% This section discusses in more detail the scope of our approach
%% with regard to (1) the type of applications and code changes that
%% could benefit most from our approach, and (2) the trade-off between
%% availability/reliability/availability and strict
%% correctness/performance/energy consumption.

\paragraph{Types of code changes} In order for \mx to be successful, the
external behaviour of the versions that are run in parallel has to be similar
enough to allow us to synchronise their execution.  Our empirical study in
\S\ref{evolution:external} shows that changes to the external behaviour of
an application are often minimal, so our approach should work well with
versions that are not too distant from one another.  Similarly, our system
relies on the assumption that versions re-converge to the same behaviour after
a divergence.  As a result, we believe \mx would be a good fit for applications
that perform a series of mostly independent requests, such as network servers.
These applications are usually structured around a main dispatch loop, which
provides a useful re-convergence point.  Our approach is also suitable to local
code changes, which have small propagation distances, thus ensuring that the
different versions will eventually re-converge to the same behaviour.

\paragraph{Trade-offs involved} Our approach is targeted toward scenarios
where the availability, reliability and security of a software system is more
important than strict correctness, high performance and low energy consumption.  

In terms of correctness guarantees, \mx is similar to previous approaches such
as failure oblivious computing~\cite{fo} which may sacrifice strict correctness
for increased availability and security (see \S\ref{sec:rem} for details
regarding possible problems caused by \mx).  However, \mx alleviates this
problem by using a previously correct piece of code to execute through the
crash, and by discovering most potential problems by regularly checking if the
two versions have the same external behaviour.  Finally, note that \mx always
reverts to running a single version when a non-resolvable divergence is
detected.

%% is never worse than simply using one version of the software: if a
%% non-crashing divergence is detected, \mx can simply continue execution
%% with a single program version.

\paragraph{CPU utilization} \mx incurs a performance overhead, as discussed in
\sref{sec:performance}.  In our experience, \mx is readily deployable to
interactive applications such as command-line utilities, text editors and other
office tools, where the performance degradation is not noticeable to the user.
We believe it is also applicable to server applications where availability is
more important than high performance.  \mx is not applicable to patches that
fix performance bugs, as the system runs no faster than the slowest
version.

Our approach of using idle CPU time to run additional versions also increases
energy consumption.  However, it is interesting to note that idle CPUs are not
``free'' either: even without considering the initial cost of purchasing the
cores left idle, an energy-efficient server consumes half its full power when
doing virtually no work---and for other servers, this ratio is usually much
worse~\cite{barroso2007}.
% and therefore might not be applicable to energy-constrained devices
%such as smartphones.

\paragraph{Deployment strategy} While our approach eases the decision of
applying a software update---as incorporating a new version would never
decrease the security and reliability of the overall multi-version
application---the number of versions that can be run in parallel is limited,
being dictated by the number of available resources (\eg the number of
available CPU cores).  As a result, we need a deployment strategy to decide
what versions to use.  For example, we could always run the last $N$ released
versions (where $X$ is the number of available resources), or we could always
keep a one-year old version, etc.  This thesis focuses on techniques for
allowing multiple versions to successfully coordinate their parallel execution,
but in future work we plan to explore deployment strategies in more detail.

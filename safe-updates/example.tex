\section{Motivating Example}
\label{sec:example}

%% Applicable in a variety of scenarios, example of one such
%% application might be the desktop and office software where users
%% care about reliability more than performance or efficiency.
%% Examples of such scenarios are provided further in this
%% section. Another application might be the software systems where
%% reliability is critically important, such as web servers,
%% databases, \etc

%% We also believe, that the proposed approach might facilitate and
%% improve the \emph{continuous deployment} technique. While this
%% technique has been frequently proposed and advocated in the
%% software engineering community because it encourages
%% experimentation, innovation, and rapid
%% iteration~\cite{johnson2009,harmess2009,linden2009}, its use is
%% still limited since updates may introduce new bugs and security
%% vulnerabilities.  We believe that our approach provides all the
%% benefits of continuous deployment without compromising software
%% reliability.

To motivate our approach, we present a real scenario involving
\lighttpd, which is representative of one type of applications which
could benefit from our approach, namely server applications with
stringent security and availability requirements.


%% that achieves high-scalability, without sacrificing
%% standards-compliance and security having a small-memory footprint
%% and a small CPU load.  As a result, \lighttpd is
\lighttpd\footnote{\url{http://www.lighttpd.net/}} is a popular open-source 
web-server used 
%(either alone or in conjunction with other web-servers)
by several high-traffic websites such as Wikipedia and Xkcd.
Despite its popularity, crash bugs are still a common
occurrence in \lighttpd, as evident from its bug tracking
database.\footnote{\url{http://redmine.lighttpd.net/issues/}}  Below
we discuss one such bug, which our approach could successfully
eliminate.

%% In October 2008, a bug was reported in \lighttpd affecting the HTTP
%% ETag
%% functionality\footnote{\url{http://redmine.lighttpd.net/issues/1800}}.
%% An ETag a fingerprint assigned by a web server to a specific version
%% of a web resource, which can be used to quickly determine if the
%% resource has changed.  The bug in \lighttpd was that invalid ETags
%% were generated when compression was used.  The bug was fixed in
%% revision
%% 2386\footnote{\url{http://redmine.lighttpd.net/projects/lighttpd/repository/revisions/2386}},

% April 9th 2009
In April 2009, a patch was
applied\footnote{\url{http://redmine.lighttpd.net/projects/lighttpd/repository/revisions/2438}}
to \lighttpd's code related to the HTTP ETag functionality.  An ETag
is a unique string assigned by a web server to a specific version of a
web resource, which can be used to quickly determine if the resource
has changed.  The patch was a one-line change, which discarded the
terminating zero when computing a hash representing the ETag.  More
exactly, line 47 in \textstt{etag.c}:

\begin{lstlisting}[numbers=none,breaklines=true,xleftmargin=0pt]
for (h=0, i=0; i < etag->used; ++i) h = (h<<5)^(h>>27)^(etag->ptr[i]);
\end{lstlisting}
\noindent was changed to:
\begin{lstlisting}[numbers=none,breaklines=true,xleftmargin=0pt]
for (h=0, i=0; i < etag->used@-1@; ++i) h = (h<<5)^(h>>27)^(etag->ptr[i]);
\end{lstlisting}

This correctly changed the way ETags are computed, but unfortunately,
it broke the support for compression, whose implementation depended on
the previous computation.  More precisely, \lighttpd's support for HTTP
compression uses caching to avoid re-compressing files which have not
changed since the last access.  To determine whether the cached
file is still valid, \lighttpd internally uses ETags.  Unfortunately,
the code implementing HTTP compression did not consider the case when
ETags are disabled.  In this case, \textstt{etags->used}
is \textstt{0}, and when the line above is
executed, \textstt{etag->used-1} underflows to a very large value, and
the code crashes while accessing \textstt{etag->ptr[i]}.
Interestingly enough, the original code was still buggy (it always
returns zero as the hash value, and thus it would never re-compress
the files), but it was not vulnerable to a crash. %denial of service
                                                  %attack.

%% \begin{figure}
%% \centering
%% \includegraphics[width=0.9\columnwidth]{safe-updates/figures/lighttpd-scenario}
%% \caption{Crash bug \#2169 from {\footnotesize \texttt{lighttpd}}.}
%% \label{fig:lighttpd-history}
%% \end{figure}

% 8 March
The segfault was diagnosed and reported in March
2010\footnote{\url{http://redmine.lighttpd.net/issues/2169}} and fixed
at the end of April
2010,\footnote{\url{http://redmine.lighttpd.net/projects/lighttpd/repository/revisions/2723}}
more than one year after it was introduced.  
%The history is depicted graphically in Figure~\ref{fig:lighttpd-history}.  
The bottom line is
that for about one year, users affected by this buggy patch
essentially had to decide between%
\begin{inparaenum}[(1)]
\item incorporating the new features
and bug fixes added to the code, but being vulnerable to this crash
bug, and
\item giving up on these new features and bug fixes and using
an old version of \lighttpd, which is not vulnerable to this bug.
\end{inparaenum}
Note that this is particularly true for the eleven-month period
between the time when the bug was introduced and the time it was
diagnosed, since during this period most users would not know how to
change the server's configuration to avoid the crash.

%% The original code, which can be seen in Listing~\ref{lst:2437}, makes the
%% \texttt{i < etag->used} comparison (line 4). Because both \texttt{i} and
%% \texttt{etag->used} are $0$, the condition \texttt{0 < 0} does not hold and
%% the loop body will never be executed.

%% The affected code can be seen in Listing~\ref{lst:2438}. Here, the condition
%% has been changed and comparison has now the form \texttt{i < etag->used-1}.
%% When executing, the \texttt{etag->used} variable will underflow and condition
%% \texttt{0 < 4294967295} will be true. Therefore, the loop body will be
%% executed and access to \texttt{etag->ptr[0]} will result in segmentation
%% fault.

Our approach provides users with a third choice; when a new version
arrives, instead of replacing the old version, we run both versions in
parallel. In our example, consider that we are using \mx to run a
version of \lighttpd from March 2009.  When the buggy April 2010 version
is released, \mx runs it in parallel with the old one.  As the two
versions execute:

\begin{itemize}
\item As long as the two versions have the same external behaviour (\eg they 
write the same values into the same files, or send the same data over
the network), they are run side-by-side and \mx ensures that they act
as one to the outside world (see \S\ref{sec:mxm});

\item{When one of the versions crashes (\eg the new version executes 
the buggy patch), \mx will patch the crashing version at runtime using
the behaviour of the non-crashing version 
(see \S\ref{sec:rem})}.  In this way, \mx can successfully survive
crash bugs in both the old and the new version, increasing the
reliability and availability of the overall application;\looseness=-1

\item When a non-crashing divergence is detected, \mx will discard one of
the versions (by default the old one, but other heuristics can be
used).  The other version can be later restarted at a convenient
synchronisation point (\eg at the beginning of the dispatch loop of
a network server).

\end{itemize}

%% When the two versions diverge because of the newly introduced bug
%% and the April 2010 version crashes, \mx will patch the crashing
%% version at runtime using the behaviour of the non-crashing
%% version. While the recovered version continues to execute, \mx uses
%% the correctly executing March 2009 version as an oracle to ensure
%% that its behaviour is correct. When any divergence is detected, \mx
%% will discard the recovered version and continue using only the old
%% version to ensure correctness. If the divergence is detected
%% during the normal execution, \mx will by default prefer the
%% behaviour of the newer version, but other heuristics can be used as
%% well.

From the user's point of view, this process is completely transparent
and does not cause any interruption in service. In our example, this
effectively eliminates the bug in \lighttpd, while still allowing
users to use the latest features and bug fixes of the recent versions.

%In our proposed approach, when a new version arrives, instead of
%replacing the old version, we run both versions in parallel.  As more
%versions arrive, we execute them in parallel with the existing ones,
%until all available resources have been exhausted, at which point we
%discard some of the versions according to some strategy.

%In our example, consider a system that is running a version
%of \lighttpd from March 2009.  When the buggy April 2010 version is
%released, our system runs it in parallel with the old one.  As the two
%versions execute, the system checks that their external behaviour is
%identical (\eg they write the same values into the same files, or send
%the same data over the network).  When the two versions diverge, the
%divergence is resolved in the favour of the more reliable version.  In
%particular, if one of the two versions crashes, the behaviour of the
%non-crashing version is used, and the other version is transparently
%modified to survive the crash. (Note that the last point is of key
%importance, as the success of our technique depends on having all
%versions running at all times.)  If the system cannot determine which
%behaviour is correct, a simple heuristic can be used, such as always
%preferring the behaviour of the newer version.  In our example, this
%effectively eliminates the bug in \lighttpd, while still allowing
%users to use the latest features and bug fixes of the recent versions.

%Figure~\ref{fig:lighttpd-history}, 


%% Bug 1800: not really using it
%%
%% The bug \#1800 was also related to the HTTP compression and ETag functionality; in
%% particular, invalid ETags were generated for compressed variants of the same
%% resource. The solution for this bug did not consider the situation when
%% ETag support is completely disabled. However, due to an incorrect implementation
%% of ETag hash function, this bug remain undetected until the revision
%% \texttt{2438} when ETag hash function has been fixed.

%% The \texttt{mod\_compress} module implementing the HTTP compression support
%% uses caching to avoid re-compression of files which has been already
%% compressed in the past. To determine whether the cached file is still valid,
%% \texttt{mod\_compress} internally uses ETag stored along with the file.

%% Then, when request for a file is made, \texttt{mod\_compress} module
%% implementation shown in Listing~\ref{lst:2386} first reads the physical file
%% along with its ETag (line 4) and tries to match this ETag with the original
%% one (line 9). However, if ETag support has been disabled, the ETag of the
%% physical file would be empty.

%% \begin{lstlisting}[label=lst:2386,caption={Refactored failing version of the function}]
%% PHYSICALPATH_FUNC(mod_compress_physical) {
%%   stat_cache_entry *sce = NULL;
%%   ...
%%   if (HANDLER_ERROR == stat_cache_get_entry(srv, con, con->physical.path, &sce)) {
%%     ...
%%   }
%%   ...
%%   /* try matching original etag of uncompressed version */
%%   etag_mutate(con->physical.etag, sce->etag);
%%   ...
%% }
%% \end{lstlisting}

%% The original code, which can be seen in Listing~\ref{lst:2437}, makes the
%% \texttt{i < etag->used} comparison (line 4). Because both \texttt{i} and
%% \texttt{etag->used} are $0$, the condition \texttt{0 < 0} does not hold and
%% the loop body will never be executed.

%% The affected code can be seen in Listing~\ref{lst:2438}. Here, the condition
%% has been changed and comparison has now the form \texttt{i < etag->used-1}.
%% When executing, the \texttt{etag->used} variable will underflow and condition
%% \texttt{0 < 4294967295} will be true. Therefore, the loop body will be
%% executed and access to \texttt{etag->ptr[0]} will result in segmentation
%% fault.

%% \begin{lstlisting}[label=lst:2437,caption={Original version of \texttt{etag\_mutate} function}]
%% int etag_mutate(buffer *mut, buffer *etag) {
%%   size_t i;
%%   uint32_t h;
%%   for (h=0, i=0; i < etag->used; ++i) h = (h<<5)^(h>>27)^(etag->ptr[i]);
%%   buffer_reset(mut);
%%   buffer_copy_string_len(mut, CONST_STR_LEN("\""));
%%   buffer_append_long(mut, h);
%%   buffer_append_string_len(mut, CONST_STR_LEN("\""));
%%   return 0;
%% }
%% \end{lstlisting}

%% \begin{lstlisting}[label=lst:2438,caption={Modified version of \texttt{etag\_mutate} function}]
%% int etag_mutate(buffer *mut, buffer *etag) {
%%   size_t i;
%%   uint32_t h;
%%   for (h=0, i=0; i < etag->used-1; ++i) h = (h<<5)^(h>>27)^(etag->ptr[i]);
%%   buffer_reset(mut);
%%   buffer_copy_string_len(mut, CONST_STR_LEN("\""));
%%   buffer_append_long(mut, h);
%%   buffer_append_string_len(mut, CONST_STR_LEN("\""));
%%   return 0;
%% }
%% \end{lstlisting}

\section{Discussion}
\label{efficient-execution:discussion}

This section discusses some of the implications of \varan's design,
including its main limitations, many of which are inherent to all
existing NVX systems.

\paragraph{CPU utilisation and memory consumption.} The performance
evaluation reported in Section~\ref{sec:evaluation} considers the
overhead in terms of throughput or clock time.  However, an NVX
framework introduces a CPU utilisation overhead linear in the number
of versions.  While this might be a serious concern in some scenarios,
leaving cores idle has a cost as well~\cite{barroso2007} and in many
cases idle cores can be profitably used to increase software reliability
and security~\cite{cox2006,multiplicity,orchestra09,diehard06,mvupdates12}.

Similarly, the memory overhead imposed by \varan is linear in the
number of versions, as in prior NVX systems.  This can lead to
degradations in performance due to memory pressure and caching
effects, as we have observed in Section~\ref{sec:comparison}.


\paragraph{Memory-based communication.} As prior NVX systems, \varan
does not support memory-based communication.  More exactly, \varan
only allows files to be mapped into memory as read-only---if the file
would be mapped as read-write, any writes by the leader would likely
lead to divergences in followers, as they would read the value written
by the leader rather than the original value.  This limitation comes
from the fact that memory-based communication cannot be intercepted by
interposing upon the system call interface, and as such is invisible
to NVX systems operating at the system call level.


\paragraph{Synchronisation.} While \varan supports multi-threaded and
multi-process applications (\S\ref{sec:threading}), there is a
potential issue with synchronisation primitives implemented entirely
in user space, as these primitives will be invisible to \varan. While
it is possible to use entirely user-space synchronisation primitives,
in our experience, they are not that frequent and standard
synchronisation primitives combine atomics with system calls (\ie
futex). We have not observed any related problems in our concurrent
benchmarks (\S\ref{sec:c10k}, \S\ref{sec:comparison}).

\paragraph{Security.} Although our focus with \varan has been on
improving software reliability, \varan could be also used to implement
existing NVX security defences~\cite{cox2006,orchestra09}.  However,
there are two additional problems that \varan introduces, as discussed
below.

First, the use of buffering, while essential for improving
performance, leads to delayed detection of divergences, providing
attackers with a window of opportunity in which to perform malicious
system calls.  However, \varan's buffer size is configurable, and
could be set to one to disable buffering. Even without buffering,
\varan's binary rewriting mechanism is more efficient than
\lstinline`ptrace`-based solutions.

Second, since \varan resides in the same address space as the
application, a return-oriented programming (ROP) attack can bypass
\varan's tracing mechanism and thus escape detection.  Furthermore,
\varan's code could be a primary target of such an attack. However,
this is partially mitigated by the fact that \varan's code is loaded
at a random memory address.

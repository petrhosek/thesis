\subsection{Comparison with prior NVX systems}
\label{sec:comparison}

While \S\ref{sec:microbenchmarks} and \S\ref{sec:c10k} illustrate the
worst-case synthetic and real-world scenarios for a system-call
monitor, in order to compare \varan directly with prior NVX systems,
we have also run it on the same set of benchmarks used to evaluate
prior systems.  In particular, we chose to compare against three %four
state-of-the-art NVX systems: %\strata~\cite{cox2006},
\orchestra~\cite{orchestra09}, \mx~\cite{mx} and
\tachyon~\cite{tachyon12}.  These systems and their benchmarks are
briefly described  in the first three columns of Table~\ref{tbl:eval-systems}.
To our knowledge, we are the first to perform an extensive performance
comparison of existing NVX systems.

\begin{table}[t]
	\centering
\begin{tabular}{lllrr}
  \toprule
  \textsc{System} & \textsc{Mechanism} & \textsc{Benchmarks}  & \textsc{Overhead} & \varan \\
  %\hline
  %\strata~\cite{cox2006} & modified kernel & \httpd (WebBench 5.0) & \strataHttpd & \httpdWebBenchOneFollower  \\
  \midrule
  \orchestra~\cite{orchestra09} & ptrace & \httpd (ApacheBench)    & \orchestraHttpd & \httpdAbOneFollower  \\
                                &        & \speczerozero & \orchestraSpec & \speczerozeroOneFollower \\
  \midrule
  \tachyon~\cite{tachyon12} & ptrace & \lighttpd (ApacheBench) & \tachyonLighttpd & \lighttpdAbOneFollower \\
                            & & \thttpd (ApacheBench) & \tachyonThttpd & \thttpdOneFollower \\
  \midrule
  \mx~\cite{mx} & ptrace & \lighttpd (http\_load) & \mxLighttpd & \lighttpdHttploadOneFollower \\
                      &  & \redis (redis-benchmark) & \mxRedis & \redisOneFollower \\
                      &  & \speczerosix & \mxSpec & \speczerosixOneFollower \\
  \bottomrule
\end{tabular}
	\caption{Existing systems we have compared \nx against.}
	\label{tbl:eval-systems}
\end{table}

Since prior systems only handle two versions, we also run \varan with
two versions only.  However, we remind the reader that one of the
strengths of \varan's decentralized architecture is that it can handle
multiple versions with minimum additional overhead.

\begin{figure}[!t]
  \centering
  \includegraphics[width=\columnwidth]{efficient-execution/graphs/spec2000.pdf}
  \caption{\speczerozero performance overhead for different number of followers.}
  \label{fig:spec2000}
\end{figure}

\boldtext{\speczerozero} %\footnote{\url{http://www.spec.org/cpu2000/}}
was used to evaluate \orchestra. We used the latest available version
1.3.1.  The cummulative results are reported in
Table~\ref{tbl:eval-systems}, as for all other benchmarks.  \orchestra
reported \orchestraSpec overhead when running two instances of
\speczerozero in parallel, while \nx introduced only
\speczerozeroOneFollower overhead. The results for individual
applications contained in the \speczerozero suite and for different
number of followers can be seen in Figure~\ref{fig:spec2000}.

\boldtext{\speczerosix} %\footnote{\url{http://www.spec.org/cpu2006/}}
has been previously used by \mx. We used the latest version 1.2.  The
overhead reported by \mx was \mxSpec, while \nx introduced only
\speczerosixOneFollower overhead.
%in contrast, \nx introduced only \speczerosixOneFollower.
Individual results can be seen in Figure~\ref{fig:spec2006}.

\boldtext{\httpd} %\footnote{\url{https://httpd.apache.org/}}
was used by \orchestra.  We used version 1.3.29, same as in the
original work~\cite{orchestra09}.  The overhead reported for
\orchestra is \orchestraHttpd using the \emph{ApacheBench}
benchmark. \nx introduces \httpdAbOneFollower overhead using the same
benchmark, which is a significant improvement.
Figure~\ref{fig:servers} shows the overhead introduced by \varan for
\httpd (and the other servers used to evaluate prior work) with
different number of folllowers. 

\begin{figure}[!t]
 \centering
 \includegraphics[width=\columnwidth]{efficient-execution/graphs/spec2006.pdf}
 \caption{\speczerosix performance overhead for different number of followers.}
 \label{fig:spec2006}
\end{figure}

%was used by both \strata and \orchestra.  We used version 1.3.29
%mentioned by \orchestra (\strata did not report the version used).
%The overhead reported for \orchestra is \orchestraHttpd using the
%\emph{ApacheBench} benchmark and \strataHttpd for \strata using
%\emph{WebBench 5.0}. \nx introduces \httpdWebBenchOneFollower overhead
%using \emph{WebBench} and \httpdAbOneFollower overhead with
%\emph{ApacheBench}, which is an improvement over previous work.

\boldtext{\lighttpd} %\footnote{\url{http://www.lighttpd.net/}}
has been used to evaluate both \mx and \tachyon.  We used version
1.4.36 (neither \mx nor \tachyon reported the version used). \mx used
the \emph{http\_load} benchmark and reported \mxLighttpd overhead
while \tachyon used the \emph{ApacheBench} benchmark and reported a
\tachyonLighttpd overhead.  When benchmarked using \emph{http\_load},
\varan introduced a \lighttpdHttploadOneFollower overhead, while with
\emph{ApacheBench} it introduced no noticeable
overhead. %\lighttpdAbOneFollower.  
In both
cases, this marks a significant improvement over the previous work.

\boldtext{\thttpd} %\footnote{\url{http://www.acme.com/software/thttpd/}}
was shown to introduce \tachyonThttpd overhead when running atop \tachyon using
the \emph{ApacheBench} benchmark. When run atop \nx using the same settings
as in \cite{tachyon12}, we have not measured any noticeable overhead.

\boldtext{\redis} %\footnote{\url{http://redis.io/}}
was used in evaluation of \mx reportedly in version 1.3.8.  The performance
overhead reported by \mx was \mxRedis using the \lstinline`redis-benchmark`
utility. When ran with \nx using the same benchmark and the same workload,
the overhead we measured was \redisOneFollower, which is again a significant
improvement over the previous work.

%\begin{figure}[!t]
  %\centering
  %\includegraphics[width=\columnwidth]{results/macro_beanstalkd.pdf}
  %\caption{Beanstalkd performance overhead.}
  %\label{fig:macro_beanstalkd}
%\end{figure}

%\begin{figure}[!t]
  %\centering
  %\includegraphics[width=\columnwidth]{results/macro_lighttpd.pdf}
  %\caption{Lighttpd performance overhead.}


\subsection{Comparison with prior NVX systems}
\label{sec:comparison}

While Sections~\ref{sec:microbenchmarks} and \ref{sec:c10k} illustrate the
worst-case synthetic and real-world scenarios for a system call monitor, in
order to compare \varan directly with prior NVX systems, we have also run it on
the same set of benchmarks used to evaluate prior systems.  In particular, we
chose to compare against two state-of-the-art NVX systems:
%\strata~\cite{cox2006}, \orchestra~\cite{orchestra09}, and
\tachyon~\cite{tachyon12}.  These systems and their benchmarks are briefly
described  in the first three columns of Table~\ref{tbl:systems}.  To our
knowledge, we are the first to perform an extensive performance comparison of
existing NVX systems.

\begin{table*}[t]
\begin{center}
\caption{Existing systems we have compared \varan against.}
\label{tbl:systems}
\begin{tabular}{lllrr}
  \toprule
  \textsc{System} & \textsc{Mechanism} & \textsc{Benchmarks} & \textsc{Overhead} & \textsc{\varan} \\

  \midrule
  \mx~\cite{mx} & ptrace & \lighttpd (http\_load) & \mxLighttpd & \lighttpdHttploadOneFollower \\
                      &  & \redis (redis-benchmark) & \mxRedis & \redisOneFollower \\
                      &  & \speczerosix & \mxSpec & \speczerosixOneFollower \\
  %\hline
  %\strata~\cite{cox2006} & modified kernel & \httpd (WebBench 5.0) & \strataHttpd & \httpdWebBenchOneFollower  \\
  \hline
  \orchestra~\cite{orchestra09} & ptrace & \httpd (ApacheBench)    & \orchestraHttpd & \httpdAbOneFollower  \\
                                &        & \speczerozero & \orchestraSpec & \speczerozeroOneFollower \\
  \hline
  \tachyon~\cite{tachyon12} & ptrace & \lighttpd (ApacheBench) & \tachyonLighttpd & \lighttpdAbOneFollower \\
                            & & \thttpd (ApacheBench) & \tachyonThttpd & \thttpdOneFollower \\
  \bottomrule
\end{tabular}
\end{center}
\end{table*}

\begin{figure}[!t]
 \centering
 \includegraphics[width=\textwidth]{efficient-execution/graphs/comparison}
 \caption{Performance overhead for the \httpd, \thttpd, and \lighttpd
   servers for different numbers of followers to allow for comparison
   with existing systems.}
 \label{fig:comparison}
\end{figure}


The last two columns of Table~\ref{tbl:systems} show the cumulative
results.  Since prior systems only handle two versions, the comparison
is done against \varan configured in the same way.  However, we remind
the reader that one of the strengths of \varan's decentralised
architecture is that it can often handle multiple versions with minimum
additional overhead, and below we also show how \varan performs on
these benchmarks when more than two versions are used.

\paragraph{\httpd} \footnote{\url{https://httpd.apache.org/}}
was used by \orchestra.  We used version 1.3.29, the same as in the
original work~\cite{orchestra09}.  The overhead reported for
\orchestra is \orchestraHttpd using the \emph{ApacheBench}
benchmark. \varan introduces \httpdAbOneFollower overhead using the
same benchmark, which is a significant improvement.
Figure~\ref{fig:comparison} shows the overhead introduced by \varan
for \httpd (and the other servers used to evaluate prior work) with
different numbers of followers.  As it can be seen, \varan scales very
well with increasing numbers of followers for these benchmarks.

%was used by both \strata and \orchestra.  We used version 1.3.29
%mentioned by \orchestra (\strata did not report the version used).
%The overhead reported for \orchestra is \orchestraHttpd using the
%\emph{ApacheBench} benchmark and \strataHttpd for \strata using
%\emph{WebBench 5.0}. \varan introduces \httpdWebBenchOneFollower overhead
%using \emph{WebBench} and \httpdAbOneFollower overhead with
%\emph{ApacheBench}, which is an improvement over previous work.

\paragraph{\lighttpd} %\footnote{\url{http://www.lighttpd.net/}}
has been used to evaluate both \mx and \tachyon.  We used version
1.4.36. %(neither \mx nor \tachyon reported the version used). 
\mx used the \emph{http\_load} benchmark and reported \mxLighttpd overhead
while \tachyon used the \emph{ApacheBench} benchmark and reported a
\tachyonLighttpd overhead.  When benchmarked using \emph{http\_load},
\varan introduced only \lighttpdHttploadOneFollower overhead, while with
\emph{ApacheBench} it introduced no noticeable
overhead. %\lighttpdAbOneFollower.  
In both
cases, this marks a significant improvement over previous work.

\paragraph{\thttpd}\footnote{\url{http://www.acme.com/software/thttpd/}}
was shown to introduce \tachyonThttpd overhead when run on top of
\tachyon using the \emph{ApacheBench} benchmark. When run on top of
\varan using the same settings as in \cite{tachyon12}, we have not
measured any noticeable overhead.

\paragraph{\redis} 1.3.8 \footnote{\url{http://redis.io/}}
was used in the evaluation of \mx.  The performance
overhead reported by \mx was \mxRedis using the \lstinline`redis-benchmark`
utility. When run with \varan using the same benchmark and the same workload,
the overhead we measured was \redisOneFollower, which is again a significant
improvement over previous work.

\begin{figure*}[!t]
  \centering
  \includegraphics[width=\textwidth]{efficient-execution/graphs/spec2000}
  \caption{\speczerozero performance overhead for different numbers of followers.}
  \label{fig:spec2000}
\end{figure*}

\paragraph{\speczerozero}\footnote{\url{http://www.spec.org/cpu2000/}} 
was used to evaluate \orchestra.  We used the latest available version
1.3.1.  \orchestra reported a \orchestraSpec overhead, while \varan
introduced only a \speczerozeroOneFollower overhead. The results for
the individual applications contained in the \speczerozero suite and
for different numbers of followers can be seen in
Figure~\ref{fig:spec2000}. The reason these applications scale poorly
with the number of followers is likely due to memory pressure and
caching effects~\cite{jaleel07}, and to the fact that our machine has
only four physical cores (with two logical cores each).  We plan to
investigate these results in more detail in future work.

\begin{figure*}[!t]
  \centering
  \includegraphics[width=\textwidth]{efficient-execution/graphs/spec2006}
  \caption{\speczerosix performance overhead for different numbers of followers.}
  \label{fig:spec2006}
\end{figure*}

\paragraph{\speczerosix}\footnote{\url{http://www.spec.org/cpu2006/}}
was previously used to evaluate \mx. We used the latest version 1.2.  The
overhead reported by \mx was \mxSpec, while \varan introduced only
a \speczerosixOneFollower overhead.
Individual results can be seen in Figure~\ref{fig:spec2006}.  


%\begin{figure}[!t]
  %\centering
  %\includegraphics[width=\columnwidth]{results/macro_beanstalkd}
  %\caption{Beanstalkd performance overhead.}
  %\label{fig:macro_beanstalkd}
%\end{figure}

%\begin{figure}[!t]
  %\centering
  %\includegraphics[width=\columnwidth]{results/macro_lighttpd}
  %\caption{Lighttpd performance overhead.}


\subsection{Microbenchmarks}
\label{sec:microbenchmarks}

To measure the overhead introduced by \varan while processing individual
system calls, we designed a series of experiments that compare a
system call intercepted and executed by \varan against the same system
call executed natively. We used five different system calls:

\begin{enumerate}

%\item \lstinline`close` is representative of an inexpensive system call.
%  We invoke it with argument \lstinline`-1`, which returns immediately
%  with an error representing a NOP system call.

\item \lstinline`close(-1)` is representative of an inexpensive system call,
  which returns immediately.

%\item \lstinline`write` is representative of system calls which involve
%  expensive I/O, but whose result can be sent entirely as a single
%  event in the ring buffer.  We invoke the system call to write a
%  512-byte buffer to \lstinline`/dev/null`.

\item \lstinline`write(DEV_NULL, ..., 512)` is representative of system
  calls which involve expensive I/O, but whose result can be sent
  entirely as a single event in the ring buffer.

%\item \lstinline`read` is representative of system calls which involve
%  expensive I/O, and whose result cannot be fully included in the
%  associated event in the ring buffer.  Instead, it has to be copied
%  via additional shared memory (\S\ref{sec:ring}). We invoke the
%  system call to read a 512-byte buffer from \lstinline`/dev/zero`.

\item \lstinline`read(DEV_NULL, ..., 512)` is representative of system calls which
  involve expensive I/O, and whose result cannot be fully included in the
  associated event in the ring buffer.  Instead, it has to be copied via
  additional shared memory (\S\ref{sec:ring}).

%\item \lstinline`open` is representative of system calls that require
%  transferring file descriptors (\S\ref{sec:leader-repl}).  We invoke
%  it with \lstinline`/dev/null` and \lstinline`O\_RDONLY` as arguments.

\item \lstinline`open("/dev/null", O_RDONLY)` is representative of system calls that
  require transferring file descriptors (\S\ref{sec:leader-repl}).

\item \lstinline`time(NULL)` is a virtual system call implemented via the
  \textit{vDSO} segment (\S\ref{sec:vsyscall}). It internally calls
  \lstinline`__vdso_time` (since glibc 2.15).  We could not measure
  the overhead of using the \textit{vsyscall} page, because it is
  deprecated on our system (and all recent versions of Linux), with
  all \textit{vsyscalls} now redirected to their \textit{syscall}
  versions.

\end{enumerate}

%% The \lstinline`open` demonstrates the cost of the file descriptor transfer; we have
%% used the \lstinline`/dev/null` file with \lstinline`O\_RDONLY` flag. \lstinline`read` and
%% \lstinline`write` use the I/O performance overhead, by reading and writing 512B
%% buffer from \lstinline`/dev/zero` and to \lstinline`/dev/null` respectively. The \lstinline`read`
%% is clearly more expensive system call, as it needs to transfer the content of
%% the buffer read using the shared memory (\S\ref{sec:streaming}). The
%% \lstinline`close` system call is a representative of all other cases.

We executed each system call one million times and computed the average
of all execution times.  Time measurements were done using the time
stamp counter (\ie the \lstinline`RDTSC` instruction). Each set of
measurements was preceded by a warm-up stage in which we executed the
system call \num{10000} times. % to warm up the caches.

\begin{figure}[!t]
  \centering
  \includegraphics[width=\textwidth]{efficient-execution/graphs/micro}
  \caption{System call microbenchmarks.}
  \label{fig:micro_syscall}
\end{figure}

%\begin{figure}[!t]
%  \centering
%  \includegraphics[width=\columnwidth]{results/micro_syscall}
%  \caption{System call microbenchmarks.}
%  \label{fig:micro_syscall}
%\end{figure}

Figure~\ref{fig:micro_syscall} shows the results.  The first set of
bars labeled \textit{native} shows the execution time without \varan.
The second set of bars labeled \textit{intercept} shows the execution
time with interception, measuring the cost of binary rewriting: for
these experiments, the intercepted system call is immediately
executed, without any additional processing.  As it can be seen, the
interception cost is small, at under \maxInterceptOvh in all
cases except for \lstinline`time`. The overhead of intercepting virtual system calls is high
in relative terms, but low in absolute ones: \vdsoIntercept vs
\vdsoNative for native execution for \lstinline`time`.

The set of bars labeled \textit{leader} shows the execution time for
each system call to be intercepted, executed and recorded by the
leader.  That is, it is the sum of the \textit{intercept} cost and the
cost of recording the system call.  For \lstinline`close` and \lstinline`write`,
the overhead is only \closeLeaderOvh and \writeLeaderOvh respectively
on top of native execution, because the arguments and results of
these system calls can be recorded in a single event.  For \lstinline`read`,
it is more expensive, at \SI{139}{\percent}, because transferring the result also
involves accessing additional shared memory.  Finally, the cost for
\lstinline`open` is the highest, since it also involves the slower transfer
of the returned file descriptor via a UNIX domain socket.

Finally, the set of bars labelled \textit{follower} shows the execution
time of the follower, which has to intercept each system call and read
its results from the ring buffer and (if necessary) shared memory.  As
expected, the costs for \lstinline`close` and \lstinline`write` are low (and
significantly lower than executing the system call), because the
entire result fits into a single event on the ring buffer. The costs
for \lstinline`read` and \lstinline`open` are higher, because they involve
additional shared memory and transferring a file descriptor,
respectively, but they are still lower than the costs incurred by the
leader.


%\boldtext{vDSO calls.}  We similarly measured the overhead of
%performing a virtual system call via the \textit{vDSO} segment
%(\S\ref{sec:vsyscall}).  The system call used was \lstinline`time`,
%which internally calls \lstinline`\_\_vdso\_time` (since \stt{glibc
%  2.15}).  We could not measure the overhead of using the
%\textit{vsyscall} page, because it is deprecated on our system (and
%all recent versions of Linux), with all \textit{vsyscalls} now
%redirected to their \textit{syscall} versions.

%\begin{figure}[!t]
%  \centering
%  \includegraphics[width=\columnwidth]{results/micro_vsyscall}
%  \caption{vsyscall and vDSO microbenchmarks.}
%  \label{fig:micro_vsyscall}
%\end{figure}

%Figure~\ref{fig:micro_syscall} shows the results, using the same set
%of bars as in Figure~\ref{fig:micro_syscall}.  We have opted this time
%to show absolute numbers, and to include the data for \lstinline`close(-1)`
%from Figure~\ref{fig:micro_syscall} for comparison.  The overhead of
%intercepting the system call is high in relative terms, but low in
%absolute ones: \vdsoIntercept vs \vdsoNative for native execution.
%The \textit{leader} and \textit{follower} costs are similarly high in
%relative terms but low in absolute ones: \vdsoLeader and
%\vdsoFollower, respectively.

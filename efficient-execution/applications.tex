\section{Applications}
\label{sec:applications}

Same as for other NVX systems, the primary application of \varan is
\emph{transparent failover}, \ie the ability to transparently switch over to
the non-crashing version in the event of failure without any disruption in the
service. Furthemore, as an execution runtime targeted towards multi-version
execution applications, \varan also supports running multiple different
software versions in parallel.  We discuss both of these scenarios in detail
below.

\subsection{Transparent failover}
\label{sec:failover}

%[Cite \cite{rx,fo,mx}]

NVX systems introduce a variety of opportunities for increasing
software reliability and availability via transparent failover.  For
instance, one can run in parallel multiple variants of an application with
different memory layouts, different software revisions or different
implementations of a given interface to survive bugs that occur 
only in some of them.   

\varan makes it easy to implement transparent failover.  When one of the
versions crashes, the \lstinline`SIGSEGV` signal handler installed in each
version notifies the coordinator, which decides what restart strategy
to use.  When one of the followers crashes, the coordinator unsubscribes it
from the list of currently-running followers, and discards it without
affecting other followers.  When the leader crashes, it designates one
of the followers as the new leader (currently the one with the
smallest internal ID), and notifies it to switch its system call table
(\S\ref{sec:rewriting}) to that of the leader, and to restart the last
system call while discarding the old (crashed) leader.

To demonstrate support for transparent failover, we reproduced a
\redis
bug\footnote{\url{https://code.google.com/p/redis/issues/detail?id=344}}
described in Section~\ref{multi-version:scenarios}.  We ran in
parallel eight consecutive revisions of \redis from the range
\lstinline`9a22de8` to \lstinline`7fb16ba`, where the last revision
introduced a bug which crashes the server by causing a segmentation
fault. We then set up a client to send an \lstinline`HMGET` command
that triggers the bug, and measured the increase in latency for that
command.  When the buggy version is a follower, we do not observe any
increase in latency, as expected.  When the buggy version is the leader,
the latency increases from \redisnormallatency to
\redisfailoverlatency.  In both cases, we observed no extra
degradation in throughput for the commands that follow.

While \mx can recover from the crash unlike \varan, which discards the crashing
version, \mx can only run two versions in parallel while \varan can run
arbitrary number of versions with significantly lower performance overhead as
shown in Section~\ref{sec:comparison}.

As an additional experiment, we ran revisions \lstinline`2437` and
\lstinline`2438` of \lighttpd (also used in the evaluation of \mx),
the latter of which introduced a crash bug.  We then set up a client
that triggers the bug and measured the latency for that request. Both when the
buggy version was the leader or a follower, there was no significant increase
in latency, which remained at around~\lighttpdnormallatency.
%set up a client that triggers the bug and measured the increase in
%latency for that command. As for the \redis scenario, there is no
%increase in latency when the buggy version is a follower.  When it is
%the leader, the latency increases from \lighttpdnormallatency to
%\lighttpdfailoverlatency.  

\subsection{Multi-version execution}
\label{sec:mv-execution}

Multiple different software versions (revisions) can be run inside an NVX
system as long as they all issue the same sequence of system calls. This
limitation is due to the fact that prior NVX systems run versions in lockstep
(\S\ref{sec:rw}).

Because \varan does not run the versions in lockstep and can use system call
rewrite rules, it can often overcome this limitation. To illustrate, we used
several \lighttpd revisions from our evolution study
(\S\ref{sec:behavior-evolution}) which introduced new system calls and as such
cannot be run in parallel by prior NVX systems that rely on lockstep execution.

As a first experiment, we ran revision \lstinline`2435` as
leader together with revision \lstinline`2436` as follower.  Revision
\lstinline`2436` introduces two additional checks using the
\lstinline`getuid` and \lstinline`getgid` system calls.  More
precisely, revisions until and including \lstinline`2435` used
\lstinline`geteuid()` and \lstinline`getegid()` C library functions to
check the user account under which the server is being run, before
issuing an \lstinline`open` system call.  This resulted in a sequence
of \lstinline`geteuid`, \lstinline`getegid` and \lstinline`open`
system calls.  Revision \lstinline`2436` replaced the use of the
aformentioned functions with \lstinline`issetugid()` which changed the
system call sequence to \lstinline`geteuid`, \lstinline`getuid`,
\lstinline`getegid`, \lstinline`getgid`, followed by \lstinline`open`
as before.

To allow for this divergence, we used the custom BPF filter shown in
Listing~\ref{lst:lighttpd}.  The filter is executed by the
follower whenever a divergence is detected.  In our experiment, this
happens when the follower executes the newly introduced
\lstinline`getuid` system call. The filter first loads the system call
number executed by the leader into the implicit BPF accumulator
(line~\ref{line:load-leader}) and checks whether the call is either
\lstinline`getegid` (line~\ref{line:check-getegid}) or
\lstinline`open` (line~\ref{line:check-open}).  The former will be
true in this case, so control will transfer to
line~\ref{line:load-getuid}, which loads the system call number
executed by the follower into the accumulator, checks whether it is
\lstinline`getuid` (line~\ref{line:check-getuid}) and finally
transfers control to line~\ref{line:allow} returning the value
\lstinline`SECCOMP_RET_ALLOW`, which instructs \varan to execute the
additional system call (\ie \lstinline`getuid`) in the follower.  Any
other combination of system calls would have killed the follower
(line~\ref{line:kill}). After executing the \lstinline`getuid` system
call and replaying the execution of \lstinline`getegid` (which the
leader also executed), \varan would detect a second divergence when
the follower tries to execute \lstinline`getgid` instead of
\lstinline`open`. This divergence would be resolved using the same
filter, taking the path on lines~\ref{line:check-open},
\ref{line:load-getgid}, \ref{line:check-getgid} and \ref{line:allow}.

Note this is only one possible filter for allowing this divergence; in
particular, one could write a filter that takes into account more
information about the context in which it should be applied, \eg by
inspecting some system call arguments.

% or more system calls preceding the divergence.


\begin{figure}[t]
\begin{lstlisting}[label={lst:lighttpd},language={[bpf]Assembler},caption={Example of a BPF rewriting rule.}]
ld event[0] /*@\label{line:load-leader}@*/
jeq #108, getegid       /* __NR_getegid *//*@\label{line:check-getegid}@*/
jeq #2, open            /* __NR_open *//*@\label{line:check-open}@*/
jmp bad
getegid:
ld [0]                  /* offsetof(struct seccomp_data, nr) *//*@\label{line:load-getuid}@*/
jeq #102, good          /* __NR_getuid *//*@\label{line:check-getuid}@*/
open:
ld [0]                  /* offsetof(struct seccomp_data, nr) *//*@\label{line:load-getgid}@*/
jeq #104, good          /* __NR_getgid *//*@\label{line:check-getgid}@*/
bad: ret #0             /* SECCOMP_RET_KILL *//*@\label{line:kill}@*/
good: ret #0x7fff0000   /* SECCOMP_RET_ALLOW *//*@\label{line:allow}@*/
\end{lstlisting}
\end{figure}

%% To translate the filter into bytecode, the user can use the provided
%% \lstinline`vx_bpf` tool and pass the output directly to \varan when starting the
%% application:

%% \begin{lstlisting}[language={bash},numbers=none]
%% vx --filter "$(vx_bpf lighttpd.filter)" \
%%   /path/lighttpd-2435 /path/lighttpd-2436 \
%%   -- -f lighttpd.conf
%% \end{lstlisting}

We used a similar filter to run revisions \lstinline`2523` and
\lstinline`2524`, the latter of which introduces an additional
\lstinline`read` system call to access the \lstinline`/dev/urandom`
file to obtain an additional source of entropy.  We were also able to
run revisions \lstinline`2577` and \lstinline`2578` where the
difference consists of an additional \lstinline`fcntl` system call to
set a \lstinline`FD_CLOEXEC` flag on one of the file descriptors.

Currently, \varan's implementation can use BPF filters only to allow adding
or removing system calls in followers.  However, this is not a
fundamental limitation, and in the future we plan
%to use the powerful pseudo-machine model employed by BPF filters
to support  other types of transformations, such as replacing one
sequence of system calls with another.


% and then notifies all versions by issuing a special
% "reranking" event and sends new ranks to all instances over the
% service channel. When the event is read by individual followers during
% replay, they receive their new ranks and take the action. 


% \todo{Memory consumption?}

% \begin{structure}
% \item Scaling w/ more versions/variants
% \item Fail-over mechanism w/ many versions/variants
% \item Multiple versions: \mx, staged deployment
% \item Variants: sanitizers? some memory allocation diffs?
% \item different compilers?  gcc vs llvm?
% \item sanitizers?
% \end{structure}

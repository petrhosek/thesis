\chapter{Related Work}
\label{chap:related}

This chapter provides an overview of previous work in the area.  We first
discuss other scenarios which involve the execution of multiple program
versions, and then discuss related work in area of $N$-version programming,
dynamic software updates, sandboxing and record \& replay. These approaches
and their common use cases are summarized in Table~\ref{tbl:taxonomy}.

\begin{table}[!h]
\begin{center}
\caption{Taxonomy of related work which summarises the existing approaches
and compares them with respect to each other.}
\label{tbl:taxonomy}
\begin{tabularx}{\textwidth}{>{\hsize=1.5\hsize}X>{\hsize=0.5\hsize}X>{\hsize=1.0\hsize}X}
\toprule
\textsc{Summary} & \textsc{Use case} & \textsc{Characteristic} \\
\midrule
\textsc{N-version programming} \newline Running multiple software versions in parallel & Availability \newline Reliability & Number of possible applications with higher resource usage \\
\midrule
\textsc{Dynamic software updates} \newline Applying software updates at runtime & Reliability \newline Maintenance & Resource efficient operation but requires intrusive changes \\
\midrule
\textsc{Sandboxing} \newline Isolating software and its effects from the system & Security & Simple and effective solution but limited to fault isolation \\
\midrule
\textsc{Record \& replay} \newline Recording the execution for future replay & Debugging \newline Testing & Asynchronous in nature restricting the possible scenarios \\
\bottomrule
\end{tabularx}
\end{center}
\end{table}

%\begin{table}[!ht]
%\begin{center}
%\caption{Taxonomy of related work which summarises the existing approaches
%and compares them with respect to each other.}
%\label{tbl:taxonomy}
%\begin{tabularx}{\textwidth}{XXXXX}
%\toprule
%& N-version programming & Dynamic software\newline updates & Sandboxing & Record \& replay \\
%\midrule
%N-version programming & -- & Easier deployment & Limited security guarantees & Higher resource usage \\
%Dynamic software\newline updates & Lower resource usage & -- & Proactive rather than reactive approach & Lower resource usage \\
%Sandboxing & Better security & Limited applicability & -- & Minimal latency \\
%Record \& replay & Wider set of applications & Easier deployment & Reproducibility and debuggability & -- \\
%\bottomrule
%\end{tabularx}
%\end{center}
%\end{table}

\section{$N$-version programming}
\label{related:nvp}

The original idea of concurrently running multiple versions of the same
application was first explored in the context of $N$-version programming, a
software development approach introduced in the 1970's in which multiple teams
of programmers independently develop functionally equivalent versions of the
same program in order to minimise the risk of having the same bugs in all
versions~\cite{chen1995}. During runtime, these versions are executed in
parallel on top of an $N$-version execution (NVX) environment in order to
improve the fault tolerance of the overall system. Both the version generation
and the synchronization mechanism required manual effort.

% More moder instantiation of N-version programming paradigm is
A possible realization of the original N-version programming paradigm is
Cocktail~\cite{cocktail}, which proposes the idea of running different web
browsers in parallel under the assumption that any two of them are unlikely to
be vulnerable to the same attacks. Compared to \mx and \varan, Cocktail's task
is simplified by exclusively targeting web browsers, which implement common web
standards, while both \mx and \varan support arbitrary off-the-shelf
applications.

%system and majority voting is used to continue in the best possible way when a
%divergence occurs.

%N-version programming was introduced in the seventies by Chen and
%Avizienis~\cite{chen1995}.  The core idea was to have multiple teams
%of programmers develop the same software independently and then run
%the produced implementations in parallel in order to improve the fault
%tolerance of the overall system.  Both version generation and the
%synchronization mechanism required manual effort.

\subsection{Multi-variant execution}

Cox \etal~\cite{cox2006} proposed a general framework for increasing
application security by running in parallel several automatically-generated
diversified variants of the same program---when a divergence is detected, an
alarm is raised. The technique was implemented in two prototypes, one in which
the variants are run on different machines, and one in which they are run on
the same machine and synchronised at the system call level, using a modified
Linux kernel. By using multiple variants with potentially disjoint exploit
sets, the proposed approach helps applications survive certain classes of
security vulnerabilities such as buffer overflows, as attackers would need to
simultaneously compromise all variants.

Berger \etal~\cite{diehard06} described a different approach which uses heap
over-provisioning and full randomisation of object placement and memory reuse
to run multiple replicas and reduce the likelihood that a memory error will
have any effect. The goal of their work is to increase reliability by
tolerating certain memory errors in exchange for space cost and execution time.

%A different solution described by Shye \etal in~\cite{shye2009} runs multiple
%instances of the same version of a single application aiming to overcome
%transient failures by monitoring and comparing their execution on the level of
%kernel system calls.  However, this solution does not aim to overcome such
%failures.

Within this paradigm, the Orchestra framework~\cite{orchestra09} used a
modified compiler to produce two versions of the same application with stacks
growing in opposite directions, runs them in parallel on top of an unprivileged
user-space monitor, and raises an alarm if any divergence is detected to
protect against stack-based buffer overflow attacks.  This work has been
further improved in~\cite{orchestra11} with a new method for signal delivery
accompanied with detailed analysis of the benchmark characteristics. However,
their approach is still limited to two versions running in lockstep with the
only difference being the opposite growing stack.

Trachsel \etal~\cite{trachsel10} used the multi-variant execution approach to
increase performance, where program variants are obtained by using different
(compiler) optimisations and algorithms.  The goal is to increase the overall
system performance by always selecting the variant which finishes its execution
first. Thereby, no synchronisation across variants is needed.

More recently, researchers have proposed additional multi-variant techniques
that fit within the $N$-version programming paradigm, \eg by employing
complementary thread schedules to survive concurrency
errors~\cite{compl-schedules11}, or using genetic programming to automatically
generate a large number of application variants that can be combined to reduce
the probability of failure or improve various non-functional
requirements~\cite{gismoe}. Yumerefendi \etal~\cite{tightlip} run multiple
sandboxed copies of the same process to detect potential privacy leaks and to
enforce access control and privacy management; Shye \etal~\cite{shye2009} use
multiple instances of the same application in order to overcome transient
hardware failures. Cadar \etal have also argued that automatically generated
software variants are a good way for exploiting the highly parallel nature of
modern hardware platforms~\cite{multiplicity}.

%Also ManySAT~\cite{manysat} %% \todo{\ldots missing rest of the sentence}

There are two key differences between our approach and the work discussed
above. First, we do not rely on automatically-generated variants, but instead
run in parallel existing software versions, which raises a number of different
technical challenges.  This means that in our case, the variants are not
semantically equivalent. This eliminates the challenge of generating
diversified variants and creates opportunities in terms of recovery from
failures, but also introduces additional challenges in terms of synchronising
the execution of the different versions.  Second, prior work has mostly focused
on detecting divergences, while our main concern is to survive them, in order
to increase the availability of the overall application.

%Recent work on NVX systems has moved in the direction of
%opportunistically using existing versions---\eg different browser
%implementations~\cite{cocktail} or different software
%revisions~\cite{mx}---or automatically synthesizing them---\eg by
%varying the memory layout~\cite{cox2006,diehard06}, or the direction
%of stack growth~\cite{orchestra09}.  Significant effort has also been
%expended on running multiple versions in parallel in the context of
%online and offline
%testing~\cite{back-to-back90,onlinevalidation,bandaid-patch07,tachyon12}.
%\nx targets NVX systems that use
%system-call level synchronization and is oblivious to the way in which
%the versions are generated.

\subsection{Multi-version execution}

Cook and Dage~\cite{cook:icse99} proposed a multi-version framework for
upgrading components. Users formally specify the specific input subdomain that
each component version should handle, after which versions are run in parallel
and the output of the version whose domain includes the current input is
selected as the overall output of the computation. The system was implemented
at the level of leaf procedures in the Tcl language. The key difference with
\mx and \varan is that this framework requires a formal description of what
input domain should be handled by each version; in comparison, \mx and \varan
target crash bugs and are fully automatic.  Moreover, \mx's goal is to have all
versions alive at all times, so crash recovery plays a key role.  Finally, both
\mx and \varan have to carefully synchronise access to shared state, which is
not an issue at the level of Tcl leaf procedures.

%\subsection{Software testing}

The multi-version execution approach has been in the past successfully used for
testing. Back-to-back testing~\cite{back-to-back90}, where the same input is
sent to different variants or versions of an application and their outputs
compared for equivalence, has been used since the 1970s.  Band-aid
patching~\cite{bandaid-patch07} proposed an online patch testing system that
splits execution before a patch, and then retroactively selects one code
version based on certain criteria.  More recently, delta
execution~\cite{onlinevalidation} proposed to run two different versions of a
single application, splitting the execution at points where the two versions
differ, and comparing their behaviour to test the patch for errors and validate
its functionality.  Similarly, Tachyon~\cite{tachyon12} is an online patch
testing system
%developed in recent independent work 
in which the old and the new version of an application are run concurrently;
when a divergence is detected, the options are to either halt the program, or
to create a manual rewrite rule specifying how to handle the divergence.

The idea of running multiple executions concurrently has also been used in an
offline testing context.  For instance, d'Amorin \etal~\cite{delta-exec-oop}
optimise the state-space exploration of object oriented code by running the
same program on multiple inputs simultaneously, while Kim
\etal~\cite{shared-exec12} improve the testing of software product lines by
sharing executions across a program family.

By comparison with this body of work, our focus is on managing
divergences across software versions at runtime in order to keep the
application running, and therefore runtime deployment and automatic
crash recovery play a central role in both \mx and \varan.

\section{Software updates}

Closely related to the execution of multiple versions is the management of
multiple versions, environment and software updates, such as deciding when to
upgrade, applying updates, \etc

Dynamic software updating (DSU) systems are concerned with the problem of
updating programs while they are running. These systems have been implemented
both in kernel-space~\cite{k42,dynamos,ksplice,proteos} and in
user-space~\cite{opus,ginseng,polus,upstare,ekiden,kitsune}.

Previous work on improving the software update process has looked at different
aspects related to managing and deploying new software versions.

\subsection{Dynamic software updates}

The dynamic update process typically consists of two stages---code update and
state transfer---with different approaches available.

Ginseng~\cite{ginseng} and K42~\cite{k42} employ indirection to enable the code
update. This simplifies the code update mechanism, but the indirection
introduces performance overhead during normal execution. Ginseng uses a
specialized compiler which creates a ``dynamically updateable program'' where
all function symbols are replaced with function pointers and all direct
function calls are replaced with indirect call through these pointers. During
update, these pointers are updated to point to the new version of the code,
which is loaded using \lstinline`dlopen`. K42 operating system has a
per-address-space object translation table used for all object invocations.
When the new kernel module is loaded, the translation table entries are updated
to point to the new version.

OPUS~\cite{opus}, DynAMOS~\cite{dynamos}, POLUS~\cite{polus} and
Ksplice~\cite{ksplice} use binary rewriting instead of indirection to replace
the entry point of the functions being updated with a jump to a trampoline.
Compared to indirection, this approach does not require the use of a
specialized compiler, but it is highly platform-dependent. We use the same
approach in \varan to handle virtual system calls (\S\ref{sec:vsyscall}).

The aforementioned systems treat individual functions or objects as the unit of
code for updates. Such systems are incapable of handling functions that rarely
end (\eg \lstinline`main` or functions that contain event-handling loops). This
was addressed by systems such as UpStare~\cite{upstare}, Ekiden~\cite{ekiden}
and Kitsune~\cite{kitsune}, which update the code by loading an entirely new
program instead of replacing individual functions. After the new version has
been loaded, either through \lstinline`fork-exec` or through
\lstinline`dlopen`, the execution needs to be restarted from the previous
point. UpStare uses stack reconstruction to achieve this by replacing the stack
frames for old functions with their new versions. Ekiden and Kitsune on the
other hand rely on manual approach requiring the programmer to direct the
execution into the equivalent update point in the new version.

%\textsc{PROTEOS}~\cite{proteos}.

Compared to \mx and \varan, in DSU systems the two versions co-exist only for
the duration of the software update, but DSU and the \rem component of \mx face
similar challenges when switching execution from one version to another.  We
hope that some of the technique developed in DSU research will also benefit the
recovery mechanism of \mx and vice versa.

%Dynamic software updating (DSU) systems such as Ginseng~\cite{ginseng},
%UpStare~\cite{upstare} or Kitsune~\cite{kitsune} are concerned with the problem
%of updating programs while they are running.  As opposed to \mx, the two
%versions co-exist only for the duration of the software update, but DSU and the
%\rem component of \mx face similar challenges when switching execution from one
%version to another.  We hope that some of the technique developed in DSU
%research will also benefit the recovery mechanism of \mx and vice versa.

\subsection{Update management and distribution}

Prior work on improving software updating has looked at different aspects
related to managing and deploying new software versions.

Beattie~\etal~\cite{beattie2002} have considered the issue of timing the
application of security updates---patching too early could result in breaking
the system by applying a broken patch, patching too late could on the other
hand lead to the risk of penetration by an attacker exploiting a well known
security issue. Using the cost functions of corruption and penetration, based
upon real world empirical data, they have shown that 10 and 30 days after the
patch's release date are the optimal times to apply the patch to minimize the
risk of a defective patch.  Such a delay still opens a lot of opportunities for
potential attackers.

Crameri \etal~\cite{crameri:updates} proposed a framework for staged
deployment. This framework integrates upgrade deployment, user-machine testing
and problem reporting into the overall upgrade process. The framework itself
clusters user machines according to their environment and software updates are
tested across clusters using several different strategies allowing for the
deployment of complex upgrades.

A solution inspired by $N$-version programming was proposed by Michael
Franz~\cite{unibus:nspw10}. Instead of executing multiple versions in parallel,
the author suggested distributing a unique version of every program to every
user. Such versions should be created automatically by a ``multicompiler'',
and distributed to users through ``App Store''. This would increase security as
it would be much more difficult to generate attack vectors by reverse
engineering of security patches for these diversified versions.

%To make such a solution work in practice, the way to manage large number of
%different versions and the overall update process is needed.
%Crameri~\etal~\cite{crameri:updates} proposed a framework for staged deployment
%which integrates upgrade deployment, user-machine testing and problem reporting
%into the overall upgrade process. The framework itself clusters users'
%machines according to their environment, tests the upgrades using cluster
%representatives and allows deployment of complex upgrades.

%Beattie~\etal~\cite{beattie2002} considered the issue of timing the application
%of security updates---patching too early could result in breaking the system by
%applying broken patch, patching too late could on the other hand lead to risk
%of penetration by an attacker exploiting a well known security issue. Using the
%cost functions of corruption and penetration, based upon real world empirical
%data, they have shown that 10 and 30 days after the patch's release date are
%the optimal times to apply the patch to minimize the risk of defective patch.
%Such delay still opens a lot of opportunities for potential attackers.

In relation to this work, \mx and \varan try to ease the decision of applying a
software update, because incorporating a new version should only increase the
security and reliability of the overall multi-version application.  However, in
practice the number of versions that can be run in parallel is dictated by the
number of available resources (\eg the number of available CPU cores), so
effective update strategies are still needed in this context, and work in this
area could provide helpful solutions to this problem.

%In relation to this work, \mx tries to encourage users to always apply
%a software update, but it would still benefit from effective update
%strategies to decide what versions to keep when resources are
%limited.

Many large-scale services, such as Facebook and Flickr use a {\it continuous
deployment} approach, where new versions are continuously released to
users~\cite{johnson2009,flickr}, but each version is often made accessible only
to a fraction of users to prevent complete outage in case of newly introduced
errors.  While this approach helps minimize the number of users affected by new
bugs, certain bugs may manifest themselves only following prolonged operation,
after the release has been deployed to the entire user base.  We believe our
proposed approach is complementary to continuous deployment, and could be
effectively combined with it.

\section{Sandboxing}
\label{related:sandboxing}

$N$-version execution environment typically uses some form of application
sandboxing. The goal is to isolate the running application from underlying
system and especially, to prevent software failures in any of the versions
being executed from affecting the rest of the system including other versions.

Prior sandbox architectures include both kernel-based
mechanisms~\cite{tron,remus,subdomain,cots-hardening} and system call
interposition monitors~\cite{wily-hacker,mapbox,jain1999,provos2002,mbox}.
Other mechanisms have been also used to build sandboxes, such as binary
rewriting~\cite{vx32,true:sp12}, system transactions~\cite{txbox} or virtual
machine introspection~\cite{garfinkel:vmi}.

\subsection{System call interposition}
\label{related:syscalls}

Regardless of the application, most operating system resources such as files
and sockets can only be accessed through the system call interface.  Therefore,
this interface is often a target for monitoring and regulation. Both \varan and
\mx operate at the system call interface layer synchronizing system calls
performed by different application versions. \varan draws inspiration from the
Ostia delegating architecture~\cite{ostia}, and from the selective binary
rewriting approach implemented by \emph{BIRD}~\cite{bird} and
\textsf{seccompsandbox}.\footnote{\url{https://code.google.com/p/seccompsandbox/}}
\mx on the other hand uses the \textsf{ptrace} interface same as many of the
existing monitors such as Orchestra~\cite{orchestra09} or
Tachyon~\cite{tachyon12}.

System call interposition has been an active area of research and there are
there many different ways in which the system call interception mechanism can
be implemented, including dynamic libraries loaded through the preloading
mechanism (\ie using \lstinline`LD_PRELOAD` or
\lstinline`DYLD_INSERT_LIBRARIES` variables) and custom C
libraries~\cite{plash}, facilities provided by the operating system (\ie
\ptrace interface or \lstinline`/proc` subsystem), kernel extensions and binary
rewriting.

Janus~\cite{wily-hacker} was one of the first systems to implement a confinement
mechanism by interposing on the system calls made by the application via the
standard tracing mechanisms, in particular \ptrace. The similar architecture
has been later adopted by MAPbox~\cite{mapbox} and Consh~\cite{consh}. However,
all of these systems suffered from numerous limitations posed by the \ptrace
interface~\cite{janus}. Jain and Sekar~\cite{jain1999} tried to address some of
these issues to build a more efficient user-space confinement mechanism.  A
later Janus~\cite{janus} evolution implemented a tracing mechanism for Solaris
using \lstinline`/proc` subsystem. To address some of the \ptrace shortcomings,
Systrace~\cite{provos2002} used a loadable kernel module.  Some system call
interpositon mechanisms were implemented entirely in
kernel~\cite{subdomain,cots-hardening}. Ostia~\cite{ostia} has shown how to
implement a secure and efficient system call interposition mechanism by combing
kernel extension with a custom user-space binary rewriting mechanism avoiding.
More recently, \textsc{Mbox}~\cite{mbox} implemented sandboxing mechanism by
combining \ptrace and \textsf{seccomp/bpf} to allow for selective system call
filtering.

Numerous intrusion detection mechanisms rely on analysis of system call
sequences looking for potential
anomalies~\cite{syscall-seq,wespi00,sekar01,gao04,sandeep06}. These systems
rely on system call tracing, which is similar to system call interposition,
although there is no need for modyfing or denying system calls.

Orchestra~\cite{orchestra09} and Tachyon~\cite{tachyon12} used system call
interposition through \ptrace interface to implement $N$-version runtimes.  To
improve the performance overhead, one of the \ptrace shortcomings when reading,
Orchestra uses shared memory injected into the application's address space to
read the values of indirect system call arguments, while Tachyon uses cross
memory attach, same as \mx (\S\ref{sec:mxm}), for the same purpose.

Due to their importance, especially when used for application sandboxing,
system call monitors themselves can also become targets for attackers.
Garfinkel~\cite{garfinkel:traps} and Watson~\cite{watson07} described common
vulnerabilities in system call monitors and the ways in which these could be
exploited.

%User-mode Linux~\cite{dike:uml}, a mechanism which allows running multiple
%virtual Linux systems on a Linux host has been also implemented using the
%\ptrace interface.

% System call interposition has been an active area of research and there are
% there many different ways in which the system call interception mechanism can
% be implemented, including dynamic libraries loaded through the preloading
% mechanism and custom C libraries, facilities provided by the operating system,
% kernel extensions and binary rewriting.

% MAPbox~\cite{mapbox}, Systrace~\cite{provos2002}, Jailer~\cite{jailer},
% BlueBoX~\cite{bluebox}, \textsc{Mbox}~\cite{mbox} used \ptrace to implement
% system Linux call monitors for intrusion detection and confinement akin to
% network firewalls.  Jailer~\cite{jailer} provides a custom kernel module as an
% alternative to the \ptrace interface to address some of its shortcomings.
% Janus~\cite{wily-hacker,janus} used the \lstinline`/proc` subsystem to
% implement a similar confinement mechanism for Solaris. 

% \subsection{Library-based mechanisms}

% System calls are rarely performed directly by the application, which rather use
% the wrapper functions provided by the C library. By linking to a custom library,
% which provides a custom version of these functions, we could intercept system
% calls performed by the appliction. This could be done either statically at link
% time, or dynamically at runtime (\eg using \lstinline`LD_PRELOAD` or
% \lstinline`DYLD_INSERT_LIBRARIES` variables). The former approach has been used
% by systems such as Plash~\cite{plash}, a sandboxing system for Linux
% implemented as a modified version of \gnu C library , while the latter approach
% has been used by research projects such as RCV~\cite{shepherding:pldi14}. The
% major benefit of this approach is efficiency and relative ease of
% implementation. However, the mechanism could be easily bypassed (\eg by
% invoking system calls directly) by the application making it unsuitable for
% security applications.

% \subsection{\ptrace}

% \ptrace is an interfaces provided by most UNIX-like operating systems,
% including Linux, providing means by which a process might observe and control
% another process. The relative ease of use makes \ptrace a popular choice for
% implementing system system call monitors, including \mx.  Systems such as
% MAPbox~\cite{mapbox}, Systrace~\cite{provos2002}, Jailer~\cite{jailer},
% BlueBoX~\cite{bluebox}, \textsc{Mbox} use \ptrace to implement systems for
% intrusion detection and confinement.  Orchestra~\cite{orchestra09} and
% Tachyon~\cite{tachyon12} used \ptrace to implement $N$-version runtimes.
% User-mode Linux~\cite{dike:uml}, a mechanism which allows running multiple
% virtual Linux systems on a Linux host has been also implemented using the
% \ptrace interface.

% Despite \ptrace's popularity, especially for rapid
% prototyping~\cite{spillane07}, there are numerous drawbacks hindering its use
% for pratical deployment: a significant performance overhead due to large number
% of context switches, problematic support for multithreaded applications, and
% the lack of filtering mechanism allowing to intercept only system calls of
% interest. The performance overhead could be partially improved by the use of
% more efficient mechanism for copying memory from/to the monitored process, such
% as shared memory as in case of Orchestra~\cite{orchestra09}, or cross memory
% attach as in case of \mx (\S\ref{sec:mxm}). The lack of filtering mechanism
% could be addressed by combining \ptrace with \textsf{seccomp/bpf} mechanism as
% shown by \textsc{Mbox}~\cite{mbox}.

% \subsection{Kernel-based mechanisms}

% An alternative to \ptrace is to implement the system call monitor entirely or
% partially in kernel space. Jailer~\cite{jailer} provides a custom kernel module
% as an alternative to the \ptrace interface. Strata~\cite{cox2006} implements
% $N$-version execution monitor inside the kernel. Ostia~\cite{ostia} combines
% a custom kernel module with user space runtime for application sandboxing.

% While this approach has numerous advantages compared to other approaches, such
% as minimal performance overhead and direct access to the application's
% execution context and memory contents, there are several drawbacks as well.
% First, this approach requires kernel patches and/or new new kernel modules
% which complicates the development, limits portability across different
% operating systems or even different kernel versions, and hinders
% maintainability. Second, the monitor must be run in privileged mode, which
% means that bugs in the implementation may compromise the system stability and
% security. Furthermore, it also makes it difficult for regular users to deploy
% and use such monitors.

% \subsection{Binary rewriting}

% Binary rewriting technique allow transforming the executable (either statically
% or dynamically) altering its functionality; as such, it can be used to
% implement system call monitor by rewriting all system call instructions into a
% control flow transfer instructions (\eg a
% \lstinline[language={[x64]Assembler}]`jmp` instruction). Existing monitors
% were built either on top of existing binary rewriting systems such as Pin~\cite{pin}
% or DynamoRIO~\cite{dynamorio02} as in case of $\delta$
% execution~\cite{onlinevalidation}, or using a purpose built binary translation
% mechanism as in case of Vx32~\cite{vx32},
% \textsf{libdetox}~\cite{libdetox:vee11}, BIRD~\cite{bird}, and
% \textsf{seccompsandbox}. The advantage of binary rewriting-based monitors is
% relatively low performance overhead, especially in the case of purpose built
% rewriters as we have demonstrated in case of \varan, the main disadvantage is a
% high complexity of implementing such monitor.

\subsection{Software fault isolation}
\label{related:sfi}

The use of software-based fault isolation for executing untrusted code has been
first described in~\cite{sfi:sosp93} for the RISC machines with simple
instruction set. However, the first effective implementation of software fault
isolation for the CISC architecture has been shown only much later
in~\cite{cisc-sfi:usenix-sec06}.

These concepts have been later used by several sandboxes designed for isolation
of plugins on the web. Xax~\cite{douceur08} uses system call interposition
together with address space isolation to leverage existing libraries and
programs on the web. Native Client~\cite{nacl} requires the code to be
recompiled to a restricted subset of the x86/ARM ISA which can be checked prior
running and confines the application using segmentation and memory protection.
The problem of protecting trusted code from the untrusted code addressed by
both of these sandboxes is orthogonal to traditional system-level sandboxing
and these systems rely on separate sandboxing mechanism for that purpose (\eg
Native Client uses either \textsf{seccomp} or \textsf{seccomp-bpf} as an
"outer" sandbox layer).

% MAPbox~\cite{mapbox}, Systrace~\cite{provos2002} and BlueBoX~\cite{bluebox}
% implement Linux system call monitors for intrusion detection and application
% confinement using the \ptrace interface. Jailer~\cite{jailer} provides a custom
% kernel module as an alternative to \ptrace to address some of its shortcomings.
% \textsc{Mbox}~\cite{mbox} combines \ptrace with \textsf{seccomp/bpf} to allow
% for selective system call filtering. Janus~\cite{wily-hacker,janus} implements
% similar confinement mechanism for Solaris using the \lstinline`/proc`
% subsystem.

%Sandboxing~\cite{bascule,drawbridge}.
%Capsicum~\cite{capsicum}.

\section{Record \& replay}
\label{related:record}

Event streaming in \varan can be seen as a variant of record-replay. Record and
replay technique has been an active research topic for more than forty years
and different approaches were developed over this period. However, unlike
traditional record-replay systems that require a persistent log, \varan keeps
the shared ring buffer in memory, and deallocates events as soon as they are
not needed, which minimises performance overhead and space requirements in the
NVX context.

%Record \& replay has been an active topic for many years and there has been
%numerous systems implemented at different layers, they all shared the same
%basic concepts.

BugNet~\cite{bugnet}, Jockey~\cite{jockey}, liblog~\cite{geels06}, and
R2~\cite{r2} use a library approach where the replay tool is provided as a
library which is inserted into a target application to allow for record and
replay. Jockey and liblog try to minimize the interference with the application
being recorded to ensure that the application behaves the same with and without
the tool, and rely solely on intercepting of a set of C library calls. BugNet
provides a special API applications must use to support record and replay while
R2 requires developers to annonate functions which can be recorded and replayed
correctly. RecPlay~\cite{recplay} also uses a library approach, but focuses on
debugging of nondeterministic parallel programs. The approach combines
record-replay mechanism with automatic data race detection implemented using
vector clocks.

Flashback~\cite{flashback} has a similar focus, enabling deterministic
debugging, but rather than library, it is implemented as a kernel extension.
The key concept is the use of shadow processes, which duplicate the state of a
process in memory allowing fast rollback of the debugged program with small
overhead. Scribe~\cite{scribe} is another kernel-based solution focused on
transparent, low-overhead record-replay with the ability to switch from
replayed to live execution. Same as in case of Flashback, while the
kernel-based implementation results in lower performance overhead, it makes
these systems less easy to develop and debug, and less safe to use compared to
library based approaches.

RR~\cite{rr} is similar to \varan by replicating an application into multiple
instances for fault tolerance. Same as \varan, it also uses a variant of
record-replay to synchronize the primary replica with others. Compared
to \varan, RR is focused on fail-stop scenarios and does not support
running different versions of the same application. The preliminary prototype
has been implemented as a Linux kernel extension, with all the associated
limitations as discussed above, and evaluated using only a single benchmark.

ReVirt~\cite{revirt}, SMP-ReVirt~\cite{smp-revirt} and iDNA~\cite{idna}
implement record/replay mechanism as a part of virtual machine hypervisor,
allowing to record and replay the entire system, including the operating system
and the target applications. ReVirt is built on top of UML, which lacks the
support for multiprocessor virtual machines. This limitations has been addressed
in SMP-ReVirt which builts on top of Xen. iDNA uses a custom virtualization
engine called Nirvana. The virtualization-based record-replay mechanism allow
faithful replay of every aspect of the application's environment, including
scheduling decisions of the operating system, with small performance overhead.
On the other hand, these systems are more difficult to deploy and their use is
more expensive as they require an entire operating system to record a single
application.

There are numerous record-replay tools focused on restricted programming models
such as distributed shared memory~\cite{russinovich96}, MPI~\cite{rrmpi} and
programming language runtimes such as Java~\cite{choi98}.

\section{Software evolution}
\label{related:evolution}

Despite the significant role that coverage information plays in
software testing, there are relatively few empirical studies on this
topic.  We discuss some representative studies below.

Early work on this topic was done by Elbaum \etal~\cite{cov-evol:icsm01}, who
have analysed how the overall program coverage changes when software evolves,
using a controlled experiment involving the \textsf{space} program, and seven
versions of the Bash shell.  One of the key findings of this study was that
even small changes in the code can lead to large differences in program
coverage, relative to a given test suite.  This is a different finding from
previous work, such as that by Rosenblum and Weyuker~\cite{cov_regr97}, which
has found that coverage remains stable over time for the KornShell benchmark.
In this thesis, we have looked at a related question, of whether overall
coverage remains stable over time, taking into consideration the changes to the
evolving test suite as well.

Zaidman \etal~\cite{coevol:emse11} have examined the co-evolution of
code and tests on two open-source and one industrial Java
applications.  The study looks at the evolution of program coverage
over time, but only computes coverage for the major and minor releases
of each system, providing around ten data points for each system.  By
looking at the co-evolution of code and tests, the analysis can infer the
development style employed by each project: one key finding is that code and
tests co-evolve either \emph{synchronously}, as when agile methods are used; or
\emph{phased}, with periods of intense coding followed by periods of intense
testing. In our experiments presented in Chapter~\ref{chap:evolution}, we have
observed both development styles.

% Our ongoing effort is to develop \covrig into a flexible platform for
% mining static and dynamic metrics from software repositories.  In
% terms of similar infrastructure efforts, SIR~\cite{sir:2005} is a
% well-known repository for software testing research, which offers a
% variety of programs written in several different languages, together
% with test suites, bug data, and scripts.  SIR also provides multiple
% versions for the same application, but typically less than a dozen.
% Furthermore, SIR does not include any support for running versions in
% isolation.  Ideally, the mechanisms provided by \covrig would be
% integrated with the rich data in SIR to enable more types of analyses
% at the intersection of software testing and evolution.

% While SIR contains mostly artificially-generated faults,
% iBUGS~\cite{ibugs} provides a semi-automated approach for extracting
% benchmarks with real bugs from project histories, using an approach
% based on commit messages and regression tests. iBUGS' idea of using
% the regression tests as a semi-automatic bug confirmation mechanism
% could be borrowed by \covrig whenever fixes are accompanied by tests,
% reducing the manual effort needed to apply it to new projects.

% While SIR is restricted to a fixed set of bugs, iBUGS~\cite{ibugs}
% provides a semi-automated approach for extracting benchmarks with real
% bugs from a project's history. iBUGS identifies the tests provided
% with a bug fix, executes them against the revision before the fix, and
% if the test fails, applies bug localization tools to extract the
% information necessary for reproducing the failure. We could improve
% the precision of our bug analysis and reduce the manual effort needed
% when applying \covrig to new projects by integrating the iBUGS approach
% into our tool.

%% Relatively few versions have been examined...

%% A recent trend in software testing and verification research is to
%% focus on analysing program changes.  In this empirical paper, we try
%% to provide...  This form of longitudinal analysis...

%% A large fraction of the cost of maintaining software is associated with
%% detecting and fixing errors introduced by recent patches.  It is well-known
%% that patches are prone to introduce failures~\cite{yin11,buggy-fixes:icse10}.

%% Recent work attempts to address this problem through various
%% approaches.  Automatic testing and verification techniques that focus
%% on patch code~\cite{katch,fse13-diff-assertions,interaction-changes13}
%% allow developers to check the quality of changes before shipping them
%% to customers.  Online validation~\cite{onlinevalidation} allows
%% checking the behaviour of a patch against real workloads before
%% deciding to place the new version in production. Finally, detecting
%% and masking faults at runtime~\cite{mx,tachyon12} allows users to take
%% the best of both the new and the old version.

%\cite{release-patterns:icsm07}


%\subsection{Execution of Multiple Versions}

%The idea of concurrently running multiple versions (or a \emph{multi-version
%execution}) of the same application was first explored in the context of
%$N$-version programming, a software development methodology introduced in the
%seventies in which multiple teams of programmers develop functionally
%equivalent versions of the same program in order to minimise the risk of having
%the same bugs in all versions~\cite{chen1995}. Furthermore, the use of an
%execution environment responsible for running the $N$-version programs and
%choosing one of the outputs was proposed.
  
%This idea has been followed up in~\cite{cox2006}, which proposes the use of
%automatically generated {\it diversified variants} of the same program to
%increase application security. Through the use of multiple variants with
%potentially disjoint exploit sets, the proposed approach makes it difficult to
%exploit a large class of security vulnerabilities such as buffer overflow, as
%attackers would need to simultaneously compromise all variants.

%The idea has been implemented in \cite{orchestra09}, which uses a modified
%compiler to produce two versions of the same application with stacks growing in
%opposite directions, together with an execution environment that monitors both
%variants at runtime, checking for any discrepancies in system calls made by the
%variants that would indicate a successful security attack on one of the
%replicas. This work has been further improved in \cite{orchestra11} with new
%method for signal delivery accompanied with detailed analysis of the benchmark
%characteristics. However, their approach is still limited to two versions
%running in lockstep with only difference being the opposite growing stack.
%Most importantly, whenever divergence between the two versions is detected,
%they are immediately stopped.

%The multi-version execution idea has been also used for different purposes ---
%Berger et al. describe a similar approach that uses address space layout
%randomization to generate multiple replicas that are executed
%concurrently~\cite{diehard06}. This approach is combined with randomization
%and replication to provide memory error tolerance. The goal of their work is to
%increase reliability by tolerating memory errors in exchange for space costs
%and execution time.

%Solution described in~\cite{shye2009} runs multiple instances of the same
%version of a single application aiming to overcome transient failures by
%monitoring and comparing their execution on the level of kernel system calls.
%However, this solution does not aim to overcome such failures.

%Running different versions of an application in parallel has also been used to
%test and validate software patches.  In~\cite{onlinevalidation}, two different
%versions of a single application are run in parallel, splitting the execution
%at points where the two versions differ, and comparing their results to test
%the patch for errors and validate its functionality. Whenever one of the two
%versions crashes, a bug is reported. This approach is limited only to a
%specific categories of patches such as refactoring or changes to rarely used
%paths (\eg error handlers). Moreover, this approach is only targeted towards
%on-line validation, not as a generally usable runtime environment.

%Trachsel et al.  use a similar approach to increase performance, where program
%variants are obtained by using different (compiler) optimizations and
%algorithms~\cite{trachsel10}.  The goal is to increase the overall system
%performance by always selecting the variant which finishes its execution first.
%Thereby, no synchronization across variants is needed.

%\subsection{Multi-version Environment and Update Management}

%Closely related to the execution of multiple versions is the management of
%multiple versions, environment and software updates, such as deciding when to
%upgrade, applying updates, \etc

%Solution, which in some sense resembles the multi-version execution idea, has
%been proposed by Michael Franz in~\cite{franz2010} --- instead of executing
%multiple versions in parallel, he proposes to distribute unique version of
%every program to every user. Such versions should be created automatically by
%a ``multicompiler'' and distributed to users through ``App Store''. This would
%increase security as it would be much more difficult to generate attack
%vectors by reverse engineering of security patches for these diversified
%versions.

%To make such a solution work in practice, the way to manage large number of
%different versions and the overall update process is needed. This idea has
%been explored in~\cite{crameri:updates}. This framework integrates upgrade
%deployment, user-machine testing and problem reporting into the overall
%upgrade process.  The framework itself clusters users' machines according to
%their environment, tests the upgrades using cluster representatives and allows
%deployment of complex upgrades.

%Beattie et al. showed in~\cite{beattie2002} that the timing of security patch
%applying can be of critical importance. Patching too early could result in
%breaking the system by applying broken patch, patching too late could on the
%other hand lead to risk of penetration by an attacker exploiting a well known
%security issue. Using the cost functions of corruption and penetration, based
%upon real world empirical data, they have shown that 10 and 30 days after the
%patch's release date are the optimal times to apply the patch to minimize the
%risk of defective patch. Such delay still opens a lot of opportunities for
%potential attackers.

%We believe that our approach can decrease this time virtually to zero
%eliminating the possibility of penetration while retaining the reliability of
%the original version.

%\begin{structure*}
%\item MonDe: Safe Updating through Monitored Deployment of New Component Versions
%\item Towards A Self-Managing Software Patching Process Using Balck-Box Persistent Manifests
%\item Predicting Problems Caused by Component Upgrades
%\item The Cracker Patch Choice: An Analysis of Post Hoc Security Techniques
%\item Yesterday, my program worked. Today, it does not. Why?
%\end{structure*}

%\subsection{Application Sandboxing and Software Fault Isolation}

%Multi-version execution requires some form of application sandboxing and
%software fault isolation. The goal is to isolate the running application from
%underlying system and especially, to prevent software failures from affecting
%the rest of the system including other applications.

%The use of software-based fault isolation for executing untrusted code has
%been described already in~\cite{wahbe1993} for the RISC machines with simple
%instruction set. On the other hand, the first effective implementation of
%software fault isolation for the CISC architecture has been shown much later
%in~\cite{mccamant2006}.

%% Control flow integrity?

%Douceur et. al shown in~\cite{douceur08} how to use sandboxing to enable
%leveraging of existing libraries and programs on the web. To achieve that,
%they used application-level virtualization; their implementation use system
%call mediation together with their own platform abstraction layer to run each
%library or program in so-called \emph{picoprocesses} which can be seen as
%stripped down virtual machine. The biggest downside of this approach is the
%need for code modifications which significantly reduces usability of this
%approach.

%Similar idea has been implemented by Yee et al. in~\cite{nacl} as an
%extension to Google Chrome web browser providing sandbox for untrusted native
%x86 code.  This can be used to develop web applications with performance of
%native applications. Their implementation consists of two parts --- inner
%sandbox which uses lightweight static analysis do detect security defects and
%outer sandbox which mediates system calls. The implementation has been further
%improved in~\cite{sehr2010} with support for ARM and x86-64 architectures and
%recently has been extended in~\cite{ansel2011} by adding support for
%Just-In-Time (JIT) compilation sandboxing. Again, the disadvantage here is the
%need for modifications of existing code and the use of modified GNU compiler
%toolchain to generate compliant binaries.

%\subsection{Virtualization and Fault-tolerant Computing}

%\begin{structure*}
  %\item Multiscale not Multicore: Efficient Heterogeneous Cloud Computing
  %\item Opportunistic Computing: A New Paradigm for Scalable Realism on Many-Cores
  %\item The Utility Coprocessor: Massively Parallel Computation from the Coffee Shop
%\end{structure*}

%Hardware-level virtualization has already been widely adopted in the industry
%for a variety of purposes.  Companies such as VMware or Microsoft provide a
%wide range of virtualization products, while several high-quality open-source
%solutions such as Xen~\cite{xen} also exist.
  
%Operating system-level virtualization is a method which allows to virtualize
%the operating system kernel into multiple isolated partitions (\ie user-space
%instances).  This type of virtualization is provided for different operating
%systems by products such as FreeBSD Jails~\cite{jails}, Linux
%VServer~\cite{vserver} or Solaris Containers~\cite{containers}. Even though
%these solutions are not as widely deployed as those for hardware-level
%virtualization, they are readily available, have lower overhead, and can be
%employed in many real-world scenarios.
  
%We aim to use \emph{application-level virtualization}, which is a lightweight
%variant of operating system-level virtualization, in which applications run in
%independent execution environments.  Even though this area has not been very
%intensively studied, many of its concepts, such as sandboxing, are becoming
%more and more common, especially in the cloud environment.   Some of these
%features are provided by the products such as VMware ThinApp~\cite{thinapp} or
%Microsoft App-V~\cite{appv}.

%Related research utilizing the concepts of application-level virtualization
%has been described in~\cite{yu2006} proposing lightweight virtual machine
%monitor for virtualization of Windows applications. The authors used namespace
%virtualization, implemented at system call level, to achieve isolation of
%resources between individual applications as well host operating system. This
%work has been followed up in~\cite{yu2008} showing three complete application
%benefiting from the approach presented in the original paper.

%More heavyweight, but also more powerful approach was taken by Dike et al.
%in~\cite{dike2001}. Using system call virtualization (via \texttt{ptrace}
%interface) and device abstraction, they managed to port Linux kernel to
%userspace. The resulting implementation runs a Linux virtual machine in a set
%of processes on a Linux host allowing various applications ranging from kernel
%development to application sandboxing and virtualization. Now, this
%implementation has become an optional module of Linux kernel.

%Virtualization approach is also often used in environments with strong
%reliability requirements, especially in the domain of cluster computing. We
%believe that our approach could be used also in these environments. Shown by
%Schroeder et al.  in~\cite{schroeder2007}, around 20\% of all failures at Los
%Alamos National Laboratory are caused by software failures making them a
%second largest contributor after hardware failures. While today's high
%performance computing systems relies mostly on checkpoint-restart schemes to
%achieve fault tolerance, such as the one proposed in~\cite{srinivasan2004}
%further used in~\cite{qin2005}. This approach is often not sufficient, mainly
%due to large decrease in the effective application utilization. Our approach,
%based on multi-version execution inspired by application-level virtualization
%idea, might present a viable alternative to these approaches.

%\begin{structure}
%\item Control-flow integrity
  %\begin{structure*}
  %\item Control-Flow Integrity
  %\item XFI: Software Guards for System Address Spaces
  %\end{structure*}
%\item Implementation related
  %\begin{structure*}
  %\item Rapid File System Development Using ptrace
  %\item Detours: Binary Interception of Win32 Functions
  %\end{structure*}
%\item Checkpointing and execution rollback
  %\begin{structure*}
  %\item Flashback: A Lightweight Extension for Rollback and Deterministic Replay for Software Debugging
  %\item Rx: Treating Bugs As Allergies
  %\item Making Applications Mobile Under Linux
  %\end{structure*}
%\end{structure}

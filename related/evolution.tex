\section{Software evolution}
\label{related:evolution}

Despite the significant role that coverage information plays in
software testing, there are relatively few empirical studies on this
topic.  We discuss some representative studies below.

Early work on this topic was done by Elbaum \etal~\cite{cov-evol:icsm01}, who
have analysed how the overall program coverage changes when software evolves,
using a controlled experiment involving the \textsf{space} program, and seven
versions of the Bash shell.  One of the key findings of this study was that
even small changes in the code can lead to large differences in program
coverage, relative to a given test suite.  This is a different finding from
previous work, such as that by Rosenblum and Weyuker~\cite{cov_regr97}, which
has found that coverage remains stable over time for the KornShell benchmark.
In this thesis, we have looked at a related question, of whether overall
coverage remains stable over time, taking into consideration the changes to the
evolving test suite as well.

Zaidman \etal~\cite{coevol:emse11} have examined the co-evolution of
code and tests on two open-source and one industrial Java
applications.  The study looks at the evolution of program coverage
over time, but only computes coverage for the major and minor releases
of each system, providing around ten data points for each system.  By
looking at the co-evolution of code and tests, the analysis can infer the
development style employed by each project: one key finding is that code and
tests co-evolve either \emph{synchronously}, as when agile methods are used; or
\emph{phased}, with periods of intense coding followed by periods of intense
testing. In our experiments presented in Chapter~\ref{chap:evolution}, we have
observed both development styles.

% Our ongoing effort is to develop \covrig into a flexible platform for
% mining static and dynamic metrics from software repositories.  In
% terms of similar infrastructure efforts, SIR~\cite{sir:2005} is a
% well-known repository for software testing research, which offers a
% variety of programs written in several different languages, together
% with test suites, bug data, and scripts.  SIR also provides multiple
% versions for the same application, but typically less than a dozen.
% Furthermore, SIR does not include any support for running versions in
% isolation.  Ideally, the mechanisms provided by \covrig would be
% integrated with the rich data in SIR to enable more types of analyses
% at the intersection of software testing and evolution.

% While SIR contains mostly artificially-generated faults,
% iBUGS~\cite{ibugs} provides a semi-automated approach for extracting
% benchmarks with real bugs from project histories, using an approach
% based on commit messages and regression tests. iBUGS' idea of using
% the regression tests as a semi-automatic bug confirmation mechanism
% could be borrowed by \covrig whenever fixes are accompanied by tests,
% reducing the manual effort needed to apply it to new projects.

% While SIR is restricted to a fixed set of bugs, iBUGS~\cite{ibugs}
% provides a semi-automated approach for extracting benchmarks with real
% bugs from a project's history. iBUGS identifies the tests provided
% with a bug fix, executes them against the revision before the fix, and
% if the test fails, applies bug localization tools to extract the
% information necessary for reproducing the failure. We could improve
% the precision of our bug analysis and reduce the manual effort needed
% when applying \covrig to new projects by integrating the iBUGS approach
% into our tool.

%% Relatively few versions have been examined...

%% A recent trend in software testing and verification research is to
%% focus on analysing program changes.  In this empirical paper, we try
%% to provide...  This form of longitudinal analysis...

%% A large fraction of the cost of maintaining software is associated with
%% detecting and fixing errors introduced by recent patches.  It is well-known
%% that patches are prone to introduce failures~\cite{yin11,buggy-fixes:icse10}.

%% Recent work attempts to address this problem through various
%% approaches.  Automatic testing and verification techniques that focus
%% on patch code~\cite{katch,fse13-diff-assertions,interaction-changes13}
%% allow developers to check the quality of changes before shipping them
%% to customers.  Online validation~\cite{onlinevalidation} allows
%% checking the behaviour of a patch against real workloads before
%% deciding to place the new version in production. Finally, detecting
%% and masking faults at runtime~\cite{mx,tachyon12} allows users to take
%% the best of both the new and the old version.

%\cite{release-patterns:icsm07}

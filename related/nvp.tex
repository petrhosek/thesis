\section{$N$-version programming}
\label{related:nvp}

The original idea of concurrently running multiple versions of the same
application was first explored in the context of $N$-version programming, a
software development approach introduced in the 1970's in which multiple teams
of programmers independently develop functionally equivalent versions of the
same program in order to minimise the risk of having the same bugs in all
versions~\cite{chen1995}. During runtime, these versions are executed in
parallel on top of an $N$-version execution (NVX) environment in order to
improve the fault tolerance of the overall system. Both the version generation
and the synchronization mechanism required manual effort.

% More moder instantiation of N-version programming paradigm is
A possible realization of the original N-version programming paradigm is
Cocktail~\cite{cocktail}, which proposes the idea of running different web
browsers in parallel under the assumption that any two of them are unlikely to
be vulnerable to the same attacks. Compared to \mx and \varan, Cocktail's task
is simplified by exclusively targeting web browsers, which implement common web
standards, while both \mx and \varan support arbitrary off-the-shelf
applications.

%system and majority voting is used to continue in the best possible way when a
%divergence occurs.

%N-version programming was introduced in the seventies by Chen and
%Avizienis~\cite{chen1995}.  The core idea was to have multiple teams
%of programmers develop the same software independently and then run
%the produced implementations in parallel in order to improve the fault
%tolerance of the overall system.  Both version generation and the
%synchronization mechanism required manual effort.

\subsection{Multi-variant execution}

Cox \etal~\cite{cox2006} proposed a general framework for increasing
application security by running in parallel several automatically-generated
diversified variants of the same program---when a divergence is detected, an
alarm is raised. The technique was implemented in two prototypes, one in which
the variants are run on different machines, and one in which they are run on
the same machine and synchronised at the system call level, using a modified
Linux kernel. By using multiple variants with potentially disjoint exploit
sets, the proposed approach helps applications survive certain classes of
security vulnerabilities such as buffer overflows, as attackers would need to
simultaneously compromise all variants.

Berger \etal~\cite{diehard06} described a different approach which uses heap
over-provisioning and full randomisation of object placement and memory reuse
to run multiple replicas and reduce the likelihood that a memory error will
have any effect. The goal of their work is to increase reliability by
tolerating certain memory errors in exchange for space cost and execution time.

%A different solution described by Shye \etal in~\cite{shye2009} runs multiple
%instances of the same version of a single application aiming to overcome
%transient failures by monitoring and comparing their execution on the level of
%kernel system calls.  However, this solution does not aim to overcome such
%failures.

Within this paradigm, the Orchestra framework~\cite{orchestra09} used a
modified compiler to produce two versions of the same application with stacks
growing in opposite directions, runs them in parallel on top of an unprivileged
user-space monitor, and raises an alarm if any divergence is detected to
protect against stack-based buffer overflow attacks.  This work has been
further improved in~\cite{orchestra11} with a new method for signal delivery
accompanied with detailed analysis of the benchmark characteristics. However,
their approach is still limited to two versions running in lockstep with the
only difference being the opposite growing stack.

Trachsel \etal~\cite{trachsel10} used the multi-variant execution approach to
increase performance, where program variants are obtained by using different
(compiler) optimisations and algorithms.  The goal is to increase the overall
system performance by always selecting the variant which finishes its execution
first. Thereby, no synchronisation across variants is needed.

More recently, researchers have proposed additional multi-variant techniques
that fit within the $N$-version programming paradigm, \eg by employing
complementary thread schedules to survive concurrency
errors~\cite{compl-schedules11}, or using genetic programming to automatically
generate a large number of application variants that can be combined to reduce
the probability of failure or improve various non-functional
requirements~\cite{gismoe}. Yumerefendi \etal~\cite{tightlip} run multiple
sandboxed copies of the same process to detect potential privacy leaks and to
enforce access control and privacy management; Shye \etal~\cite{shye2009} use
multiple instances of the same application in order to overcome transient
hardware failures. Cadar \etal have also argued that automatically generated
software variants are a good way for exploiting the highly parallel nature of
modern hardware platforms~\cite{multiplicity}.

%Also ManySAT~\cite{manysat} %% \todo{\ldots missing rest of the sentence}

There are two key differences between our approach and the work discussed
above. First, we do not rely on automatically-generated variants, but instead
run in parallel existing software versions, which raises a number of different
technical challenges.  This means that in our case, the variants are not
semantically equivalent. This eliminates the challenge of generating
diversified variants and creates opportunities in terms of recovery from
failures, but also introduces additional challenges in terms of synchronising
the execution of the different versions.  Second, prior work has mostly focused
on detecting divergences, while our main concern is to survive them, in order
to increase the availability of the overall application.

%Recent work on NVX systems has moved in the direction of
%opportunistically using existing versions---\eg different browser
%implementations~\cite{cocktail} or different software
%revisions~\cite{mx}---or automatically synthesizing them---\eg by
%varying the memory layout~\cite{cox2006,diehard06}, or the direction
%of stack growth~\cite{orchestra09}.  Significant effort has also been
%expended on running multiple versions in parallel in the context of
%online and offline
%testing~\cite{back-to-back90,onlinevalidation,bandaid-patch07,tachyon12}.
%\nx targets NVX systems that use
%system-call level synchronization and is oblivious to the way in which
%the versions are generated.

\subsection{Multi-version execution}

Cook and Dage~\cite{cook:icse99} proposed a multi-version framework for
upgrading components. Users formally specify the specific input subdomain that
each component version should handle, after which versions are run in parallel
and the output of the version whose domain includes the current input is
selected as the overall output of the computation. The system was implemented
at the level of leaf procedures in the Tcl language. The key difference with
\mx and \varan is that this framework requires a formal description of what
input domain should be handled by each version; in comparison, \mx and \varan
target crash bugs and are fully automatic.  Moreover, \mx's goal is to have all
versions alive at all times, so crash recovery plays a key role.  Finally, both
\mx and \varan have to carefully synchronise access to shared state, which is
not an issue at the level of Tcl leaf procedures.

%\subsection{Software testing}

The multi-version execution approach has been in the past successfully used for
testing. Back-to-back testing~\cite{back-to-back90}, where the same input is
sent to different variants or versions of an application and their outputs
compared for equivalence, has been used since the 1970s.  Band-aid
patching~\cite{bandaid-patch07} proposed an online patch testing system that
splits execution before a patch, and then retroactively selects one code
version based on certain criteria.  More recently, delta
execution~\cite{onlinevalidation} proposed to run two different versions of a
single application, splitting the execution at points where the two versions
differ, and comparing their behaviour to test the patch for errors and validate
its functionality.  Similarly, Tachyon~\cite{tachyon12} is an online patch
testing system
%developed in recent independent work 
in which the old and the new version of an application are run concurrently;
when a divergence is detected, the options are to either halt the program, or
to create a manual rewrite rule specifying how to handle the divergence.

The idea of running multiple executions concurrently has also been used in an
offline testing context.  For instance, d'Amorin \etal~\cite{delta-exec-oop}
optimise the state-space exploration of object oriented code by running the
same program on multiple inputs simultaneously, while Kim
\etal~\cite{shared-exec12} improve the testing of software product lines by
sharing executions across a program family.

By comparison with this body of work, our focus is on managing
divergences across software versions at runtime in order to keep the
application running, and therefore runtime deployment and automatic
crash recovery play a central role in both \mx and \varan.

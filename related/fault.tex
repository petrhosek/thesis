\section{Fault tolerance}

\mx's main focus is on surviving errors. Prior work in this area has employed
several techniques to accomplish this
goal~\cite{rx,compl-schedules11,fo,exec-trans06,vigilante,clearview,microreboots,shepherding:pldi14}.

Rx~\cite{rx} helps applications recover from software failures by
rolling back the program to a recent checkpoint upon a software
failure, and then re-executing it in a modified environment.  \mx
similarly rolls back execution to a recent checkpoint, but instead of
modifying the environment, it uses the code of a different version to
survive the bug.  The two approaches are complementary, and could be
easily combined to support a larger number of errors and application
types.

Failure-oblivious computing~\cite{fo} helps software survive memory
errors by simply discarding invalid writes and fabricating values to
return for invalid reads, enabling applications to continue their
normal execution path. Similar to failure-oblivious computing,
execution transactions~\cite{exec-trans06} help survive software bugs
by terminating the function in which the bug has occurred and
continuing to execute the code immediately following the
corresponding function call.  Our approach shares some of the
philosophy of these two techniques, as we cannot always guarantee that
the crashing version will correctly execute through the divergence
when using the other version's code.
% (see \S\ref{sec:rem}).  
However, by using a previously correct piece of code to execute
through the crash and regularly checking for divergences in the
external behaviour, our approach provides stronger guarantees than
those obtained by fabricating read values or terminating the function
in which the bug occurred.

Recovery shepherding~\cite{shepherding:pldi14} allows applications to survive
certain types of errors by attaching to the process when the error occurs,
repairing the execution, tracking the effects of the repair and detaching from
the process after all the effects disappear. RCV, a system implementing the
technique, enables applications to survive divide-by-zero and null-dereference
errors by discarding the invalid writes and fabricating the default value for
invalid reads same as failure-oblivious computing~\cite{fo}.

Jolt~\cite{jolt} and Bolt~\cite{bolt} allow programs to escape from infinite
and long-running loops which normally render the application unresponsive. Both
these systems detect whether the application is making any progression as the
opposite would indicate long-running or possibly infinite loop. When such loop
is detected, these systems deploy multiple strategies to escape the loop and
continue execution.  While Jolt uses a specialised compiler, Bolt same as \mx,
operates on off-the-shelf binaries. The technique for detecting and escaping
loops is orthogonal to \mx's recovery mechanism and could be easily integrated
into \rem to expand its recovery capabilities.

Research on automatic generation of filters based on vulnerability
%~\cite{song:oakland06,vigilante,bouncer,shieldgen} 
signatures~\cite{song:oakland06,vigilante} 
or on patch
%~\cite{clearview,demsky:repair,candea:dimmunix}
generation in deployed systems~\cite{clearview}
also target applications with high-availability requirements, and the
generated patches work by installing lightweight input filters or by changing
the values of memory locations at runtime.

Recovery-oriented computing~\cite{roc} advocates the re-engineering of software
systems to allow applications to recover from errors. Within this paradigm,
microrebooting~\cite{microreboots} proposes building systems out of
individually recoverable components, which can be rebooted to survive bugs,
without disturbing the rest of the application. Acceptability-oriented
Computing~\cite{aoc:oopsla03} argues that instead of attempting to build
error-free systems, designers should define properties the execution must
satisfy to be acceptable by the users, and then build a system as a set of
layered components which enforce these properties.

\section{Software updates}

Closely related to the execution of multiple versions is the management of
multiple versions and software updates, such as deciding when to upgrade,
applying updates, \etc Previous work on improving the software update process
has also looked at different aspects related to managing and deploying new
software versions.

\subsection{Dynamic software updates}

Dynamic software updating (DSU) systems are concerned with the problem of
updating programs while they are running. These systems have been implemented
both in kernel-space~\cite{k42,dynamos,ksplice,proteos} and in
user-space~\cite{opus,ginseng,polus,upstare,ekiden,kitsune}.

The dynamic update process typically consists of two stages---code update and
state transfer---and existing systems employed different approaches to handle
these.

Ginseng~\cite{ginseng} and K42~\cite{k42} employ indirection to enable the code
update. This simplifies the code update mechanism, but the indirection
introduces performance overhead during normal execution. Ginseng uses a
specialised compiler which creates a ``dynamically updatable program'' where
all function symbols are replaced with function pointers and all direct
function calls are replaced with indirect call through these pointers. During
update, these pointers are updated to point to the new version of the code,
which is loaded using \lstinline`dlopen`. K42 operating system has a
per-address-space object translation table used for all object invocations.
When the new kernel module is loaded, the translation table entries are updated
to point to the new version.

OPUS~\cite{opus}, DynAMOS~\cite{dynamos}, POLUS~\cite{polus} and
Ksplice~\cite{ksplice} use binary rewriting instead of indirection to replace
the entry point of the functions being updated with a jump to a trampoline.
Compared to indirection, this approach does not require the use of a
specialised compiler, but it is highly platform-dependent. We use the same
approach in \varan to handle virtual system calls (\S\ref{sec:vsyscall}).

The aforementioned systems treat individual functions or objects as the unit of
code for updates. Such systems are incapable of handling functions that rarely
end (\eg \lstinline`main` or functions that contain event-handling loops). This
was addressed by systems such as UpStare~\cite{upstare}, Ekiden~\cite{ekiden}
and Kitsune~\cite{kitsune}, which update the code by loading an entirely new
program instead of replacing individual functions. After the new version has
been loaded, either through \lstinline`fork-exec` or through
\lstinline`dlopen`, the execution needs to be restarted from the previous
point. UpStare uses stack reconstruction to achieve this by replacing the stack
frames for old functions with their new versions. Ekiden and Kitsune on the
other hand rely on manual approach requiring the programmer to direct the
execution into the equivalent update point in the new version.

\textsc{Proteos}~\cite{proteos} and TTST~\cite{ttst} use a custom
instrumentation component implemented on top of LLVM compiler framework to
automate state transfer and state management to a large degree, only requiring
programmer-provided state transfer callback for complex state changes
(and cases of pointer ambiguity). Unlike previous systems which use in-place
live update strategies, \textsc{Proteos} and TTST confine different program
versions into separate processes which simplifies state managements and enables
simple fail-over in case of a failed update. While \textsc{Proteos} is a new
research operating system, TTST is implemented entirely in Linux user-space.

Compared to \mx and \varan, in DSU systems the two versions co-exist only for
the duration of the software update, but DSU and the \rem component of \mx face
similar challenges when switching execution from one version to another.  We
hope that some of the technique developed in DSU research will also benefit the
recovery mechanism of \mx and vice versa.

%Dynamic software updating (DSU) systems such as Ginseng~\cite{ginseng},
%UpStare~\cite{upstare} or Kitsune~\cite{kitsune} are concerned with the problem
%of updating programs while they are running.  As opposed to \mx, the two
%versions co-exist only for the duration of the software update, but DSU and the
%\rem component of \mx face similar challenges when switching execution from one
%version to another.  We hope that some of the technique developed in DSU
%research will also benefit the recovery mechanism of \mx and vice versa.

\subsection{Update management and distribution}

Prior work on improving software updating has looked at different aspects
related to managing and deploying new software versions.

Beattie~\etal~\cite{beattie2002} considered the issue of timing the
application of security updates---patching too early could result in breaking
the system by applying a broken patch, patching too late could on the other
hand lead to the risk of penetration by an attacker exploiting a well known
security issue. Using the cost functions of corruption and penetration, based
upon real world empirical data, they have shown that 10 and 30 days after the
patch's release date are the optimal times to apply the patch to minimise the
risk of a defective patch.  Such a delay still opens a lot of opportunities for
potential attackers.

Crameri \etal~\cite{crameri:updates} proposed a framework for staged
deployment. This framework integrates upgrade deployment, user-machine testing
and problem reporting into the overall upgrade process. The framework itself
clusters user machines according to their environment and software updates are
tested across clusters using several different strategies allowing for the
deployment of complex upgrades.

A solution inspired by $N$-version programming was proposed by Michael
Franz~\cite{unibus:nspw10}. Instead of executing multiple versions in parallel,
the author suggested distributing a unique version of every program to every
user. Such versions should be created automatically by a ``multicompiler'',
and distributed to users through ``App Store''. This would increase security as
it would be much more difficult to generate attack vectors by reverse
engineering of security patches for these diversified versions.

%To make such a solution work in practice, the way to manage large number of
%different versions and the overall update process is needed.
%Crameri~\etal~\cite{crameri:updates} proposed a framework for staged deployment
%which integrates upgrade deployment, user-machine testing and problem reporting
%into the overall upgrade process. The framework itself clusters users'
%machines according to their environment, tests the upgrades using cluster
%representatives and allows deployment of complex upgrades.

%Beattie~\etal~\cite{beattie2002} considered the issue of timing the application
%of security updates---patching too early could result in breaking the system by
%applying broken patch, patching too late could on the other hand lead to risk
%of penetration by an attacker exploiting a well known security issue. Using the
%cost functions of corruption and penetration, based upon real world empirical
%data, they have shown that 10 and 30 days after the patch's release date are
%the optimal times to apply the patch to minimize the risk of defective patch.
%Such delay still opens a lot of opportunities for potential attackers.

In relation to this work, \mx and \varan try to ease the decision of applying a
software update, because incorporating a new version should only increase the
reliability and availability of the overall multi-version application.
However, in practice the number of versions that can be run in parallel is
dictated by the number of available resources (\eg the number of available CPU
cores), so effective update strategies are still needed in this context, and
work in this area could provide helpful solutions to this problem.

%In relation to this work, \mx tries to encourage users to always apply
%a software update, but it would still benefit from effective update
%strategies to decide what versions to keep when resources are
%limited.

Many large-scale services, such as Facebook and Flickr use a {\it continuous
deployment} approach, where new versions are continuously released to
users~\cite{johnson2009,flickr}, but each version is often made accessible only
to a fraction of users to prevent complete outage in case of newly introduced
errors.  While this approach helps minimise the number of users affected by new
bugs, certain bugs may manifest themselves only following prolonged operation,
after the release has been deployed to the entire user base.  We believe our
proposed approach is complementary to continuous deployment, and could be
effectively combined with it.

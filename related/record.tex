\section{Record \& replay}
\label{related:record}

Event streaming in \varan can be seen as a variant of record-replay. Record and
replay technique has been an active research topic for more than forty years
and different approaches were developed over this period. However, unlike
traditional record-replay systems that require a persistent log, \varan keeps
the shared ring buffer in memory, and deallocates events as soon as they are
not needed, which minimises performance overhead and space requirements in the
NVX context.

%Record \& replay has been an active topic for many years and there has been
%numerous systems implemented at different layers, they all shared the same
%basic concepts.

BugNet~\cite{bugnet}, Jockey~\cite{jockey}, liblog~\cite{geels06}, and
R2~\cite{r2} use a library approach where the replay tool is provided as a
library which is inserted into a target application to allow for record and
replay. Jockey and liblog try to minimize the interference with the application
being recorded to ensure that the application behaves the same with and without
the tool, and rely solely on intercepting of a set of C library calls. BugNet
provides a special API applications must use to support record and replay while
R2 requires developers to annonate functions which can be recorded and replayed
correctly. RecPlay~\cite{recplay} also uses a library approach, but focuses on
debugging of nondeterministic parallel programs. The approach combines
record-replay mechanism with automatic data race detection implemented using
vector clocks.

Flashback~\cite{flashback} has a similar focus, enabling deterministic
debugging, but rather than library, it is implemented as a kernel extension.
The key concept is the use of shadow processes, which duplicate the state of a
process in memory allowing fast rollback of the debugged program with small
overhead. Scribe~\cite{scribe} is another kernel-based solution focused on
transparent, low-overhead record-replay with the ability to switch from
replayed to live execution. Same as in case of Flashback, while the
kernel-based implementation results in lower performance overhead, it makes
these systems less easy to develop and debug, and less safe to use compared to
library based approaches.

RR~\cite{rr} is similar to \varan by replicating an application into multiple
instances for fault tolerance. Same as \varan, it also uses a variant of
record-replay to synchronize the primary replica with others. Compared
to \varan, RR is focused on fail-stop scenarios and does not support
running different versions of the same application. The preliminary prototype
has been implemented as a Linux kernel extension, with all the associated
limitations as discussed above, and evaluated using only a single benchmark.

ReVirt~\cite{revirt}, SMP-ReVirt~\cite{smp-revirt} and iDNA~\cite{idna}
implement record/replay mechanism as a part of virtual machine hypervisor,
allowing to record and replay the entire system, including the operating system
and the target applications. ReVirt is built on top of UML, which lacks the
support for multiprocessor virtual machines. This limitations has been addressed
in SMP-ReVirt which builts on top of Xen. iDNA uses a custom virtualization
engine called Nirvana. The virtualization-based record-replay mechanism allow
faithful replay of every aspect of the application's environment, including
scheduling decisions of the operating system, with small performance overhead.
On the other hand, these systems are more difficult to deploy and their use is
more expensive as they require an entire operating system to record a single
application.

There are numerous record-replay tools focused on restricted programming models
such as distributed shared memory~\cite{russinovich96}, MPI~\cite{rrmpi} and
programming language runtimes such as Java~\cite{choi98}.

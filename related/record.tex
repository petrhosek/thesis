\section{Record \& replay}
\label{related:record}

Event streaming in \varan can be seen as a variant of record \& replay. Record
\& replay technique has been an active research topic for more than forty years
and different approaches have been developed over this period. However, unlike
traditional record \& replay systems that require a persistent log, \varan keeps
the shared ring buffer in memory, and deallocates events as soon as they are
not needed, which minimises performance overhead and space requirements in the
NVX context.

%Record \& replay has been an active topic for many years and there has been
%numerous systemssimplemented at different layers, they all shared the same
%basic concepts.

BugNet~\cite{bugnet}, Jockey~\cite{jockey}, liblog~\cite{geels06}, and
R2~\cite{r2} use a library approach where the replay tool is provided as a
library which is inserted into a target application to allow for record \&
replay. Jockey and liblog try to minimise the interference with the application
being recorded to ensure that the application behaves the same with and without
the tool, and rely solely on intercepting of a set of C library calls. BugNet
provides a special API applications must use to support record \& replay while
R2 requires developers to annotate functions which can be recorded and replayed
correctly. RecPlay~\cite{recplay} also uses a library approach, but focuses on
debugging of non-deterministic parallel programs. The approach combines the
record \& replay mechanism with automatic data race detection implemented using
vector clocks.

Flashback~\cite{flashback} has a similar focus, enabling deterministic
debugging, but rather than library, it is implemented as a kernel extension.
The key concept is the use of shadow processes, which duplicate the state of a
process in memory allowing fast rollback of the debugged program with small
overhead. Scribe~\cite{scribe} is another kernel-based solution focused on
transparent, low-overhead record \& replay with the ability to switch from
replayed to live execution. While the kernel-based implementation results in
lower performance overhead, it makes these systems less easy to develop and
debug, and less safe to use compared to library based approaches.

RR~\cite{rr} is similar to \varan by replicating an application into multiple
instances for fault tolerance. As \varan, it also uses a variant of
record \& replay to synchronise the primary replica with others. Unlike \varan, RR
is focused on fail-stop scenarios and does not support running different
versions of the same application. The preliminary prototype has been
implemented as a Linux kernel extension, with all the associated limitations
discussed above, and evaluated using only a single benchmark.

ReVirt~\cite{revirt}, SMP-ReVirt~\cite{smp-revirt} and iDNA~\cite{idna}
implement record \& replay mechanism as part of a virtual machine hypervisor,
allowing to record and replay the entire system, including the operating system
and the target applications. ReVirt is built on top of UML, which lacks support
for multiprocessor virtual machines. This limitation has been addressed in
SMP-ReVirt which builds on top of Xen. iDNA uses a custom virtualisation engine
called Nirvana. The virtualisation-based record \& replay mechanism allow faithful
replay of every aspect of the application's environment, including the
scheduling decisions of the operating system, with a small performance
overhead.  On the other hand, these systems are more difficult to deploy and
their use is more expensive as they require an entire operating system to
record a single application.

Aftersight~\cite{aftersight}\marginnote{14} uses virtualisation-based record \&
replay mechanism to enable running heavyweight analyses on production workloads
by decoupling the analysis from normal execution. The system supports three
different modes of operation: in the synchronous safety and best-effort safety
modes, the analysis is being run in parallel with the workload, while in the
offline mode, the execution log is stored for later analysis akin to
traditional record \& replay systems. The synchronous safety model resembles
the \mx's mode of operation where the workload and the analysis are run in
lockstep.  The best-effort safety mode is similar to \varan, and in particular
the live sanitisation deployment model described in
Section~\ref{sec:sanitization}, where there is no synchronisation between the
output of the workload and the analysis.

Paranoid Android~\cite{paranoid-android}\marginnote{14} uses a similar approach
for protecting mobile phones---rather than running the attack detection
mechanism directly on the device, negatively affecting the performance and the
battery life, the phone records a minimal execution trace, and sends it to the
server, where it is replayed inside a virtual machine. This enables the use of
multiple attack detection techniques that can be all run in parallel. Same as
\mx, Paranoid Android also uses \ptrace mechanism for system call interception.
Similarly to \varan's event streaming mechanism, Paranoid Android uses
asynchronous record \& replay mechanism to decouple the execution of the
monitored application from its replicas.

%There are numerous record \& replay tools focused on restricted programming
%models such as distributed shared memory~\cite{russinovich96}, MPI~\cite{rrmpi}
%and programming language runtimes such as Java~\cite{choi98}.

% \cite{shadowhoneypot}
% \cite{aftersight}
% \cite{paranoid-android}

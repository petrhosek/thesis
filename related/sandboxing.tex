\section{Sandboxing}
\label{related:sandboxing}

$N$-version execution environment typically uses some form of application
sandboxing. The goal is to isolate the running application from underlying
system and especially, to prevent software failures in any of the versions
being executed from affecting the rest of the system including other versions.

Prior sandbox architectures include both kernel-based
mechanisms~\cite{tron,remus,subdomain,cots-hardening} and system call
interposition monitors~\cite{wily-hacker,mapbox,jain1999,provos2002,mbox}.
Other mechanisms have been also used to build sandboxes, such as binary
rewriting~\cite{vx32,true:sp12}, system transactions~\cite{txbox} or virtual
machine introspection~\cite{garfinkel:vmi}.

\subsection{System call interposition}
\label{related:syscalls}

Regardless of the application, most operating system resources such as files
and sockets can only be accessed through the system call interface.  Therefore,
this interface is often a target for monitoring and regulation. Both \varan and
\mx operate at the system call interface layer synchronizing system calls
performed by different application versions. \varan draws inspiration from the
Ostia delegating architecture~\cite{ostia}, and from the selective binary
rewriting approach implemented by \emph{BIRD}~\cite{bird} and
\textsf{seccompsandbox}.\footnote{\url{https://code.google.com/p/seccompsandbox/}}
\mx on the other hand uses the \textsf{ptrace} interface same as many of the
existing monitors such as Orchestra~\cite{orchestra09} or
Tachyon~\cite{tachyon12}.

System call interposition has been an active area of research and there are
there many different ways in which the system call interception mechanism can
be implemented, including dynamic libraries loaded through the preloading
mechanism (\ie using \lstinline`LD_PRELOAD` or
\lstinline`DYLD_INSERT_LIBRARIES` variables) and custom C
libraries~\cite{plash}, facilities provided by the operating system (\ie
\ptrace interface or \lstinline`/proc` subsystem), kernel extensions and binary
rewriting.

Janus~\cite{wily-hacker} was one of the first systems to implement a confinement
mechanism by interposing on the system calls made by the application via the
standard tracing mechanisms, in particular \ptrace. The similar architecture
has been later adopted by MAPbox~\cite{mapbox} and Consh~\cite{consh}. However,
all of these systems suffered from numerous limitations posed by the \ptrace
interface~\cite{janus}. Jain and Sekar~\cite{jain1999} tried to address some of
these issues to build a more efficient user-space confinement mechanism.  A
later Janus~\cite{janus} evolution implemented a tracing mechanism for Solaris
using \lstinline`/proc` subsystem. To address some of the \ptrace shortcomings,
Systrace~\cite{provos2002} used a loadable kernel module.  Some system call
interpositon mechanisms were implemented entirely in
kernel~\cite{subdomain,cots-hardening}. Ostia~\cite{ostia} has shown how to
implement a secure and efficient system call interposition mechanism by combing
kernel extension with a custom user-space binary rewriting mechanism avoiding.
More recently, \textsc{Mbox}~\cite{mbox} implemented sandboxing mechanism by
combining \ptrace and \textsf{seccomp/bpf} to allow for selective system call
filtering.

Numerous intrusion detection mechanisms rely on analysis of system call
sequences looking for potential
anomalies~\cite{syscall-seq,wespi00,sekar01,gao04,sandeep06}. These systems
rely on system call tracing, which is similar to system call interposition,
although there is no need for modyfing or denying system calls.

Orchestra~\cite{orchestra09} and Tachyon~\cite{tachyon12} used system call
interposition through \ptrace interface to implement $N$-version runtimes.  To
improve the performance overhead, one of the \ptrace shortcomings when reading,
Orchestra uses shared memory injected into the application's address space to
read the values of indirect system call arguments, while Tachyon uses cross
memory attach, same as \mx (\S\ref{sec:mxm}), for the same purpose.

Due to their importance, especially when used for application sandboxing,
system call monitors themselves can also become targets for attackers.
Garfinkel~\cite{garfinkel:traps} and Watson~\cite{watson07} described common
vulnerabilities in system call monitors and the ways in which these could be
exploited.

%User-mode Linux~\cite{dike:uml}, a mechanism which allows running multiple
%virtual Linux systems on a Linux host has been also implemented using the
%\ptrace interface.

% System call interposition has been an active area of research and there are
% there many different ways in which the system call interception mechanism can
% be implemented, including dynamic libraries loaded through the preloading
% mechanism and custom C libraries, facilities provided by the operating system,
% kernel extensions and binary rewriting.

% MAPbox~\cite{mapbox}, Systrace~\cite{provos2002}, Jailer~\cite{jailer},
% BlueBoX~\cite{bluebox}, \textsc{Mbox}~\cite{mbox} used \ptrace to implement
% system Linux call monitors for intrusion detection and confinement akin to
% network firewalls.  Jailer~\cite{jailer} provides a custom kernel module as an
% alternative to the \ptrace interface to address some of its shortcomings.
% Janus~\cite{wily-hacker,janus} used the \lstinline`/proc` subsystem to
% implement a similar confinement mechanism for Solaris. 

% \subsection{Library-based mechanisms}

% System calls are rarely performed directly by the application, which rather use
% the wrapper functions provided by the C library. By linking to a custom library,
% which provides a custom version of these functions, we could intercept system
% calls performed by the appliction. This could be done either statically at link
% time, or dynamically at runtime (\eg using \lstinline`LD_PRELOAD` or
% \lstinline`DYLD_INSERT_LIBRARIES` variables). The former approach has been used
% by systems such as Plash~\cite{plash}, a sandboxing system for Linux
% implemented as a modified version of \gnu C library , while the latter approach
% has been used by research projects such as RCV~\cite{shepherding:pldi14}. The
% major benefit of this approach is efficiency and relative ease of
% implementation. However, the mechanism could be easily bypassed (\eg by
% invoking system calls directly) by the application making it unsuitable for
% security applications.

% \subsection{\ptrace}

% \ptrace is an interfaces provided by most UNIX-like operating systems,
% including Linux, providing means by which a process might observe and control
% another process. The relative ease of use makes \ptrace a popular choice for
% implementing system system call monitors, including \mx.  Systems such as
% MAPbox~\cite{mapbox}, Systrace~\cite{provos2002}, Jailer~\cite{jailer},
% BlueBoX~\cite{bluebox}, \textsc{Mbox} use \ptrace to implement systems for
% intrusion detection and confinement.  Orchestra~\cite{orchestra09} and
% Tachyon~\cite{tachyon12} used \ptrace to implement $N$-version runtimes.
% User-mode Linux~\cite{dike:uml}, a mechanism which allows running multiple
% virtual Linux systems on a Linux host has been also implemented using the
% \ptrace interface.

% Despite \ptrace's popularity, especially for rapid
% prototyping~\cite{spillane07}, there are numerous drawbacks hindering its use
% for pratical deployment: a significant performance overhead due to large number
% of context switches, problematic support for multithreaded applications, and
% the lack of filtering mechanism allowing to intercept only system calls of
% interest. The performance overhead could be partially improved by the use of
% more efficient mechanism for copying memory from/to the monitored process, such
% as shared memory as in case of Orchestra~\cite{orchestra09}, or cross memory
% attach as in case of \mx (\S\ref{sec:mxm}). The lack of filtering mechanism
% could be addressed by combining \ptrace with \textsf{seccomp/bpf} mechanism as
% shown by \textsc{Mbox}~\cite{mbox}.

% \subsection{Kernel-based mechanisms}

% An alternative to \ptrace is to implement the system call monitor entirely or
% partially in kernel space. Jailer~\cite{jailer} provides a custom kernel module
% as an alternative to the \ptrace interface. Strata~\cite{cox2006} implements
% $N$-version execution monitor inside the kernel. Ostia~\cite{ostia} combines
% a custom kernel module with user space runtime for application sandboxing.

% While this approach has numerous advantages compared to other approaches, such
% as minimal performance overhead and direct access to the application's
% execution context and memory contents, there are several drawbacks as well.
% First, this approach requires kernel patches and/or new new kernel modules
% which complicates the development, limits portability across different
% operating systems or even different kernel versions, and hinders
% maintainability. Second, the monitor must be run in privileged mode, which
% means that bugs in the implementation may compromise the system stability and
% security. Furthermore, it also makes it difficult for regular users to deploy
% and use such monitors.

% \subsection{Binary rewriting}

% Binary rewriting technique allow transforming the executable (either statically
% or dynamically) altering its functionality; as such, it can be used to
% implement system call monitor by rewriting all system call instructions into a
% control flow transfer instructions (\eg a
% \lstinline[language={[x64]Assembler}]`jmp` instruction). Existing monitors
% were built either on top of existing binary rewriting systems such as Pin~\cite{pin}
% or DynamoRIO~\cite{dynamorio02} as in case of $\delta$
% execution~\cite{onlinevalidation}, or using a purpose built binary translation
% mechanism as in case of Vx32~\cite{vx32},
% \textsf{libdetox}~\cite{libdetox:vee11}, BIRD~\cite{bird}, and
% \textsf{seccompsandbox}. The advantage of binary rewriting-based monitors is
% relatively low performance overhead, especially in the case of purpose built
% rewriters as we have demonstrated in case of \varan, the main disadvantage is a
% high complexity of implementing such monitor.

\subsection{Software fault isolation}
\label{related:sfi}

The use of software-based fault isolation for executing untrusted code has been
first described in~\cite{sfi:sosp93} for the RISC machines with simple
instruction set. However, the first effective implementation of software fault
isolation for the CISC architecture has been shown only much later
in~\cite{cisc-sfi:usenix-sec06}.

These concepts have been later used by several sandboxes designed for isolation
of plugins on the web. Xax~\cite{douceur08} uses system call interposition
together with address space isolation to leverage existing libraries and
programs on the web. Native Client~\cite{nacl} requires the code to be
recompiled to a restricted subset of the x86/ARM ISA which can be checked prior
running and confines the application using segmentation and memory protection.
The problem of protecting trusted code from the untrusted code addressed by
both of these sandboxes is orthogonal to traditional system-level sandboxing
and these systems rely on separate sandboxing mechanism for that purpose (\eg
Native Client uses either \textsf{seccomp} or \textsf{seccomp-bpf} as an
"outer" sandbox layer).

% MAPbox~\cite{mapbox}, Systrace~\cite{provos2002} and BlueBoX~\cite{bluebox}
% implement Linux system call monitors for intrusion detection and application
% confinement using the \ptrace interface. Jailer~\cite{jailer} provides a custom
% kernel module as an alternative to \ptrace to address some of its shortcomings.
% \textsc{Mbox}~\cite{mbox} combines \ptrace with \textsf{seccomp/bpf} to allow
% for selective system call filtering. Janus~\cite{wily-hacker,janus} implements
% similar confinement mechanism for Solaris using the \lstinline`/proc`
% subsystem.

%Sandboxing~\cite{bascule,drawbridge}.
%Capsicum~\cite{capsicum}.

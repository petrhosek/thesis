\chapter{Conclusion}
\label{chap:conclusion}

Software updates present a difficult challenge to the software maintenance
process. Too often, updates result in failures, and users face the
uncomfortable choice between using an old stable version which misses recent
bug and security fixes, and using a new version which improves the software in
certain ways, only to introduce other bugs and security vulnerabilities.

We tackle this problem using a simple but effective \emph{multi-version
execution} based approach. Whenever a new update becomes available, instead of
upgrading the software to the new version, we run the new version in parallel
with the old one; by carefully coordinating their executions and selecting the
behaviour of the more reliable version when they diverge, we create a more
secure and dependable multi-version application. With the widespread
availability of multi-core processors, multi-version execution might become a
viable approach for increasing the reliability and availability of updated
software systems.

In this thesis, we have proposed two different schemes for implementing the
multi-version execution technique. The first is a failure recovery scheme,
which allows programs to survive errors introduced by software updates. The
second is a transparent failover scheme, which allows $N$ versions to be run in
parallel; if one these versions fails, the mechanism transparently switches to
another version without any disruption in service.

To enable these scenarios, we need a runtime monitor that enables the execution
of multiple versions in parallel, coordinates and synchronises their execution,
making them appear as a single application to any outside entities.
Unfortunately, existing monitors impose a large performance overhead and none
of them scales well with the number of versions, since the runtime monitor
usually acts as a performance bottleneck. Moreover, none of the existing
monitors handles the case when the execution of versions run in parallel
diverges. We have presented two different designs for building runtime monitor
which address these drawbacks; the first, called \mx is focused on surviving
crashes caused by bugs introduced in software updates and the second, called
\varan is aimed at running a large number of versions side-by-side with low
overhead.

\mx is a multi-version execution system implementing a novel fault recovery
mechanism targeting application updates which might introduce potential crash
bugs. We have shown how \mx can be successfully applied in practice by
recovering from real crashes in several real applications such as \gnu
\coreutils, a set of user-level UNIX applications; \lighttpdgen, a popular web
server used by several high-traffic websites; and \redis, an advanced key-value
data store server used by many well-known services.

\varan is a multi-version execution framework that combines binary rewriting
with a novel event-streaming architecture to significantly reduce performance
overhead and scale well with the number of versions, without relying on
intrusive kernel modifications. Our evaluation shows that \varan can run
multi-version applications based on popular C10k network servers with only a
modest performance overhead, and can be effectively used to increase software
reliability.

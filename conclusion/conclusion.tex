\chapter{Conclusion}
\label{chap:conclusion}

Software updates are an integral part of the software development and
maintenance process, but unfortunately they present a high risk, as new
releases often introduce new bugs and security vulnerabilities; as a
consequence, many users refuse to upgrade their software, relying instead on
outdated versions, which often leave them exposed to known software bugs and
security vulnerabilities.

In this thesis we have proposed a novel multi-version execution approach, a
variant of $N$-version execution, for improving the software update process.
Whenever a new program update becomes available, instead of upgrading the
software to the newest version, we run the new version in parallel with the old
one, and carefully synchronise their execution to create a more reliable
multi-version application.

\todo{Describe the two different schemes for multi-version execution.}
\todo{Mention the results of the empirical study.}

We have also shown two different approaches for implementing the multi-version
execution approach: \mx, focused on recovering from crashes caused by the
faulty software updates; and \varan, focused on running large number of
versions in parallel with minimal performance overhead.

\mx uses static binary analysis, system call interposition, lightweight
checkpointing and runtime state manipulation to implement a novel fault recovery
mechanism, which allows for recovery of the crashing version using the code
of the other, non-crashing version. We have shown how \mx can be applied
successfully to several real applications, and recover from real crashes
reported by their users.

\varan combines selective binary rewriting with high-performance event
streaming to significantly reduce performance overhead, without sacrificing the
size of the trusted computing base, nor flexibility or ease of debugging.  Our
experimental evaluation has demonstrated that \varan can run C10k network
servers with low performance overhead, and can be used in various scenarios
such as transparent failover and live sanitisation.

\todo{Briefly talk about other applications and future work.}

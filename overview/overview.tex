\chapter{Overview}
\label{chap:overview}

With the growing size of the modern computer software systems, the number of
errors present in such systems grows steeply.  Moreover, with increasing size
of the code base, the possibility of finding bugs, both manually and
automatically, is decreasing very rapidly. On the other hand, increasing
reliance on computers and computer software systems in every aspect of human
lives implies strong need for reliable software systems.

Software updates are an integral part of the software maintenance process, but
unfortunately present a high risk, with many users and administrators refusing
to upgrade their software and relying instead on outdated versions, which often
leaves them and their systems exposed to software bugs and security
vulnerabilities.

The Hearbleed bug~\cite{heartbleed} is a prime example of this behavior. A
fixed version of OpenSSL was released on the same day the Heartbleed bug was
publicly disclosed.  However, a month after the publication, despite the wide
media coverage, 1.5\% (12,043) of the 800,000 most popular sites in the world
were still using the vulnerable version of OpenSSL making their servers
susceptible to attacks~\cite{heartbleed-prevalent}.

Nowadays, many services including the large-scale services such as Google,
Facebook or Amazon employ \emph{continuous deployment}, whereby new versions
are continuously released to users~\cite{johnson2009}. Multiple versions can be
released throughout the day, and each version is often accessible only to a
fraction of users to prevent complete outage in case of an error caused by one
of the new versions.

While this approach helps minimize the number of users affected by newly
introduced bugs, it is not foolproof. In particular, bugs introduced by new
releases may manifest themselves only in coincidental cases or following
prolonged operation, after the release has been deployed to the entire user
base. Errors and failures affecting majority of the user-base therefore occurs
even to these applications. Such outages might cause significant financial
losses and frustration among users, and it is therefore critical to avoid them.

An example of such disruption is the recent Verizon
outage~\cite{verizon-outage2014} of the billing and activation system,
affecting large number of users which were unable to log into their accounts.
After the incident, Verizon issued the following statement:
\begin{quotation}
The Verizon Wireless billing system was fully restored early today, shortly
after midnight. The issue affecting some customer access to account information
was an unintended consequence of a software update performed by the company on
its billing system two days ago.
\end{quotation}
The outage, which according to reports lasted for 48 hours, was so impactful
that it generated a lot of negative reactions from frustrated customers
including its own \#verizonfail and \#verizonoutage Twitter hashtags.  This as
well as other similar incidents (\eg Blogger~\cite{blogger-incident2011} or
Facebook~\cite{facebook-incident2010}) show the possible consequences of failed
software updates.

The problems caused by failed software updates are also not restricted to
server applications or services. \vim\footnote{\url{http://www.vim.org/}},
arguably one of the most popular text editors, distributed with the popular
Ubuntu Linux distribution has been also affected by failed software update in
the past.  In version 7.1.127, while trying to fix a memory leak, \vim
developers introduced a double \textstt{free} bug that caused \vim to crash
whenever the user tried to use a path completion feature.  This bug made its
way into Ubuntu 8.04, affecting a large number of
users.\footnote{\url{https://bugs.launchpad.net/ubuntu/+source/vim/+bug/219546}}

To tackle the software update problem, we propose \emph{multi-version
execution}, a novel technique
%, which fits within the overall theme of $N$-version programming~\cite{nvp77,chen1995},
designed to improve the software update process to increase software
availability and reliability by exploiting the abundance of resources (\eg idle
processor time) made available by the modern highly parallel platforms.
Whenever a new system update becomes available, instead of upgrading the
software to the newest version, we \emph{run the new version in parallel with
the old}. As new versions arrive, we continue to execute them in parallel with
the existing ones, until all available resources have been exhausted, or a
user-specified threshold has been reached.  At that point, we can either
discard the very oldest versions, or we can use more sophisticated replacement
strategies.

We believe that multi-version execution updates is a timely solution in the
context of today's computing platforms~\cite{multiplicity}. The last decade has
seen the emergence of new hardware platforms, ranging from multi-core
processors to large-scale data centers, which provide an abundance of
computational resources and a high degree of parallelism. These platforms
already benefit applications with a great amount of inherent concurrency.
However, there has been relatively little progress in exploiting this abundance
of resources to improve software reliability and availability, especially in
the context of software updates.

Typical data centers are designed for peak load to ensure service-level
agreements are met. Servers are rarely completely idle but they do not operate
near their maximum utilization; most of the time, they operate at between 10\%
and 50\% of their maximum utilization levels. However, even an energy-efficient
server consumes half its full power when doing virtually no work---and for
normal servers, this ratio is much worse~\cite{barroso2007}.  Therefore,
instead of dynamically switch off unused servers which does not prove to be
effective, we aim to use the abundance of these resources (\ie processing
power) to increase software availability and reliability.

%In recent years,
%we have witnessed the emergence of new technology---ranging from multicore
%processors to large-scale data centers---which provide an abundance of
%computational resources and a high degree of parallelism.  Furthermore, these
%resources are often left idle, in which state they consume (read waste) a large
%fraction of their full-utilization energy levels.  For example, recent
%studies~\cite{barroso2007} have shown that servers in a data center usually
%operate at between 10\% and 50\% of their maximum utilization; however, even an
%energy efficient server consumes half its full power when doing virtually no
%work---and for regular servers, this ratio is much worse.

Our approach aims to improve the software update process by taking advantage of
idle resources---such as idle cores on a CPU and idle servers in a data
center---to run multiple versions of an application in parallel, with the goal
of improving the overall reliability and security of the software being
upgraded.  Our current focus is on multi-core processors, but we believe this
solution could be adapted to work on other parallel platforms as well.
Furthermore, this update mechanism could be extended to work with a large
number of versions running in parallel and configured to balance conflicting
requirements such as performance, reliability and energy consumption.  

%%%%%%%%%%%%%%%%%%%%%%%%%%%%%%%%%%%%%%%%%%%%%%%%%%%%%%%%%%%%%%%%%%%%%%%%%%%%

%We aim to tackle this problem using a simple but effective approach based on
%application-level virtualization.  Whenever a new program update becomes
%available, instead of upgrading the software to the newest version, we {\it
%run the new version in parallel with the old}.  As new versions arrive, we
%continue to execute them in parallel with the existing ones, until all
%available resources have been exhausted, or a user-specified threshold has
%been reached.  At that point, we can either discard the very oldest versions,
%or we can use more sophisticated replacement strategies.  This approach can
%be extended to work with a large number of versions run in parallel, and can
%be applied to several different platforms, ranging from multicore processors
%to large-scale data centers.  

%% There are three main challenges that we need to address to make this
%% approach viable.  First, we need to implement mechanisms for
%% coordinating the parallel execution of multiple program versions.
%% Second, when executions diverge, we need to select the output of the
%% more reliable one, and merge them back once executions converge again.
%% Finally, we need to ensure that the overall system is able to scale up
%% and down the number of program versions run in parallel in order to
%% balance conflicting requirements such as performance, reliability, and
%% energy consumption.

%The rest of this paper is organized as follows. Section~\ref{sec:background}
%discusses related work and  Section~\ref{sec:motivation} motivates our
%approach by using a scenarios based on Google Chrome web browser and Vim text
%editor.  Then, Section~\ref{sec:challenges} discusses the main challenges that
%we need to address to make this approach work in practice and
%Section~\ref{sec:prototype} presents a protype implementing safe updates
%in the context of multi-core processors. Finally, Section~\ref{sec:opportunities}
%describes the potential future work and Section~\ref{sec:conclusion} concludes.

%The rest of this paper is organized as follows. Section~\ref{sec:background}
%discusses related work and  Section~\ref{sec:motivation} motivates our
%approach by using a scenarios based on Google Chrome web browser and Vim text
%editor.  Then, Section~\ref{sec:challenges} discusses the main challenges that
%we need to address to make this approach work in practice and
%Section~\ref{sec:prototype} presents a protype implementing safe updates
%in the context of multi-core processors while Section~\ref{sec:evaluation}
%shows a case study for our approach. Finally, Section~\ref{sec:opportunities}
%describes future work and Section~\ref{sec:conclusion} concludes.

%% Following work presents our approach approach, its advantages and
%% disadvantages. Furthermore, it provides comparison and summarizes
%% differences from the existing solutions aiming to achieve similar
%% goals.

\section{Problem statement}

\section{Organization}

\section{Previous work}

The text of this thesis is based on, and borrows heavily from the work and the
writing in the following papers:

\begin{itemize}
\item \fullcite{mvupdates12}
\item \fullcite{mx}
\item \fullcite{covrig:issta14}
\item \fullcite{varan:asplos15}
\end{itemize}

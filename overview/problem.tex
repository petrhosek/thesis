\section{Solution}
\label{overview:solution}

We tackle the problem of faulty software updates using a simple but effective
\emph{multi-version execution} based approach. Whenever a new update becomes
available, instead of upgrading the software to the new version, we run the new
version in parallel with the old one; by carefully coordinating their
executions and selecting the behaviour of the more reliable version when they
diverge, we create a more secure and dependable multi-version application.

% To tackle the software update problem, we propose \emph{multi-version
% execution}, a novel technique, which fits within the overall theme of
% $N$-version programming~\cite{avizienis:nvp,chen1995}. The goal of this
% technique is to improve the software update process to increase software
% availability and reliability by exploiting the abundance of resources (\eg idle
% processor time) made available by modern highly parallel platforms. Whenever a
% new system update becomes available, instead of upgrading the software to the
% newest version, we run the new version in parallel with the old one; by
% carefully coordinating their executions and selecting the behaviour of the more
% reliable version when they diverge, we create a more secure and dependable
% multi-version application. As new versions arrive, we continue to execute them
% in parallel with the existing ones, until all available resources have been
% exhausted, or a user-specified threshold has been reached.  At that point, we
% can either discard the very oldest versions, or we can use more sophisticated
% replacement strategies.

The goal of multi-version execution is to increase availability and reliability
of updated software by exploiting the abundance of resources (\eg idle
processor time) made available by modern highly parallel platforms.  We believe
that multi-version execution updates is a timely solution in the context of
today's computing platforms~\cite{multiplicity}. The last decade has seen the
emergence of new hardware platforms, ranging from multi-core processors to
large-scale data centres, which provide an abundance of computational resources
and a high degree of parallelism. These platforms already benefit applications
with a great amount of inherent concurrency.  However, there has been
relatively little progress in exploiting this abundance of resources to improve
software reliability and availability, especially in the context of software
updates.

Typical data centres are designed for peak load to ensure service-level
agreements are met. Servers are rarely completely idle but they do not operate
near their maximum utilisation; most of the time, they operate at between 10\%
and 50\% of their maximum utilisation levels. However, even an energy-efficient
server consumes half its full power when doing virtually no work---and for
normal servers, this ratio is much worse~\cite{barroso2007}.  Therefore,
instead of dynamically switching off unused servers which is not effective, we
aim to use the abundance of these resources (\ie processing power) to increase
software availability and reliability.

In this thesis, we present two different schemes for implementing this
multi-version execution technique. The first is a \emph{failure recovery scheme},
which allows program to survive errors introduced by software updates. The
mechanism is completely autonomous and operates on the off-the-shelf-binaries
making it easy to use and deploy. This scheme is limited to two versions run in
parallel. The second is a \emph{transparent failover scheme}, which allows $N$
versions to be run in parallel; if the primary version fails, the mechanism
transparently switches to another version without any disruption in service.
This scheme is designed to have minimal performance overhead enabling the use
of a large number of versions, but offers limited guarantees in case of a
failure.

In order for our approach to be successful, the external behaviour of the
versions that are run in parallel has to be similar enough to allow us to
synchronise their execution. In practice, it means that the external behaviour
of all the versions has to be the same. The empirical study presented in this
thesis shows that changes to the external behaviour of an application are often
minimal, so our approach should work well with versions that are not too
distant from one another (\ie several revisions apart).

Similarly, our system relies on the assumption that versions re-converge to the
same behaviour after a divergence.  Therefore, we believe multi-version
execution would be a good fit for applications that perform a series of mostly
independent requests, such as network servers. These applications are usually
structured around a main dispatch loop, which provides a useful re-convergence
point. Our approach is also suitable to local code changes, which have small
propagation distances, thus ensuring that the different versions will
eventually re-converge to the same behaviour.

Our approach is targeted toward scenarios where the availability and
reliability of a software system is more important than strict correctness,
high performance and low energy consumption.  In terms of correctness
guarantees, it is similar to previous approaches such as failure oblivious
computing~\cite{fo} which may sacrifice strict correctness for increased
availability and reliability. We aim to alleviate potential problems by
regularly checking if all versions exhibit the same external behaviour and by
reverting to running a single version when a non-resolvable divergence is
detected making this a best-effort approach.

%Our current focus is on multi-core processors, but we believe this solution
%could be adapted to work on other parallel platforms as well.  Furthermore,
%this update mechanism could be extended to balance conflicting requirements
%such as performance, reliability and energy consumption.

There are three main challenges that we need to address to make this approach
viable. First, we need to implement mechanisms for effectively coordinating the
parallel execution of multiple program versions.  Second, when executions
diverge, we need to select the output of the more reliable one.  Finally, we
would like to be able to scale the number of program versions run in parallel
in order to balance conflicting requirements such as performance, reliability,
and energy consumption.

%We aim to tackle this problem using a simple but effective approach based on
%application-level virtualization.  Whenever a new program update becomes
%available, instead of upgrading the software to the newest version, we {\it
%run the new version in parallel with the old}.  As new versions arrive, we
%continue to execute them in parallel with the existing ones, until all
%available resources have been exhausted, or a user-specified threshold has
%been reached.  At that point, we can either discard the very oldest versions,
%or we can use more sophisticated replacement strategies.  This approach can
%be extended to work with a large number of versions run in parallel, and can
%be applied to several different platforms, ranging from multicore processors
%to large-scale data centers.  

%% is never worse than simply using one version of the software: if a
%% non-crashing divergence is detected, \mx can simply continue execution
%% with a single program version.



\section{Organisation}
\label{overview:organisation}

The rest of the thesis is organized as follows:

\begin{chapterdescription}
\item[Chapter~\ref{chap:multi-version}] gives an overview of the $N$-version,
  multi-variant and multi-version execution techniques and sets the background
  for the rest of the thesis. The chapter also presents several real-world
  scenarios which are used throughout the thesis as a part of the evaluation.

\item[Chapter~\ref{chap:evolution}] presents a study of software evolution in
  several real-world open-source systems and acts as a motivation for many of
  the design choices made in the systems presented in this thesis.

\item[Chapter~\ref{chap:efficient-execution}] describes \varan, an efficient
  $N$-version monitor designed to run a: large number of versions in parallel.
  The chapter presents the high-level architecture as well as details of the
  prototype implementation, and includes a thorough evaluation including a
  comparison with previous systems.

\item[Chapter~\ref{chap:safe-updates}] presents \mx, a multi-version execution
  runtime focused on fail-recovery from crashes caused by bugs introduced in
  software updates. The chapter gives a detailed overview of the fail-recovery
  mechanism and explains how \mx survives bugs in several existing open-source
  applications.

\item[Chapter~\ref{chap:applications}] introduces other possible applications
  of multi-version and $N$-version execution techniques, such as live
  sanitization, record and replay and security honeypots.

\item[Chapter~\ref{chap:related}] discusses work related to $N$-version and
  multi-version execution both in the reliability and the security context.
\end{chapterdescription}

Finally, the thesis concludes with Chapter~\ref{chap:conclusion}, which
summarizes the contributions, discusses the possible future work and raises
additional research questions.

%The rest of this paper is structured as follows.
%Section~\ref{sec:overview} gives a high-level overview of our
%approach, while Section~\ref{sec:prototype} presents our prototype
%implementation in detail.  Then, Section~\ref{sec:evaluation}
%evaluates our prototype on a set of micro- and macro-benchmarks,
%Section~\ref{sec:applications} shows the applicability to different
%application scenarios, and Section~\ref{sec:discussion} discusses the
%main implications of our design.  Finally, Section~\ref{sec:related}
%presents related work and Section~\ref{sec:conclusion} concludes.

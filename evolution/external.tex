\section{External behaviour evolution}
\label{evolution:external}

%% \begin{figure}[t]
%%   \centering
%%   \includegraphics[width=\columnwidth]{safe-updates/graphs/diff}
%%   \caption{Source code differences across 164 versions of {\footnotesize \texttt{lighttpd}}.}
%%   \label{fig:differences}
%% \end{figure}

Another major part of our study is the analysis of the \emph{external
behaviour} of the application. In the context of Linux applications, we define
the external behaviour of an application as its sequence of system calls, which
are the primary mechanism for an application to change the state of its
environment.  Note that the key insight here is that we are only concerned with
externally observable behaviour, and are oblivious to the way the external
behaviour is generated.

Our assumption is that the changes to the external behaviour of an application
are relatively small. To verify this assumption, we used our infrastructure to
trace the external behaviour of two applications from our set across the
selected range of revisions: \lighttpd and \lighttpdtwo. This choice is
pragmatic, other applications either ran excessively long when the system call
tracing was enabled, their traces contained a lot of non-determinism which made
the comparison impossible (this would not be a problem in the multi-version
execution context as shown in~\S\ref{sec:mxm} and~\S\ref{sec:threading}), or
their test suite required extensive modifications to allow for tracing.

% \paragraph{Selection of revisions} The revision intervals for both applications
% were selected to contain the two known bugs---in particular, for \vim the bug
% introduced in the version 7.1.127 affecting the path completion feature, and for
% \lighttpd the bug introduced in the revision 2436 affecting the ETag handling. The
% version interval we used in case of \vim was 7.1--7.1.330. These versions were
% developed and released in slightly-over one year time-frame (from 12 May 2007
% to 21 June 2008). In case of \lighttpd, we used the interval spanning revisions
% 2379--2635 (overlapping versions 1.4.21--1.4.24), which were developed in
% approximately ten months time frame (from 3 February to 11 October 2009).
% During this period, code patches in \lighttpd varied between
% \lighttpdMinPatch and \lighttpdMaxPatch~LOC, with a median value of 
% \lighttpdMedPatch~LOC.

%% This suite consists of 19 tests files, each of them consisting of
%% number of individual tests. 
%% For the purpose of our experiment, we have selected a subset of 7 core
%% test files excluding those targeting standalone modules.

%\paragraph{Setup}
The infrastructure used for system call tracing is conceptually similar to the
one used for collecting coverage information, but there are some notable
differences. We used \lxc containers directly rather than through \docker to
allow the use of the Linux \perf
tool,\footnote{\url{https://perf.wiki.kernel.org/index.php/Main_Page}} which
was used for recording the application's execution and obtaining the system
call traces. These containers were controlled by an Ansible
framework\footnote{\url{http://www.ansible.com/}} using a custom \lxc
connector.

To collect the external behaviour information, we built each revision and
recorded its execution while running the standard test suite. The use of \perf
has some advantages compared to other similar tools such as
\strace,\footnote{\url{http://sourceforge.net/projects/strace/}} most notably
the performance overhead, which is significantly lower as \perf is implemented
largely in the kernel while \strace uses the \ptrace interface. This minimises
the potential side-effects of the tracing infrastructure on the test suite
execution.  The downside is the size of the recorded data which can be
considerable, especially for I/O intensive applications such as \redis, as the
\perf tool uses a custom binary format which contains a large number of
additional information compared to the simple textual output of \strace.

% while running all the tests from the \lighttpd regression suite targeting the
% core functionality (a total of seven tests, but each test contains a large
% number of test cases issuing HTTP requests).
% while running tests from the Vim regression suite. This suite consists
% of 64 tests, which aim to exercise most of the functionality provided by Vim
% text editor. For the purpose of our experiment, we have selected a subset
% of 60 tests excluding those targeting GUI.

% To build the applications, we used the existing build targets and configuration
% options. \vim versions were compiled using the commands:
% \begin{lstlisting}[language=bash,numbers=none]
% ./configure --without-x --disable-gui
% \end{lstlisting}
% and
% \begin{lstlisting}[language=bash,numbers=none]
% make -j 4 CFLAGS="-O2 -D_FORTIFY_SOURCE=1 -fno-strength-reduce -Wall"
% \end{lstlisting}
% The \texttt{-D\_FORTIFY\_SOURCE=1} compiler flag was used to overcome an issue
% introduced by the GCC compiler
% optimization,\footnote{\url{https://groups.google.com/d/topic/vim_dev/HU5aZfCv4iM/discussion}}
% which caused some of the tests from the Vim regression suite to crash.

All tests were executed in an \lxc container based on the Ubuntu 12.04
template.  To eliminate the possible sources of non-determinism, we have
disabled address-space layout randomisation while running the tests. To
account for the remaining non-determinism, we have repeated the tracing five
times for each revision, as we did in the code coverage
study (\S\ref{sec:code-cov}).

System call traces generated by \perf were normalised and
post-processed. We first split the original trace on a per-process basis so
that the trace for each process ended up stored in a separate file.
Moreover, we used the order in which processes were started as a basis for the
naming scheme to allow comparison of the traces across different runs and
versions: the main thread of the first process would be stored in the file
\texttt{000-00.trace}, the first thread spawned by the main thread would be
stored in the file \texttt{000-01.trace}, the main thread of the second
process would use the file \texttt{001-00.trace}, \etc

% We then normalised all differences caused by timing, \eg we collapsed all
% sequences of \textstt{accept+poll} system calls, which represent repeated
% polling operations.  We have also collapsed all \textstt{read+stat+read} and
% \textstt{read+open+close} sequences sometimes used to check for file existence,
% as their occurrence is often dependent on the result of previous system calls
% (\ie if one of the previous calls returned \textstt{EAGAIN} error code). This
% was necessary to eliminate any inherent non-determinism in the traces and to
% allow further comparison.

Trace files were then post-processed by eliminating return values and arguments
of system calls which are known to be non-deterministic, \ie memory addresses,
process identifiers, and timing information. This post-processing step
might reduce the precision of our comparison, but we performed it for two
different reasons:%
\begin{inparaenum}[(1)]
\item many system calls have indirect arguments, accepting addresses of data
  structures residing in the process address space, and these addresses may
  differ across individual executions;
\item the return value of some system calls depends on the current system
  resource usage (\eg number of running processes and threads, amount of
  available memory) which could differ from one run to another.
\end{inparaenum}
However, this sort of non-determinism would not show up in a multi-version
execution context as explained in~\S\ref{sec:mxm}.

We then compared the post-processed traces of consecutive revisions using the
edit distance. To distinguish the differences in the external behaviour
introduced by the changes in the code from differences introduced by changes in
the test suite, we have only compared consecutive revisions which do not modify
tests. To determine the differences between the two revisions, we calculated
the edit distance for every pair of traces between the two revisions. Since we
traced each revision five times, this gives us 25 edit distance values
for every pair of revisions.

The next step was to determine whether the observed differences are caused by
actual changes in the code or by non-determinism. To do so, we calculated the
mean and the variance for each vector of edit distance values; these were used
to calculate the \emph{index of dispersion}, which is defined as a ratio of the
variance to the mean and is often used as a normalized measure of the
dispersion of a probability distribution. Obtaining an index value greater than
one is an indicator of over-dispersion meaning that the variability present in
the given data set is greater than what would be expected based on a given
statistical model. Since the frequencies of each sequence of system calls in
all possible traces are not known a priori, we must choose a frequency
distribution for abnormal sequences. There are several possibilities for
choosing this distribution, the simplest of which is to assume that the
abnormal distribution is uniform~\cite{helman97}. We then interpret the
presence of over-dispersion as ``noise'' due to non-determinism, and as such we
discard that measurement. Otherwise, we use the mean value as the actual value
for the set. Through this process, for each pair of revisions considered, we
obtain a vector of edit distances where individual elements are edit distances
of individual processes.

\subsection{Trace differences}

\begin{figure}[t]
  \begin{center}
    \includegraphics[width=\columnwidth]{evolution/graphs/trace}
    \caption{Co-evolution of executable code and external behaviour. Each
    increment represents a change.}
    \label{fig:coexternal}
  \end{center}
\end{figure}

Our results are shown in Figure~\ref{fig:coexternal}, which correlates the
differences in post-processed system call traces with the source code changes.
Out of 274 \lighttpd revisions considered, 199 introduced no changes in
external behaviour, which accounts for 72.6\% of revisions considered. In case
of \lighttpdtwo, this number is even higher, with 180 out 209 revisions
introducing no changes accounting for 86.1\% of revisions. %Finally, in case of
%\vim, there was \ldots, out of \ldots revisions, which introduced no changes
%accounting for \ldots\%.

These results show that changes in externally observable behaviour are
sporadic. In particular, the revisions which introduced the bugs described in
Section~\ref{multi-version:scenarios} are among the revisions which did not
introduce any external behaviour changes, yet these revisions are responsible
for critical crash bugs.

\subsection{Manual analysis}

\begin{figure}[t]
  \begin{center}
    \includegraphics[width=\columnwidth]{evolution/graphs/analysis}
    \caption{Correlation of differences in post-processed system call traces
      with differences in source code. The named versions are the only ones
      introducing external behaviour changes.}
    \label{fig:coanalysis}
  \end{center}
\end{figure}

To get more data points, we also did a manual analysis on a subset of revisions
for two applications---\lighttpd and \vim.

In the case of \lighttpd, we compared 164 successive revisions, namely revisions in
the range 2379--2635 of branch \texttt{lighttpd-1.4.x}, which were developed
and released over a span of approximately ten months, from January to October
2009. In the case of \vim, we compared 330 successive versions, namely versions
7.1--7.1.330, which were developed and released in approximately 13 months,
from May 2007 to June 2008.

%To understand the amount of code changes in these versions, we 
%computed the number of lines of code (LOC) that have changed from
%one version to the next.  
%% Figure~\ref{fig:differences} summarises these differences.  This graph
%% shows that patches in \lighttpd are relatively small, most of them
%% affecting less than 30 LOC.

%During this period, code patches in \lighttpd varied between
%\lighttpdMinPatch and \lighttpdMaxPatch~LOC, with a median value of 
%\lighttpdMedPatch~LOC.

For the manual analysis, we traced the system calls made by these revisions
using the \strace\footnote{\url{http://sourceforge.net/projects/strace/}} tool,
while running a subset of the tests from the \lighttpd regression suite targeting
the core functionality (a total of seven tests, but each test contains a large
number of test cases issuing HTTP requests).

%% This suite consists of 19 tests files, each of them consisting of
%% number of individual tests. 
%% For the purpose of our experiment, we have selected a subset of 7 core
%% test files excluding those targeting standalone modules.

As in the case of the automated analysis, the recorded system call traces
were normalised and post-processed. We split the traces on a per-process basis,
and normalised all differences caused by timing, \eg we collapsed all sequences
of \lstinline`accept`--\lstinline`poll` and
\lstinline`select`--\lstinline`wait` system calls, which represent repeated
polling operations. We have also eliminated any \lstinline`brk` system calls
from traces, as these calls are usually used for managing heap memory and their
occurrence is often dependent on process scheduling.  Finally, for the manual
analysis, we also eliminated individual system call arguments and return
values. This post-processing step might reduce the precision of our comparison,
but we performed to make the manual analysis actually feasible. This would not
be an issue in the multi-version execution context as the multi-version
execution environment would intercept all sources of non-determinism
(\S\ref{sec:mxm}).

Our results are shown in Figure~\ref{fig:coanalysis}, which correlates the
differences in post-processed system call traces with the source code changes.
The graph shows that changes in externally observable behaviour occur only
sporadically. In fact, 156 out of 163 revisions of \lighttpd and 322
out 329 revisions of \vim (which account for more than 95\% of all the
revisions considered in both cases) introduced no changes in external
behaviour.

%% external behaviour observable via system
%% calls tracing.  In the remaining cases, these changes were \ldots, such
%% as the one introduced in revision 2612 as a result of
%% replacing \textstt{poll} system calls with their
%% \textstt{epoll} counter-parts.

%% as a result of newly implemented
%% support for SELinux resulting in 25 changes over 59 tests, were caused mainly by
%% different ordering of system calls or by splitting individual call into multiple
%% different calls.

%%%%%%%%%%%%%%%%%%%%%%%%%%%%%%%

%In fact, 156 versions (which account for around 95\% of all the
%versions considered) introduced no changes in external behaviour.  In
%particular, the revision which introduced the bug described in
%\sref{sec:example} is one of the versions that introduces no changes, yet this
%revision is responsible for a critical crash bug.

% These results clearly show that our assumption is correct and in majority of
% cases (99.59\%), there were no differences in application behavior, observable
% via system calls tracing.  Remaining changes, such as the one introduced in
% version 7.1.296 as a result of newly implemented support for SELinux resulting
% in 25 changes over 59 tests, were caused mainly by different ordering of system
% calls or by splitting individual call into multiple different calls.

% The differences in system call traces across all 164 revisions are
% summarised in Table~\ref{tab:differences}. These results clearly show
% that our assumption is correct and in the majority of cases (96.76\%),
% the sequences of system calls 

% To better understand the results, we have also looked at the number of
% differences per revision. Results in Table~\ref{tab:perrevision} show that
% differences in system calls are not evenly spaced but they are often located
% in particular versions, such as the aforementioned change in version 7.1.296
% responsible for 3489 differences across all tests.

% \begin{table}
%   \centering
%   \begin{tabular}{r @{\qquad}c}
%     \toprule
%     Version & \#Differences \\
%     \midrule
%     7.1.017 & 182 \\
%     7.1.025 & 2 \\
%     7.1.045 & 16 \\
%     7.1.186 & 69 \\
%     7.1.256 & 3 \\
%     7.1.270 & 2 \\
%     7.1.296 & 3489 \\
%     \bottomrule
%   \end{tabular}
% \end{table}

% \begin{table}
%   \centering
%   \begin{tabular}{r @{\qquad}c}
%     \toprule
%     Revision & \#Differences \\
%     \midrule
%     2436 & 182 \\
%     2466 & 2 \\
%     2524 & 16 \\
%     2550 & 69 \\
%     2578 & 3 \\
%     2606 & 2 \\
%     2612 & 3489 \\
%     \bottomrule
%   \end{tabular}
% \end{table}

% \begin{table}
%   \begin{subtable}[t]{0.5\textwidth}
%     \centering
%     \begin{tabular}{r @{\qquad}c c}
%       \toprule
%       & \multicolumn{2}{c}{Frequency} \\
%       \cmidrule(r){2-3}
%       \#Differences & Versions & Percentage \\
%       \midrule
%       0 & 322 & 97.872\% \\
%       2 & 2 & 0.608\% \\
%       3 & 1 & 0.304\% \\
%       16 & 1 & 0.304\% \\ 
%       69 & 1 & 0.304\% \\
%       182 & 1 & 0.304\% \\
%       3489 & 1 & 0.304\% \\
%       \bottomrule
%     \end{tabular}
%     \caption{330 revisions of \vim.}
%   \end{subtable}
%   ~
%   \begin{subtable}[t]{0.5\textwidth}
%     \centering
%     \begin{tabular}{r @{\qquad}c c}
%       \toprule
%       & \multicolumn{2}{c}{Frequency} \\
%       \cmidrule(r){2-3}
%       \#Differences & Versions & Percentage \\
%       \midrule
%       0 & 156 & 95.706\% \\
%       1 & 1 & 0.613\% \\
%       7 & 1 & 0.613\% \\
%       10 & 1 & 0.613\% \\ 
%       14 & 1 & 0.613\% \\
%       24 & 1 & 0.613\% \\
%       42 & 1 & 0.613\% \\
%       273 & 1 & 0.613\% \\
%       \bottomrule
%     \end{tabular}
%     \caption{164 revisions of \lighttpd.}
%   \end{subtable}
%   \caption{Differences in post-processed system call traces aggregated per revision.}
%   \label{tab:perrevision}
% \end{table}

% Differences in system call traces across all 330 versions are summarized in
% Table~\ref{tab:differences}. These results clearly show that our assumption is
% correct and in majority of cases (99.59\%), there were no differences in
% application behavior, observable via system calls tracing.  Remaining changes,
% such as the one introduced in version 7.1.296 as a result of newly implemented
% support for SELinux resulting in 25 changes over 59 tests, were caused mainly by
% different ordering of system calls or by splitting individual call into multiple
% different calls.

% \begin{table}
%   \centering
%   \begin{tabular}{r @{\qquad}c c}
%     \toprule
%     & \multicolumn{2}{c}{Frequency} \\
%     \cmidrule(r){2-3}
%     \#Differences & Tests & Percentage \\
%     \midrule
%     0 & 19654 & 99.564\% \\
%     1 & 5 & 0.025\% \\ 
%     2 & 9 & 0.010\% \\
%     3 & 2 & 0.046\% \\
%     4 & 3 & 0.015\% \\
%     6 & 3 & 0.015\% \\
%     10 & 1 & 0.005\% \\
%     16 & 1 & 0.005\% \\
%     25 & 59 & 0.299\% \\
%     33 & 1 & 0.005\% \\
%     156 & 1 & 0.005\% \\
%     2014 & 1 & 0.005\% \\
%     \bottomrule
%   \end{tabular}
%   \caption{Differences in post-processed system call traces between 330
%   versions of Vim text editor over the entire Vim's regression suite.}
%   \label{tab:differences}
% \end{table}

% The most important finding is that in version 7.1.127, which introduced the
% bug in path completion functionality described in Section~\ref{sub:vim}, there
% were no observable differences in system call traces when running the complete
% regression suite. Yet this version still introduced critical bug, which
% affected thousands of users. This clearly demonstrates viability and necessity
% of our approach.

% We have also looked at differences in source code across all 330 versions of
% Vim text editor. Histogram which can be seen in the
% Figure~\ref{fig:differences} summarizes these differences.  This graph shows
% that in the majority of cases, differences across versions in terms of LOC are
% very small, with patches consisting of 5-10 changed LOC being the most common.

% \begin{figure}[!ht]
%   \begin{center}
%     \includegraphics[width=\textwidth]{differences}
%     \caption{Source code differences between 330 versions of Vim text editor.}
%     \label{fig:differences}
%   \end{center}
% \end{figure}

% For comparison, we have also correlated differences in post-processed system
% call traces with differences in source code. From the
% Figure~\ref{fig:correlation} and measurements described herein before, it is
% obvious that while number of changes in code-base grow continuously between
% each version, changes in externally observable behavior (\ie system calls)
% occur only sporadically. Again, this result supports our original assumption
% and further motivates our approach.

% \begin{figure}[!ht]
%   \begin{center}
%     \includegraphics[width=\textwidth]{correlation}
%     \caption{Correlation of differences in post-processed system calls traces
%     with differences in source code between 330 versions of Vim text editor.}
%     \label{fig:correlation}
%   \end{center}
% \end{figure}

% To verify this assumption, we compared 164 successive revisions of the
% \lighttpd web server, namely revisions in the range 2379--2635 of
% branch \textstt{lighttpd-1.4.x}, which were developed and released
% over a span of approximately ten months, from January to October 2009.
% %19 January 2009 to 11 October 2009).
% To understand the amount of code changes in these versions, we 
% computed the number of lines of code (LOC) that have changed from
% one version to the next.  
% %% Figure~\ref{fig:differences} summarises these differences.  This graph
% %% shows that patches in \lighttpd are relatively small, most of them
% %% affecting less than 30 LOC.
% During this period, code patches in \lighttpd varied between
% \lighttpdMinPatch and \lighttpdMaxPatch~LOC, with a median value of 
% \lighttpdMedPatch~LOC.

%% This suite consists of 19 tests files, each of them consisting of
%% number of individual tests. 
%% For the purpose of our experiment, we have selected a subset of 7 core
%% test files excluding those targeting standalone modules.

%% The differences in system call traces across all 164 revisions are
%% summarised in Table~\ref{tab:differences}. These results clearly show
%% that our assumption is correct and in the majority of cases (96.76\%),
%% the sequences of system calls 

%% external behaviour observable via system
%% calls tracing.  In the remaining cases, these changes were \ldots, such
%% as the one introduced in revision 2612 as a result of
%% replacing \textstt{poll} system calls with their
%% \textstt{epoll} counter-parts.

%% as a result of newly implemented
%% support for SELinux resulting in 25 changes over 59 tests, were caused mainly by
%% different ordering of system calls or by splitting individual call into multiple
%% different calls.

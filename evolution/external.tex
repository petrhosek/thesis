%% \begin{figure}[t]
%%   \centering
%%   \includegraphics[width=\columnwidth]{safe-updates/graphs/diff}
%%   \caption{Source code differences across 164 versions of {\footnotesize \texttt{lighttpd}}.}
%%   \label{fig:differences}
%% \end{figure}


Our approach is based largely on the assumption that during software evolution,
the changes to the external behaviour of an application are relatively small.
In the context of Linux applications, the external behaviour of an application
consists of its sequence of system calls, which are the primary mechanism for
an application to change the state of its environment.  Note that the key
insight here is that we are only concerned with \emph{externally observable
behaviour}, and are oblivious to the way the external behaviour is generated.

To verify this assumption, we used the \covrig infrastructure to trace the
external behavior of two applications across different revisions and compare
these traces:

\begin{enumerate}
\item[\lighttpd\footnote{\url{http://www.lighttpd.net/}}] is a popular and
  lightweight web server used by several high-traffic websites. 

\item[\vim\footnote{\url{http://www.vim.org/}}] is arguably one of the most
  popular text editors.
\end{enumerate}

\paragraph{Selection of revisions} The revision intervals for both applications
were selected to contain the two known bugs---in particular, for \vim the bug
introduced in the version 7.1.127 affecting the path completion feature, and for
\lighttpd the bug introduced in the revision 2436 affecting the ETag handling. The
version interval we used in case of \vim was 7.1--7.1.330. These versions were
developed and released in slightly-over one year time-frame (from 12 May 2007
to 21 June 2008). In case of \lighttpd, we used the interval spanning revisions
2379--2635 (overlapping versions 1.4.21--1.4.24), which were developed in
approximately ten months time frame (from 3 February to 11 October 2009).

%% Figure~\ref{fig:differences} summarises these differences.  This graph
%% shows that patches in \lighttpd are relatively small, most of them
%% affecting less than 30 LOC.
During this period, code patches in \lighttpd varied between
\lighttpdMinPatch and \lighttpdMaxPatch~LOC, with a median value of 
\lighttpdMedPatch~LOC.

%% This suite consists of 19 tests files, each of them consisting of
%% number of individual tests. 
%% For the purpose of our experiment, we have selected a subset of 7 core
%% test files excluding those targeting standalone modules.

\paragraph{Setup} To compare the external behaviour of each version, we traced
the system calls made by these versions using the
\textstt{strace}\footnote{\url{http://sourceforge.net/projects/strace/}} tool.

%while running all the tests from the \lighttpd regression suite targeting the
%core functionality (a total of seven tests, but each test contains a large
%number of test cases issuing HTTP requests).
%while running tests from the Vim regression suite. This suite consists
%of 64 tests, which aim to exercise most of the functionality provided by Vim
%text editor. For the purpose of our experiment, we have selected a subset
%of 60 tests excluding those targeting GUI.

All tests were executed on a machine running a 64-bit version of Ubuntu 10.04.
To build the applications, we used the existing build targets and configuration
options. \vim versions were compiled using the commands:
\begin{lstlisting}[language=bash,numbers=none]
./configure --without-x --disable-gui
\end{lstlisting}
and
\begin{lstlisting}[language=bash,numbers=none]
make -j 4 CFLAGS="-O2 -D_FORTIFY_SOURCE=1 -fno-strength-reduce -Wall"
\end{lstlisting}
The \texttt{-D\_FORTIFY\_SOURCE=1} compiler flag was used to overcome an issue
introduced by the GCC compiler
optimization,\footnote{\url{https://groups.google.com/d/topic/vim_dev/HU5aZfCv4iM/discussion}}
which caused some of the tests from the Vim regression suite to crash.

To eliminate possible sources of non-determinism, we have disabled the
address-space randomisation while running the tests. To further account for any
non-deterministic behaviour, we have repeated the tracing three times for each
test case and compared the traces across runs.

The system call traces were further normalised and post-processed.  We first
split the original trace on a per-process basis so that the trace for each
process was stored in an individual file.  Moreover, we used the order in which
processes were started as a basis for the naming scheme to allow comparison of
the traces across different runs and versions. We then normalised all
differences caused by timing, \eg we collapsed all sequences of
\textstt{accept+poll} system calls, which represent repeated polling
operations.  We have also collapsed all \textstt{read+stat+read} and
\textstt{read+open+close} sequences sometimes used to check for file existence,
as their occurrence is often dependent on the result of previous system calls
(\ie if one of the previous calls returned \textstt{EAGAIN} error code). This
was necessary to eliminate any inherent non-determinism in the traces and to
allow further comparison.

Trace files were then post-processed by eliminating individual system call
arguments and return values.  This post-processing step might reduce the
precision of our comparison, but we performed it for two different reasons:%
\begin{inparaenum}[(1)]
\item many system calls have indirect arguments, accepting address of data
  structure residing in the process address space, and these addresses may
  differ across versions;
\item the return value of many system calls depend on the current system
  resource usage (\eg number of running processes and threads, amount of
  available memory) which could differ from one run to another.
\end{inparaenum}
Finally, we concatenated the trace files for all tests in a each revision,
ending up with one trace for each run of the test suite. We then compared the
traces of consecutive revisions using the edit distance.

\subsection{Trace Differences}

\begin{figure}[t]
  \begin{center}
    \includegraphics[width=\columnwidth]{evolution/graphs/traces}
    \caption{Correlation of differences in post-processed system call
      traces with differences in source code. The named versions
      are the only ones introducing external behaviour changes.}
    \label{fig:correlation}
  \end{center}
\end{figure}

Our results are shown in Figure~\ref{fig:correlation}, which correlates the
differences in post-processed system call traces with the source code changes.
The graph shows that changes in externally observable behaviour occur only
sporadically.  In fact, 156 out of 163 revisions of \lighttpd and 322 out 329
revisions of \vim (which account for more than 95\% of all the revisions
considered in both cases) introduced \textit{no changes} in external behaviour.
In particular, the revisions which introduced bugs described in
\sref{multi-version:scenarios} are among revisions which did not introduce any
changes, yet these revisions are responsible for critical crash bugs.

To better understand the results, we have also looked at the number of
differences per version. Results in Table~\ref{tab:perversion} show that
differences in system calls are not evenly spaced but they are often located
in particular versions, such as the aforementioned change in version 7.1.296
responsible for 3489 differences across all tests.

\begin{table}
  \centering
  \begin{tabular}{r @{\qquad}c}
    \toprule
    Version & \#Differences \\
    \midrule
    7.1.017 & 182 \\
    7.1.025 & 2 \\
    7.1.045 & 16 \\
    7.1.186 & 69 \\
    7.1.256 & 3 \\
    7.1.270 & 2 \\
    7.1.296 & 3489 \\
    \bottomrule
  \end{tabular}
\end{table}

\begin{table}
  \centering
  \begin{tabular}{r @{\qquad}c}
    \toprule
    Revision & \#Differences \\
    \midrule
    2436 & 182 \\
    2466 & 2 \\
    2524 & 16 \\
    2550 & 69 \\
    2578 & 3 \\
    2606 & 2 \\
    2612 & 3489 \\
    \bottomrule
  \end{tabular}
\end{table}

\begin{table}
  \centering
  \begin{tabular}{r @{\qquad}c c}
    \toprule
    & \multicolumn{2}{c}{Frequency} \\
    \cmidrule(r){2-3}
    \#Differences & Versions & Percentage \\
    \midrule
    0 & 322 & 97.872\% \\
    2 & 2 & 0.608\% \\
    3 & 1 & 0.304\% \\
    16 & 1 & 0.304\% \\ 
    69 & 1 & 0.304\% \\
    182 & 1 & 0.304\% \\
    3489 & 1 & 0.304\% \\
    \bottomrule
  \end{tabular}
  \caption{Differences in post-processed system call traces between 330
  versions of Vim text editor aggregated per version on Vim's regression
  suite.}
  \label{tab:perversion}
\end{table}

\begin{table}
  \centering
  \begin{tabular}{r @{\qquad}c c}
    \hline
    \#Differences & Versions & Percentage \\
    \hline
    0 & 156 & 95.706\% \\
    1 & 1 & 0.613\% \\
    7 & 1 & 0.613\% \\
    10 & 1 & 0.613\% \\ 
    14 & 1 & 0.613\% \\
    24 & 1 & 0.613\% \\
    42 & 1 & 0.613\% \\
    273 & 1 & 0.613\% \\
    \hline
  \end{tabular}
  \caption{Differences in post-processed system call traces between 164
  revisions of \lighttpd aggregated per version on the subset of \lighttpd's
  regression suite.}
  \label{tab:perversion}
\end{table}

% Differences in system call traces across all 330 versions are summarized in
% Table~\ref{tab:differences}. These results clearly show that our assumption is
% correct and in majority of cases (99.59\%), there were no differences in
% application behavior, observable via system calls tracing.  Remaining changes,
% such as the one introduced in version 7.1.296 as a result of newly implemented
% support for SELinux resulting in 25 changes over 59 tests, were caused mainly by
% different ordering of system calls or by splitting individual call into multiple
% different calls.

% \begin{table}
%   \centering
%   \begin{tabular}{r @{\qquad}c c}
%     \toprule
%     & \multicolumn{2}{c}{Frequency} \\
%     \cmidrule(r){2-3}
%     \#Differences & Tests & Percentage \\
%     \midrule
%     0 & 19654 & 99.564\% \\
%     1 & 5 & 0.025\% \\ 
%     2 & 9 & 0.010\% \\
%     3 & 2 & 0.046\% \\
%     4 & 3 & 0.015\% \\
%     6 & 3 & 0.015\% \\
%     10 & 1 & 0.005\% \\
%     16 & 1 & 0.005\% \\
%     25 & 59 & 0.299\% \\
%     33 & 1 & 0.005\% \\
%     156 & 1 & 0.005\% \\
%     2014 & 1 & 0.005\% \\
%     \bottomrule
%   \end{tabular}
%   \caption{Differences in post-processed system call traces between 330
%   versions of Vim text editor over the entire Vim's regression suite.}
%   \label{tab:differences}
% \end{table}

The most important finding is that in version 7.1.127, which introduced the
bug in path completion functionality described in Section~\ref{sub:vim}, there
were no observable differences in system call traces when running the complete
regression suite. Yet this version still introduced critical bug, which
affected thousands of users. This clearly demonstrates viability and necessity
of our approach.

We have also looked at differences in source code across all 330 versions of
Vim text editor. Histogram which can be seen in the
Figure~\ref{fig:differences} summarizes these differences.  This graph shows
that in the majority of cases, differences across versions in terms of LOC are
very small, with patches consisting of 5-10 changed LOC being the most common.

% \begin{figure}[!ht]
%   \begin{center}
%     \includegraphics[width=\textwidth]{differences}
%     \caption{Source code differences between 330 versions of Vim text editor.}
%     \label{fig:differences}
%   \end{center}
% \end{figure}

For comparison, we have also correlated differences in post-processed system
call traces with differences in source code. From the
Figure~\ref{fig:correlation} and measurements described herein before, it is
obvious that while number of changes in code-base grow continuously between
each version, changes in externally observable behavior (\ie system calls)
occur only sporadically. Again, this result supports our original assumption
and further motivates our approach.

% \begin{figure}[!ht]
%   \begin{center}
%     \includegraphics[width=\textwidth]{correlation}
%     \caption{Correlation of differences in post-processed system calls traces
%     with differences in source code between 330 versions of Vim text editor.}
%     \label{fig:correlation}
%   \end{center}
% \end{figure}

To verify this assumption, we compared 164 successive revisions of the
\lighttpd web server, namely revisions in the range 2379--2635 of
branch \textstt{lighttpd-1.4.x}, which were developed and released
over a span of approximately ten months, from January to October 2009.
%19 January 2009 to 11 October 2009).
To understand the amount of code changes in these versions, we 
computed the number of lines of code (LOC) that have changed from
one version to the next.  
%% Figure~\ref{fig:differences} summarises these differences.  This graph
%% shows that patches in \lighttpd are relatively small, most of them
%% affecting less than 30 LOC.
During this period, code patches in \lighttpd varied between
\lighttpdMinPatch and \lighttpdMaxPatch~LOC, with a median value of 
\lighttpdMedPatch~LOC.

%% This suite consists of 19 tests files, each of them consisting of
%% number of individual tests. 
%% For the purpose of our experiment, we have selected a subset of 7 core
%% test files excluding those targeting standalone modules.

%% The differences in system call traces across all 164 revisions are
%% summarised in Table~\ref{tab:differences}. These results clearly show
%% that our assumption is correct and in the majority of cases (96.76\%),
%% the sequences of system calls 

%% external behaviour observable via system
%% calls tracing.  In the remaining cases, these changes were \ldots, such
%% as the one introduced in revision 2612 as a result of
%% replacing \textstt{poll} system calls with their
%% \textstt{epoll} counter-parts.

%% as a result of newly implemented
%% support for SELinux resulting in 25 changes over 59 tests, were caused mainly by
%% different ordering of system calls or by splitting individual call into multiple
%% different calls.

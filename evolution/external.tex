%% \begin{figure}[t]
%%   \centering
%%   \includegraphics[width=\columnwidth]{safe-updates/graphs/diff}
%%   \caption{Source code differences across 164 versions of {\footnotesize \texttt{lighttpd}}.}
%%   \label{fig:differences}
%% \end{figure}


Our approach is based largely on the assumption that during software
evolution, the changes to the external behaviour of an application are
relatively small.  In the context of Linux applications, the external
behaviour of an application consists of its sequence of system calls,
which are the primary mechanism for an application to change the state of
its environment.  Note that the key insight here is that we are only
concerned with \textit{externally observable behaviour}, and are
oblivious to the way the external behaviour is generated.  As a trivial
example, given two versions of a routine that outputs the smallest
element of an array, our approach considers them equivalent even if
the first version scans the array from the first to the last element,
while the other scans it in reverse order.

To verify this assumption, we compared 164 successive revisions of the
\lighttpd web server, namely revisions in the range 2379--2635 of
branch \textstt{lighttpd-1.4.x}, which were developed and released
over a span of approximately ten months, from January to October 2009.
%19 January 2009 to 11 October 2009).
To understand the amount of code changes in these versions, we 
computed the number of lines of code (LOC) that have changed from
one version to the next.  
%% Figure~\ref{fig:differences} summarises these differences.  This graph
%% shows that patches in \lighttpd are relatively small, most of them
%% affecting less than 30 LOC.
During this period, code patches in \lighttpd varied between
\lighttpdMinPatch and \lighttpdMaxPatch~LOC, with a median value of 
\lighttpdMedPatch~LOC.\looseness=-1

%% This suite consists of 19 tests files, each of them consisting of
%% number of individual tests. 
%% For the purpose of our experiment, we have selected a subset of 7 core
%% test files excluding those targeting standalone modules.

To compare the external behaviour of each version, we traced the system
calls made by these versions using the
\textstt{strace}\footnote{\url{http://sourceforge.net/projects/strace/}}
tool, while running all the tests from the \lighttpd regression suite
targeting the core functionality (a total of seven tests, but
each test contains a large number of test cases issuing HTTP requests).

To eliminate possible sources of non-determinism, we have disabled
address-space randomisation while running the tests. To further account for any
non-deterministic behaviour, we have repeated the tracing three times for each
test case and compared the resulting traces across runs.  All tests were
executed on a machine running a Linux 2.6.40.6 x86-64 kernel and the GNU C
library 2.14.

The system call traces were further normalised and post-processed.  We
first split the original trace on a per-process basis, and
%% so that the trace of each different process used
%% by \lighttpd was stored in an individual file.
%% Moreover, we used the order in which processes were started as a basis
%% for the naming scheme to allow comparison of the traces for each
%% process across different runs and versions.
%%
normalised all differences caused by timing (which would not
affect \mx's operation), \eg we collapsed all sequences of
\textstt{accept}-\textstt{poll} system calls, which represent repeated
polling operations.
%
%and we eliminated all logging-related system calls
%
We have also collapsed all
\textstt{read}-\textstt{stat}-\textstt{read} and
\textstt{read}-\textstt{open}-\textstt{close} sequences sometimes
used to check for file existence, as occurrence is often dependent
on result of previous system calls (\ie if one of the previous
calls returned \textstt{EAGAIN} error code). This was necessary to
eliminate possible source of non-determinism and to allow further
comparison of traces.

Trace files were then post-processed by eliminating individual system
call arguments and return values.  This post-processing step might
reduce the precision of our comparison, but we performed it
for two different reasons:%
\begin{inparaenum}[(1)]
\item many system calls accept as arguments addresses of data structures
residing in the virtual address space, and these addresses may differ
across versions (but \mx handles this while mediating the effect of
system calls, as described in \S\ref{sec:mxm}).

\item some system calls return information on current system resources
(\eg number of processes and threads, amount of free/used memory)
which would differ from one run to the other.
\end{inparaenum}
%
Finally, 
%% we concatenated the traces of all \lighttpd processes spawned
%% by each version.  In the end, we had one trace for each run of a
%% \lighttpd version on a test case in the regression suite.  
for each test case, we compared the traces of consecutive \lighttpd
versions using the edit distance.

%% The differences in system call traces across all 164 revisions are
%% summarised in Table~\ref{tab:differences}. These results clearly show
%% that our assumption is correct and in the majority of cases (96.76\%),
%% the sequences of system calls 

%% external behaviour observable via system
%% calls tracing.  In the remaining cases, these changes were \ldots, such
%% as the one introduced in revision 2612 as a result of
%% replacing \textstt{poll} system calls with their
%% \textstt{epoll} counter-parts.

%% as a result of newly implemented
%% support for SELinux resulting in 25 changes over 59 tests, were caused mainly by
%% different ordering of system calls or by splitting individual call into multiple
%% different calls.

%% \begin{table}
%%   \centering
%%   \begin{tabular}{r @{\qquad}c c}
%%     \hline
%%     \#Differences & Tests & Percentage \\
%%     \hline
%%     0 & 1104 & 96.757\% \\
%%     1 & 8 & 0.701\% \\ 
%%     2 & 13 & 1.140\% \\
%%     6 & 7 & 0.613\% \\
%%     10 & 1 & 0.088\% \\
%%     12 & 1 & 0.088\% \\
%%     19 & 1 & 0.088\% \\
%%     25 & 1 & 0.088\% \\
%%     29 & 1 & 0.088\% \\
%%     37 & 1 & 0.088\% \\
%%     39 & 1 & 0.088\% \\
%%     47 & 1 & 0.088\% \\
%%     77 & 1 & 0.088\% \\
%%     \end{tabular}
%%   \caption{Differences in post-processed system call traces between 164
%%   revisions of \lighttpd.}% over the subset of \lighttpd's regression suite.}
%%   \label{tab:differences}
%% \end{table}

\begin{figure}[t]
  \begin{center}
    \includegraphics[width=\columnwidth]{evolution/graphs/lighttpd-traces}
    \caption{Correlation of differences in post-processed system call
      traces with differences in source code across 164 revisions of
      \lighttpd.  The seven named revisions
      are the only ones introducing external behaviour changes.}
    \label{fig:correlation}
  \end{center}
\end{figure}


Our results are shown in Figure~\ref{fig:correlation}, which
correlates the differences in post-processed system call traces with
the source code changes.  The graph shows that changes in externally
observable behaviour occur only sporadically.  In fact, 156 versions
(which account for around 95\% of all the versions considered)
introduce \textit{no changes} in external behaviour.  In particular,
the revision which introduced the bug described in \sref{sec:example}
is one of the versions that introduces no changes, yet this revision
is responsible for a critical crash bug.

%% We believe this initial study supports our original assumption, and is
%% encouraging for the viability of our approach. 
%% %% provides some initial evidence in support of the assumption that
%% %% changes to external behaviour are relatively small, which is
%% %% encouraging for the viability of our approach.
%% We next discuss our experience applying \mx to \redis
%% (\S\ref{sec:redis}) and \lighttpd (\S\ref{sec:lighttpd}).



Table~\ref{tab:perversion} aggregates the changes to external behaviour
on a per-version basis.  As shown in this Table, 156 versions (which
account for 95.706\% of all versions considered) introduced no changes
in external behaviour, while the other versions introduced between 1
and 273 differences. 

\begin{table}
  \centering
  \begin{tabular}{r @{\qquad}c c}
    \hline
    \#Differences & Versions & Percentage \\
    \hline
    0 & 156 & 95.706\% \\
    1 & 1 & 0.613\% \\
    7 & 1 & 0.613\% \\
    10 & 1 & 0.613\% \\ 
    14 & 1 & 0.613\% \\
    24 & 1 & 0.613\% \\
    42 & 1 & 0.613\% \\
    273 & 1 & 0.613\% \\
    \hline
  \end{tabular}
  \caption{Differences in post-processed system call traces between 164
  revisions of \lighttpd aggregated per version on the subset of \lighttpd's
  regression suite.}
  \label{tab:perversion}
\end{table}

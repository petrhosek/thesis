\section{Study Setup}
\label{sec:study-setup}

We used the \covrig infrastructure to understand the evolution of
\numSystems popular open-source applications written in C/C++, over a
combined period of \numYears years. The six evaluated applications are:

\begin{enumerate}

\item[\gnu~\binutils\footnote{\url{http://www.gnu.org/software/binutils/}}]
is a set of utilities for inspecting and modifying object files,
libraries and binary programs.  We selected for analysis the twelve
utilities from the \stt{binutils} folder (\stt{addr2line}, \stt{ar},
\stt{cxxfilt}, \stt{elfedit}, \stt{nm}, \stt{objcopy}, \stt{objdump},
\stt{ranlib}, \stt{readelf}, \stt{size}, \stt{strings} and \stt{strip}),
which are standard user-level programs under many UNIX distributions.

\item[\beanstalkd\footnote{\url{http://kr.github.io/beanstalkd/}}]
is a simple and fast work queue originally designed for reducing the latency of
page views in high-volume web applications.

\item[\git\footnote{\url{http://git-scm.com/}}]
is one the most popular distributed version control systems used by the
open-source developer community.

\item[\lighttpd\footnote{\url{http://www.lighttpd.net/}}]
is a lightweight web server optimized for high performance environments.

\item[\lighttpdtwo\footnote{\url{http://redmine.lighttpdtwo.net/projects/lighttpdtwo2/}}]
is the new major version of the \lighttpd web server developed entirely from
scratch by the same team of developers.

\item[\memcached\footnote{\url{http://memcached.org/}}]
is a general-purpose distributed memory caching system used by several popular
sites such as Craigslist, Digg and Twitter.

\item[\redis\footnote{\url{http://redis.io/}}]
is a popular key-value data store used by many well-known services such as
Twitter, GitHub and StackOverflow.

\item[\zeromq\footnote{\url{http://zeromq.org/}}]
is a high-performance asynchronous messaging middleware library used by a
number of organisations such as Los Alamos Labs, NASA and CERN.

%\item {\bf GNU diffutils} is a collection of four widely-used
%programs: \stt{diff}, \stt{sdiff}, \stt{diff3} and \stt{cmp}, part of
%many popular UNIX distributions.

\end{enumerate}

The \numSystems applications are representative for C/C++ open-source
code: GNU \binutils are user-level utilities, \git is a version
control system, \beanstalkd, \lighttpdtwo, \memcached and \redis are server
applications, while \zeromq is a library.  All applications include a
regression test suite.

\begin{table}[t]
\caption{Summary of applications used in our study.
\textit{ELOC} represents the number of executable lines of code and
\textit{TLOC} the number of lines in test files in the last revision
analysed.}
\begin{center}
\begin{tabular}{llrlr}
\toprule
\multicolumn{1}{c}{}     & \multicolumn{2}{c}{\sc Code}& \multicolumn{2}{c}{\sc Tests} \\
\cmidrule(r){2-3}\cmidrule(l){4-5}
\textsc{Application} & \textsc{Language} & \textsc{ELOC} & \textsc{Language} & \textsc{TLOC}          % & \bf Time         
\\ \midrule
\beanstalkd  & C         & \beanstalkdSize & C        & \beanstalkdTsize  % & \beanstalkdTestTime
\\
\binutils    & C         & \binutilsSize  & DejaGnu   & \binutilsTsize    % & \binutilsTestTime 
\\
\git         & C         & \gitSize       & C/shell   & \gitTsize         % & \gitTestTime 
\\
\lighttpd    & C         & \lighttpdSize  & Perl    & \lighttpdTsize    % & \lighttpdtwoTestTime 
\\
\lighttpdtwo    & C         & \lighttpdtwoSize  & Python    & \lighttpdtwoTsize    % & \lighttpdtwoTestTime 
\\
\memcached   & C         & \memcachedSize & C/Perl    & \memcachedTsize   % & \memcachedTestTime 
\\
\redis       & C         & \redisSize     & Tcl       & \redisTsize       % & \redisTestTime    
\\
\zeromq      & C++       & \zeromqSize    & C++       & \zeromqTsize      % & \zeromqTestTime   
\\ \bottomrule
\end{tabular}
\end{center}
\label{tbl:systems}
\end{table}

\paragraph{Basic characteristics} Table~\ref{tbl:systems} shows some basic
characteristics of these systems: the language in which the code and tests are
written, the number of executable lines of code (ELOC) and the number of lines
of test code (TLOC) in the last revision analysed. To accurately measure the
number of ELOC, we leveraged the information stored by the compiler in
\texttt{gcov} graph files, while to measure the number of TLOC we did a simple
line count of the test files (using \texttt{cloc}, or \texttt{wc~-l} when
\texttt{cloc} cannot detect the file types).

The code size for these applications varies from only \memcachedSize
ELOC for \memcached to \gitSize ELOC for \git.  The test code is written in
a variety of languages and ranges from \lighttpdtwoTsize lines of Python
code for \lighttpdtwo to \gitTsize lines of C and shell code for \git.
The test code is 36\% larger than the application code in the case
of \git, approximately as large as the application code for
\memcached, around 40\% of the application code for \redis and \zeromq,
and only around 10\% and 19\% of the application code for \lighttpdtwo and
\binutils respectively.  Running the test suite on the last version 
takes only a few seconds for \binutils, \lighttpdtwo, and \zeromq,
\memcachedTestTime seconds for \memcached, \redisTestTime seconds for 
\redis, and 30 minutes for \git, using a four-core Intel Xeon 
E3-1280 machine with 16 GB of RAM.

The version control system used by all these applications is \git.  Four
of these projects---\git, \memcached, \redis, and \zeromq ---are hosted
on the \github\footnote{\url{https://github.com/}} online project site.
The other two---\binutils and \lighttpdtwo---use their own \git hosting.


\paragraph{Selection of revisions} Our goal was to select a comparable number
of revisions across applications. The methodology was to start from the current
version at the day of our experiments, and select an equal number of previous
revisions for all systems. We only counted revisions which modify executable
code, tests or both because this is what our analyses look at. We decided to
select 250 such revisions from each system because some systems had non-trivial
dependency issues further back than this, which prevented us from properly
compiling or running them.  We still had to install the correct dependencies
where appropriate, \eg downgrade \stt{libev} for older versions of \lighttpdtwo
and \stt{libevent} for \memcached.

Note that not all revisions compile, either due to development errors
%(an example of this would be someone forgetting to add a file) 
or portability issues (\eg system header files differing across OS
distributions).
%% Note that we distinguish between this kind of permanent errors (which
%% disallow us to compile all versions earlier than some revision) and
%% more transient compilation errors that affect only some program
%% versions.  
Redis has the largest number of such transient compilation
errors---\redisTransientCompErrs.  The prevailing reasons are
missing \stt{\#include} directives, \eg \stt{unistd.h} for
the \stt{sleep} function, and compiler warnings subsequently treated as errors.
The missing \stt{\#include} directives most likely slipped past the
developers because on some systems other \stt{libc} headers cause the
missing headers to be indirectly included. The compiler warnings were
generated because newer compiler versions, such as the one that we used,
are more pedantic.
Other reasons include forgotten files and even missing semicolons.

We decided to fix the errors which had likely not been seen at the
time a particular revision was created, for example by adding the
compile flag \stt{-Wno-error} in \binutils so that warnings do not
terminate the build process. In all situations when we could not
compile a revision, we rolled over the changes to the next revisions
until we found one where compilation was successful.  Revisions which
do not successfully compile are not counted towards the 250 limit.

Another important decision concerns the granularity of the revisions
being considered.  Modern decentralised software repositories based on
version control systems such as \git do not have a linear structure
and the development history is a directed acyclic graph rather than a
simple chain.  Different development styles generate different
development histories; for example, \git, \redis and \zeromq exhibit a
large amount of branching and merging while the other three systems
have a mostly linear history.  Our decision was to focus on the main branch,
and treat each merge into it as a single revision. In other words, we
considered each feature branch a single indivisible unit.  Our
motivation for this decision was twofold: first, development branches
are often spawned by individual developers in order to work on a
certain issue and are often ``private'' until they are merged into the
main branch.  As a result, sub-revisions in such branches are often
unusable or even non-compilable, reflecting work-in-progress.  Second,
the main branch is generally the one tracked by most users, therefore
analysing revisions at this level is a good match in terms of
understanding what problems are seen in the field.  This being said,
there are certainly development styles and/or research questions that
would require tracking additional branches; however, we believe that
for our benchmarks and research questions this level of granularity
provides meaningful answers.

% On a secondary note, we remark that an additional complication with
% this approach is that version control systems do not associate a
% branch name to each revision, so some detective work might be required
% to follow the main development branch.  However, since
% the projects exhibiting a branching structure are hosted on \github, an implicit central
% integrator exists (the project owner) and we considered their history
% to be the official one, essentially always following the first parent
% in a merge.


\begin{table}[t]
\centering
\caption{Revisions used in our study.
  {\em OK}:~code compiles and tests complete successfully,
  {\em TF}:~some tests fail,
  {\em TO}:~tests time out,
  {\em CF}:~compilation fails,
  {\em Time}:~the number of months analysed.}
\begin{tabular}{lrrrrr}
\toprule
\multicolumn{1}{c}{}          &       \multicolumn{3}{c}{\sc OK+TF+TO=250}                 &            \multicolumn{2}{c}{}                   \\
\cmidrule{2-4}
\textsc{Application} & \textsc{OK} & \textsc{TF} & \textsc{TO} & \textsc{CF} & \textsc{Time}           \\
\midrule
\beanstalkd  &  \beanstalkdOK & \beanstalkdTransientTestErrs & \beanstalkdTransientTestTimeouts & \beanstalkdTransientCompErrs  &  {\beanstalkdTimespan}mo \\
\binutils    &  \binutilsOK   & \binutilsTransientTestErrs  & \binutilsTransientTestTimeouts  & \binutilsTransientCompErrs  &  {\binutilsTimespan}mo \\
\git         &  \gitOK        & \gitTransientTestErrs       & \gitTransientTestTimeouts       & \gitTransientCompErrs       &  {\gitTimespan}mo  \\
\lighttpd    &  \lighttpdOK   & \lighttpdTransientTestErrs  & \lighttpdTransientTestTimeouts  & \lighttpdTransientCompErrs  &  {\lighttpdTimespan}mo  \\
\lighttpdtwo    &  \lighttpdtwoOK   & \lighttpdtwoTransientTestErrs  & \lighttpdtwoTransientTestTimeouts  & \lighttpdtwoTransientCompErrs  &  {\lighttpdtwoTimespan}mo  \\
\memcached   &  \memcachedOK  & \memcachedTransientTestErrs & \memcachedTransientTestTimeouts & \memcachedTransientCompErrs &  {\memcachedTimespan}mo \\
\redis       &  \redisOK      & \redisTransientTestErrs     & \redisTransientTestTimeouts     & \redisTransientCompErrs     &  {\redisTimespan}mo     \\
\zeromq      &  \zeromqOK     & \zeromqTransientTestErrs    & \zeromqTransientTestTimeouts    & \zeromqTransientCompErrs    &  {\zeromqTimespan}mo    \\
\bottomrule
\end{tabular}
\label{tbl:revisions}
\end{table}


Table~\ref{tbl:revisions} summarises the revisions that we selected:
they are grouped into those that compile and pass all the tests
(\textit{OK}), compile but fail some tests (\textit{TF}),
and compile but time out while running the test suite
(\textit{TO}).
The time limit that we enforced was empirically selected for
each system such that it is large enough to allow a correct revision
to complete all tests. As shown in the table, timeouts were a rare
occurrence, with at most one occurrence per application.

Table~\ref{tbl:revisions} also shows the development time span
considered, which ranges from only 5-6 months for \git and \redis,
which had a fast-paced development during this period, to almost 4
years for \memcached. The age of the projects at the first version
that we analysed ranges from a little over 2 years for \lighttpdtwo
(version 2), to 11 years for \binutils.
%6 years memcached, 4 years redis, 2yr 9mo zeromq, 8 years git

\paragraph{Setup} All the programs analysed were compiled to record coverage
information. In addition, we disabled compiler optimisations, which generally
interact poorly with coverage measurements. For this we used existing build
targets and configuration options if available, otherwise we configured the
application with the flags \lstinline`CFLAGS='-O0 -coverage'` and
\lstinline`LDFLAGS=-coverage`. All code from the system headers, \ie
\stt{/usr/include/} was excluded from the results.

Each revision was run in a virtualised environment based on 64-bit version of
Ubuntu 12.10 (12.04.3 for \git) running inside an \lxc container.  To take
advantage of the inherent parallelism of this approach, the containers were
spawned in one of 28 long-running \xen VMs, each with a 4~Ghz CPU, 6~GB of RAM,
and 20~GB of storage, running a 64-bit version of Ubuntu 12.04.3.

The following sections present the main findings of our analysis.  They first
reiterate and then examine in detail our target research questions (RQs).

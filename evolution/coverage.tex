\section{Code and test evolution}
\label{sec:code-test-evol}

\begin{figure}[t]
%\input{evolution/graphs/eloc.tex}
\includegraphics[width=\textwidth]{evolution/graphs/eloc}
\caption{Evolution of executable lines of code.}
\label{fig:codebase-evol}
\end{figure}

% Do executable and test code evolve in sync?

Figure~\ref{fig:codebase-evol} shows the evolution of each system in
terms of ELOC.  As discussed above, we measured the number of ELOC in
each revision by using the information stored in \gcov graph files.
This eliminates all lines which were not compiled, such as those
targeting architectures different from our machine.  One of the main
reasons for which we have decided to measure ELOC rather than other
similar metrics is that they can be easily related to the dynamic
metrics, such as patch coverage, presented in
Sections~\ref{sec:code-cov}.

%% Number of lines added or modified in the revision. These were split
%% into lines executed, not executed and not executable,
%% using \stt{gcov} data as above.

As evident from this figure, all \numSystems systems grow over time,
with periods of intense development that increase the ELOC
significantly, alternating with periods of code tuning and testing,
where the code size increases at a slower pace.  It is interesting to
note that there are also several revisions where the number of ELOC
decreases (\eg in \zeromq): upon manual inspection, we noticed that
they relate to refactorings such as using macros or removing duplicate
code.

%% An interesting behavior occurs in \lighttpdtwo and \memcached at the
%% beginning of the period analysed, with higher ELOC churn than during
%% the rest of their evolution.  We hypothesize that as a project reaches
%% maturity, churn disappears leaving place to a smoother evolution.

The total number of ELOC added or modified varies
between \redisPatchTotal for \redis and \lighttpdtwoPatchTotal
for \lighttpdtwo, while the end-to-end difference in ELOC varies
between \beanstalkdDeltaSize for \beanstalkd and \lighttpdtwoDeltaSize
for \lighttpdtwo.

\begin{figure}[t]
\centering
\includegraphics[width=\textwidth]{evolution/graphs/tloc}
\caption{Evolution of textual lines of test code.}
\label{fig:tloc-evol}
\end{figure}


\begin{figure}[t]
\centering
\includegraphics[width=\textwidth]{evolution/graphs/eloctloc}
\caption{Co-evolution of executable and test code. Each increment represents a change.}
\label{fig:coeloctloc}
\end{figure}


Figure~\ref{fig:tloc-evol} presents the evolution of the size of the test suite
in each system, measured in textual lines of test code (TLOC).  For each
system, we manually identified the files responsible for regression testing and
recorded the number of lines contained in them at each revision. It can be seen
that test evolution is less dynamic than code evolution, developers adding less
test code than regular code. In some cases, developers even remove large parts
of the test code. This is most obvious in case of \beanstalkd whose test suite
has been rewritten from scratch several times over the analysed period.


To better understand the co-evolution of executable and test code, we
merged the above data and plotted in Figure~\ref{fig:coeloctloc}
only whether a revision changes the code or tests: that is,
the \emph{Code} and \emph{Test} values increase by one when a change is
made to the code, respectively to the tests in a revision, and stay constant
otherwise.  As it can be seen, while the \emph{Code} line smoothly
increases over time, the \emph{Test} line frequently stays constant
across revisions, indicating that testing is often a \textit{phased}
activity~\cite{coevol:emse11}, that takes place only at certain times
during development. One exception is \git, where code and
tests evolve more \textit{synchronously}, with a large number of
revisions modifying both code and tests.

\subsection{Patch characteristics}

% How many patches touch only code, only tests, none, or both?

Each revision defines a \textit{patch}, which consists of the totality
of changes introduced by that revision.  Software patches represent
the building blocks of software evolution, and 
%can be roughly seen as the delta between two consecutive software versions.  Software patches
can affect code, regression tests, or infrastructure components such
as build scripts, and play a variety of roles, including bug fixing,
feature addition, and better testing.  
%% In this section, we aim to
%% understand the main characteristics of software patches, how well they
%% are covered by the evolving regression test suite and whether there is
%% any correlation between patch coverage and presence of bugs.

%%\begin{table}[t]
%%\caption{Number of patches of each type: those that touch executable application code but not test code, those that touch both, those that only touch test code, and those that touch neither.}
%%\begin{center}
%%\begin{tabular}{|l|r|r|r|r|}
%%\cline{2-4}
%%\multicolumn{1}{c}{}          &       \multicolumn{3}{|c|}{\bf Sum/app = 250}                 &            \multicolumn{1}{c}{}             \\ \hline
%%\bf App      & \bf C $\land$ $\lnot$T         & \bf C $\land$ T                  & \bf $\lnot$C $\land$ T  & \bf $\lnot$C $\land$ $\lnot$T        \\ \hline
%%\binutils    & \binutilsOnlyExecutableRevs    &  \binutilsTestAndExecutableRevs  & \binutilsOnlyTestRevs   & \binutilsNoTestNoExecutableRevs      \\ \hline
%%\git         & \gitOnlyExecutableRevs         &  \gitTestAndExecutableRevs       & \gitOnlyTestRevs        & \gitNoTestNoExecutableRevs      \\ \hline
%%\lighttpdtwo    & \lighttpdtwoOnlyExecutableRevs    &  \lighttpdtwoTestAndExecutableRevs  & \lighttpdtwoOnlyTestRevs   & \lighttpdtwoNoTestNoExecutableRevs      \\ \hline
%%\memcached   & \memcachedOnlyExecutableRevs   &  \memcachedTestAndExecutableRevs & \memcachedOnlyTestRevs  & \memcachedNoTestNoExecutableRevs     \\ \hline
%%\redis       & \redisOnlyExecutableRevs       &  \redisTestAndExecutableRevs     & \redisOnlyTestRevs      & \redisNoTestNoExecutableRevs         \\ \hline
%%\zeromq      & \zeromqOnlyExecutableRevs      &  \zeromqTestAndExecutableRevs    & \zeromqOnlyTestRevs     & \zeromqNoTestNoExecutableRevs        \\ \hline
%%\end{tabular}
%%\end{center}
%%\label{tbl:patch-types}
%%\end{table}

\begin{figure}[t]
\centering
\includegraphics[width=\textwidth]{evolution/graphs/patchtype}
\caption{Breakdown of patches by type: affecting executable application code but not test code, affecting both, affecting only test code, and neither.}
\label{fig:patch-types}
\end{figure}

Figure~\ref{fig:patch-types} classifies patches into those that modify
executable application code but not the test code (\textit{Code
only}), those that modify both executable application code and test
code (\textit{Code+Test}), and those that modify test code but not
executable application code (\textit{Test only}).  Note that for each
application, these three values sum to 250, since we only selected
revisions which modify executable code and/or tests, as discussed
previously.  Figure~\ref{fig:patch-types} also shows the number of
patches from the time span analysed that modify neither executable
program code nor tests (\textit{Other}).

The first observation is that a substantial amount of time is spent in
maintenance activities that do not involve code nor tests.  For example, during
the period analysed, in addition to the 250 target patches, there were around
120 additional such patches in \binutils, \memcached and \zeromq,
\beanstalkdNoTestNoExecutableRevs in \beanstalkd and \gitNoTestNoExecutableRevs
in \git. Note that some of these patches may modify code that is excluded
during preprocessing on our machine, but most cases involved changes to build
scripts, documentation, and other similar software artefacts.

From the 250 patches selected for each application, the majority only
modify code, with a relatively small number of patches
(\gitOnlyTestRevs in \git, and under 52 for the others) touching only
tests. The number of revisions that modify both code and tests can
offer some indication of the development style used: at one end of the
spectrum there is \redis, with only one such patch, suggesting that
coding and testing are quite separate activities; at the other end
there is \git, with \gitTestAndExecutableRevs such patches, suggesting a
development discipline in which code changes are frequently
accompanied by a test case. % exercising them.


%%NB: hunks and files may appear due to more than executable code, i.e. removed code and non-executable code in code files

% What is the distribution of patch sizes?
% How spread out is each patch through the code?

The size of a patch and the number of locations affected by it can provide
useful guidance for the external behaviour changes---small, localized changes
are likely to introduce small changes in behaviour. The \textit{Lines} column
in Table~\ref{tbl:exec-patch} provides information about the size of the
executable code patches analysed in each system, measured in ELOC. Note that
our measurements ignore changes in the amount of whitespace, \eg whitespace at
the end of the line, characters, because our target programming languages, C
and C++, are insensitive to such modifications.  Most patches are small, with
the median number of ELOC ranging from \redisPatchMedian to \gitPatchMedian.

%% All systems exhibit a large standard deviation of this metric,
%% corresponding to a skewed distribution of executable lines across the
%% patches. In fact, the average patch size is significantly larger than
%% the median for all systems, going up to nine times the median
%% for \lighttpdtwo.
%% We further looked at the total number of changed files, number of
%% changed executable files and number of changed test files. This
%% metrics along with the number of {\em hunks}, discussed next, are a
%% good measure of code churn.

To understand the degree to which patches are spread out through the code, we
also recorded the number of areas in the code---\textit{hunks} in \git
terminology---and the number of files containing executable code which suffered
changes.  More formally, a hunk groups together all the lines added or modified
in a patch which are at a distance smaller than the \textit{context size}.  We
used the default unified diff format with a context size of three lines when
computing the hunks.\footnote{See
\url{http://www.gnu.org/software/diffutils/manual/html_node/} for more
details.}  The \textit{Hunks} column in Table~\ref{tbl:exec-patch} shows that
the median number of hunks varies between \binutilseHunkThreeMedian and
\zeromqeHunkThreeMedian.

Finally, the median number of files modified by a patch is only
\rediseFileMedian for all benchmarks with the exception of \vim and \zeromq,
for which it is \zeromqeFileMedian. The fraction of patches that modify a
single file is, in increasing order, \vimOneeFilePatches for \vim,
\zeromqOneeFilePatches for \zeromq, \gitOneeFilePatches for \git,
\beanstalkdOneeFilePatches in \beanstalkd, \lighttpdtwoOneeFilePatches for
\lighttpdtwo, \memcachedOneeFilePatches for \memcached, \redisOneeFilePatches
for \redis, and \binutilsOneeFilePatches for \binutils.

While these results are based on static analysis only, the fact that most
patches are small and localized suggests that individual patches are likely to
have only a small impact on the external behaviour. This hypothesis is further
confirmed in Section~\ref{evolution:external} which analyses the dynamic
behaviour.

%% \binutils: \binutilsOneELOCPatches, \binutilsOneeHunkPatches, \binutilsOneeFilePatches \\
%% \git: \gitOneELOCPatches, \gitOneeHunkPatches, \gitOneeFilePatches \\
%% \lighttpdtwo\: \lighttpdtwoOneELOCPatches, \lighttpdtwoOneeHunkPatches, \lighttpdtwoOneeFilePatches \\
%% \memcached: \memcachedOneELOCPatches, \memcachedOneeHunkPatches, \memcachedOneeFilePatches \\
%% \redis: \redisOneELOCPatches, \redisOneeHunkPatches, \redisOneeFilePatches \\
%% \zeromq: \zeromqOneELOCPatches, \zeromqOneeHunkPatches, \zeromqOneeFilePatches \\

\section{Code coverage}
\label{sec:code-cov}

% Is test suite execution deterministic?

\begin{table}[t]
\centering
\caption{The median number of executable lines, hunks from executable files, 
and executable files in a patch.  Only data from patches which add or
modify executable code is considered.}
\begin{tabular}{lrrr}
\toprule
\textsc{Application} & \textsc{Lines} & \textsc{Hunks} & \textsc{Files}            \\
\midrule
\beanstalkd  & \beanstalkdPatchMedian  & \beanstalkdeHunkThreeMedian  & \beanstalkdeFileMedian  \\
\binutils    & \binutilsPatchMedian  & \binutilseHunkThreeMedian  & \binutilseFileMedian  \\
\git         & \gitPatchMedian       & \giteHunkThreeMedian       & \giteFileMedian       \\
\lighttpd    & \lighttpdPatchMedian  & \lighttpdeHunkThreeMedian  & \lighttpdeFileMedian  \\
\lighttpdtwo    & \lighttpdtwoPatchMedian  & \lighttpdtwoeHunkThreeMedian  & \lighttpdtwoeFileMedian  \\
\memcached   & \memcachedPatchMedian & \memcachedeHunkThreeMedian & \memcachedeFileMedian \\
\redis       & \redisPatchMedian     & \rediseHunkThreeMedian     & \rediseFileMedian     \\
\vim         & \vimPatchMedian       & \vimeHunkThreeMedian    & \vimeFileMedian    \\
\zeromq      & \zeromqPatchMedian    & \zeromqeHunkThreeMedian    & \zeromqeFileMedian    \\
\bottomrule
\end{tabular}
\label{tbl:exec-patch}
\end{table}

\begin{table}[t]
\centering
\caption{Number of revisions where the test suite non-deterministically 
succeeds/fails, and the maximum, median and average number of lines
which are non-deterministically executed in a revision.}
\begin{tabular}{lrrrr}
\toprule
\multicolumn{1}{c}{}  & \textsc{Nondet.} & \multicolumn{3}{c}{\sc Nondet. ELOC} \\ 
\cmidrule{3-5}
\textsc{Application} & \multicolumn{1}{c}{\sc Result}  & \textsc{Max} & \textsc{Median} & \textsc{Average} \\
\midrule
\beanstalkd  &  \beanstalkdRevsTestsMixedResults  & \beanstalkdNonDetMax  & \beanstalkdNonDetMedian   & \beanstalkdNonDetAverage \\
\binutils    &  \binutilsRevsTestsMixedResults  & \binutilsNonDetMax  & \binutilsNonDetMedian   & \binutilsNonDetAverage \\
\git         &  \gitRevsTestsMixedResults       & \gitNonDetMax       & \gitNonDetMedian        & \gitNonDetAverage \\
\lighttpd    &  \lighttpdRevsTestsMixedResults  & \lighttpdNonDetMax  & \lighttpdNonDetMedian   & \lighttpdNonDetAverage \\
\lighttpdtwo    &  \lighttpdtwoRevsTestsMixedResults  & \lighttpdtwoNonDetMax  & \lighttpdtwoNonDetMedian   & \lighttpdtwoNonDetAverage \\
\memcached   &  \memcachedRevsTestsMixedResults & \memcachedNonDetMax & \memcachedNonDetMedian  & \memcachedNonDetAverage \\
\redis       &  \redisRevsTestsMixedResults     & \redisNonDetMax     & \redisNonDetMedian      & \redisNonDetAverage \\
\vim         &  \vimRevsTestsMixedResults    & \vimNonDetMax    & \vimNonDetMedian     & \vimNonDetAverage \\
\zeromq      &  \zeromqRevsTestsMixedResults    & \zeromqNonDetMax    & \zeromqNonDetMedian     & \zeromqNonDetAverage \\
\bottomrule
\end{tabular}
\label{tbl:nondet}
\end{table}


%% (While we would have preferred to report coverage at the basic block
%% level, for simplicity we opted for the information that is directly
%% available from \gcov). \todo{I don't think basic block coverage is
%% used that much to make it worth mentioning. If line coverage is not
%% enough people go to branch coverage}
As a part of our study focuses on coverage metrics, we first investigate
whether code coverage is deterministic, \ie whether the regression test suite
in a given revision achieves the same coverage every time it is executed. As we
show, non-determinism has implications in the reproducibility of test results
and the fault detection capability of the tests as well as for multi-version
execution.

We measured the overall coverage achieved by the regression test suite using
\gcov.  Interestingly, we found that most programs from our experiments except
\binutils, \lighttpd and \vim are non-deterministic, obtaining slightly
different coverage in each run of the test suite.  Therefore, we first
quantified this non-determinism by running the test suite five times for each
revision and measuring how many revisions obtained mixed results, \ie one run
reported success while another reported failure.  We were surprised to see a
fair number of revisions displaying this behaviour, as listed in
Table~\ref{tbl:nondet} under the column \textit{Nondet Result}.
%We believe that many of these failures were
%invluenced by the Docker environment, as some tests rely on custom timeouts
%which make them more fragile in the environment with slightly different
%performance characteristics.


We further counted for each pair of runs the number of lines whose
coverage status differs. We used a 0/1 metric, \ie we only considered
a difference when one of the five runs never executes a line and
another one executes it. We only did this for revisions in which the
test suite completes successfully to avoid spurious results that would
occur if we compare a run which completed with one that was
prematurely terminated due to a failure.  As shown in Table~\ref{tbl:nondet},
\binutils, \lighttpd and \vim seems to be completely deterministic with respect
to its test suite, while \redis, for example, contains on average
\redisNonDetAverage lines that are non-deterministically executed.

When reporting the overall coverage numbers, we accumulated the
coverage information across all five runs.\footnote{With the exception
of \git, where for convenience we considered a single run, as the
number of lines affected by non-determinism represent less than
$0.3\%$ of the total codebase.} Therefore, the results aim to count a
line as covered if the test suite {\em may} execute it.  The blue
(upper) lines in Figure~\ref{fig:coverage} plot the overall line
coverage for all benchmarks.  It can be seen that coverage level
varies significantly, with \beanstalkd at one end achieving
only \beanstalkdCoverageAverage coverage on average, and \git at the
other achieving
\gitCoverageAverage, while in-between \lighttpdtwo achieves
\lighttpdtwoCoverageAverage, \redis~\redisCoverageAverage,
\zeromq~\zeromqCoverageAverage, and
\memcached~\memcachedCoverageAverage.

We manually investigated the non-determinism and pinpointed three
sources:%
\begin{inparaenum}[(1)]
\item multi-threaded code,
\item ordering of network events, and
\item non-determinism in the test harness.
\end{inparaenum} As an example from the first category, the test from \zeromq
called \stt{test\_shutdown\_stress} creates 100 threads to check the connection
shutdown sequence. In a small percentage of runs, this test was exposing a race
condition.\footnote{\url{https://github.com/zeromq/zeromq4-x/commit/de239f3}}
In the third category, some \redis tests generate and store random integers,
non-deterministically executing the code implementing the internal database data
structures.  The \memcached test \stt{expirations.t} is representative of tests
that make assumptions based on hardcoded wall-clock time values, which cause
failures under certain circumstances. The test timings were previously
adjusted\footnote{\url{https://github.com/memcached/memcached/commit/890e3cd}}
in response to failures under Solaris' \stt{dtrace} and we believe that some of
the failures that we encountered were influenced by the Docker environment.
%\redis and \zeromq explicitly
%use the \stt{pthreads} library, and are thus affected by scheduler
%non-determinism. \lighttpdtwo and \memcached are event-based servers,
%through the use of \stt{libev} and \stt{libevent} respectively; \redis
%uses its own implementation of event-loop. They are all affected by
%non-determinism because they can receive network events
%asynchronously.

These results show that a multi-version execution environment needs to account
for sources of non-determinism to be practical. \varan and \mx intercept and
emulate accesses to environment values which are non-deterministic, such as time
and date (\eg through file \lstinline`/etc/localtime` or system call
\lstinline`gettimeofday()`) or random number generators (\eg through file
\lstinline`/dev/random` or system call \lstinline`getrandom()`). The strict
system call ordering enforced across versions by both \varan and \mx ensures
that all networking events are delivered in the same order in all versions.

For non-determinism related to concurrency, deterministic multi-threading
mechanisms (DMT) have been an active area of research for several years, and
while there is no practical solution which could handle all possible cases with
reasonable overhead, the existing solutions are already usable for many
real-world applications~\cite{coredet:asplos10,dthreads:sosp11}. \varan uses
logical clocks to enforce ordering across threads. While this approach provides
weaker guarantees than those typically provided by other DMT systems, it can
still handle most cases, as described in Section~\ref{sec:threading}).

\begin{figure}[t]
\centering
\includegraphics[width=\textwidth]{evolution/graphs/coverage}
\caption{Evolution of the overall line and branch coverage.}
\label{fig:coverage}
\end{figure}

%The potential drawback of non-determinism is the inability of coverage
%comparison across revisions, lack of reproducibility and consequent
%difficulty in debugging. Developers and researchers relying on test
%suite executions should take non-determinism into account, by either
%quantifying its effects, or by using tools that enforce deterministic
%execution across versions~\cite{mx}, as appropriate.
%Tests with non-deterministic expectations---such as the
%ones presented above---are fragile and should be rewritten. For
%example, tests relying on wall-clock time could be rewritten as
%event-based tests~\cite{imunit}.

% One way of dealing with non-determinism is through multi-version
% execution.  Our tool \mx~\cite{mx} allows both the old and the new revision to
% be run in parallel during the test suite execution which eliminates any
% non-determinism across the two versions allowing for straightforward
% coverage comparison and easier debugging. % We are also working on a new
% % tool which allows executing multiple versions in parallel.

% How does the overall code coverage evolve?  Is it stable over time?

\subsection{Coverage evolution}

One interesting question is whether coverage stays constant over time.
As evident from Figure~\ref{fig:coverage}, for \binutils, \git,
\memcached, and \redis, the overall coverage remains stable over time,
with their coverage changing with less than 2 percentage points within
the analysed period. On the other hand, the coverage of
\lighttpdtwo and \zeromq increase significantly during the time span
considered, with \lighttpdtwo increasing from only
\lighttpdtwoInitialCoverage to 49.37\% (ignoring the last two
versions for which the regression suite fails), and \zeromq increasing
from \zeromqInitialCoverage to \zeromqFinalCoverage. \beanstalkd is the only
application whose coverage decreases over time, from \beanstalkdInitialCoverage
to \beanstalkdFinalCoverage, as its test suite has been reimplemented several
times over the project lifetime (as shown in Figure~\ref{fig:tloc-evol}) with
each implementation achieving different coverage. An interesting observation is
that coverage evolution is not strongly correlated to the co-evolution of
executable and test code. Even when testing is a phased activity, coverage
remains constant because the already existing tests execute part of the newly
added code.

% \binutils: \binutilsInitialCoverage to \binutilsFinalCoverage \\
% \git: \gitInitialCoverage to \gitFinalCoverage \\
% \lighttpdtwo: \lighttpdtwoInitialCoverage to \lighttpdtwoFinalCoverage \\
% \memcached: \memcachedInitialCoverage to \memcachedFinalCoverage \\
% \redis: \redisInitialCoverage to \redisFinalCoverage \\
% \zeromq: \zeromqInitialCoverage to \zeromqFinalCoverage \\

One may notice that a few revisions from \lighttpdtwo, \memcached and \redis
cause a sudden decrease in coverage. This happens because either bugs in the
program or in the test suite prevent the regression tests from
successfully running to completion. In all cases, these bugs are fixed
after just a few revisions.

\begin{figure}[t]
\begin{lstlisting}[label=lst:zeromqassert,basicstyle=\footnotesize\ttfamily,xleftmargin=0pt,numbers=none,caption={Example of an assertion macro used in \zeromq codebase.}]
#define zmq_assert(x) \
  do {\
    if (unlikely (!(x))) {\
      fprintf (stderr, "Assertion failed: %s (%s:%d)\n", #x, \
          __FILE__, __LINE__);\
      zmq::zmq_abort (#x);\
    }\
  } while (false)
\end{lstlisting}
\end{figure}

Figure~\ref{fig:coverage} also shows that branch coverage closely
follows line coverage.  The difference between line and branch
coverage is relatively small, with the exception of \memcached
and \zeromq. The larger difference is due to the frequent use of
certain code patterns which generate multiple branches on a single
line, such as the one shown in Listing~\ref{lst:zeromqassert}, which
comes from the \zeromq codebase.  The \lstinline`zmq_assert` macro is
expanded into a single line resulting in 100\% line coverage, but only
50\% branch coverage when executed in a typical run of the program
(where assertions do not fail).

The fact that line and branch coverage closely follow one another
suggests that in many situations only one of these two metrics might be
needed.  For this reason, in the remaining of the paper, we report
only line coverage.

Finally, we have looked at the impact on coverage of revisions that
only add or modify tests (\textit{Test only} in
Figure~\ref{fig:patch-types}).  An interesting observation is that
many of these revisions bring no improvements to coverage. For
example, in \lighttpdtwo only 26 out of \lighttpdtwoOnlyTestRevs such
revisions improve coverage. The other 26 either do not affect coverage
(18) or decrease it (8).  The revisions which do not affect coverage
can be a sign of test driven development, \ie the tests are added
before the code which they are intended to exercise. The revisions
which decrease coverage are either a symptom of non-determinism---six
of them, with small decreases in coverage---or expose bugs or bigger
changes in the testing infrastructure (the other two).  These two
revisions exhibit a drop in coverage of several thousands lines of
code. In one case, the tests cause \lighttpdtwo to time out, which leads
to a forceful termination and loss of coverage data.  This problem is
promptly fixed in the next revision.  In the other case, the new tests
require a specific (new) module to be built into the server,
terminating the entire test suite prematurely otherwise.

\begin{figure}[t]
\includegraphics[width=\columnwidth]{evolution/graphs/patchcoverage}
\caption{Patch coverage distribution. Each colour represents a range of
coverage values with the bar size indicating the percentage of patches whose
coverage lies in the respective range.}
\label{fig:patch-coverage}
\end{figure}

\subsection{Patch coverage}
\label{sec:pcoverage}
\label{sec:lpcoverage}

% What is the distribution of patch coverage across revisions?

We define {\em patch coverage} as the ratio between the number of
executed lines of code added or modified by a patch and the total
number of executable lines in the patch, measured in the revision that
adds the patch.

Figure~\ref{fig:patch-coverage} shows the distribution of the patch coverage for each
system. Each column corresponds to all patches which affect executable
code in a system, normalised to 100\%. The patches are further grouped into
four categories depending on their coverage.
As it can be observed, the patch coverage distribution is
bi-modal across applications: the majority of the patches
in \git, \memcached and \zeromq achieve over 75\% coverage, while the
majority of the patches in \beanstalkd, \binutils, \lighttpdtwo and \redis achieve
under 25\%.  One interesting aspect is that for all applications,
there are relatively few patches with coverage in the middle ranges:
most of them are either poorly ($\le$25\%) or thoroughly (\textgreater75\%)
covered.

\begin{table}[t]
\centering
\caption{Overall patch coverage bucketed by the size of the patch in ELOC. \textsc{NP} is the number of patches in the bucket and \textsc{C} is their overall coverage.  Only patches which add or modify executable code are considered.}
\begin{tabular}{lrcrcrc}
\toprule
\multicolumn{1}{c}{} & \multicolumn{2}{c}{\sc $\le$10} & \multicolumn{2}{c}{\sc 11-100} & \multicolumn{2}{c}{\sc >100}  \\
\cmidrule(r){2-3} \cmidrule{4-5} \cmidrule(l){6-7}
\textsc{Application} & NP & C & NP & C & NP & C  \\
\midrule
\beanstalkd & \beanstalkdOverallPatchCovEntriesZero & \beanstalkdOverallPatchCovZero & \beanstalkdOverallPatchCovEntriesTen & \beanstalkdOverallPatchCovTen & \beanstalkdOverallPatchCovEntriesHundred & \beanstalkdOverallPatchCovHundred \\
\binutils & \binutilsOverallPatchCovEntriesZero & \binutilsOverallPatchCovZero & \binutilsOverallPatchCovEntriesTen & \binutilsOverallPatchCovTen & \binutilsOverallPatchCovEntriesHundred & \binutilsOverallPatchCovHundred \\
\git & \gitOverallPatchCovEntriesZero & \gitOverallPatchCovZero & \gitOverallPatchCovEntriesTen & \gitOverallPatchCovTen & \gitOverallPatchCovEntriesHundred & \gitOverallPatchCovHundred \\
\lighttpd & \lighttpdOverallPatchCovEntriesZero & \lighttpdOverallPatchCovZero & \lighttpdOverallPatchCovEntriesTen & \lighttpdOverallPatchCovTen & \lighttpdOverallPatchCovEntriesHundred & \lighttpdOverallPatchCovHundred \\
\lighttpdtwo & \lighttpdtwoOverallPatchCovEntriesZero & \lighttpdtwoOverallPatchCovZero & \lighttpdtwoOverallPatchCovEntriesTen & \lighttpdtwoOverallPatchCovTen & \lighttpdtwoOverallPatchCovEntriesHundred & \lighttpdtwoOverallPatchCovHundred \\
\memcached & \memcachedOverallPatchCovEntriesZero & \memcachedOverallPatchCovZero & \memcachedOverallPatchCovEntriesTen & \memcachedOverallPatchCovTen & \memcachedOverallPatchCovEntriesHundred & \memcachedOverallPatchCovHundred \\
\redis & \redisOverallPatchCovEntriesZero & \redisOverallPatchCovZero & \redisOverallPatchCovEntriesTen & \redisOverallPatchCovTen & \redisOverallPatchCovEntriesHundred & \redisOverallPatchCovHundred \\
\vim & \vimOverallPatchCovEntriesZero & \vimOverallPatchCovZero & \vimOverallPatchCovEntriesTen & \vimOverallPatchCovTen & \vimOverallPatchCovEntriesHundred & \vimOverallPatchCovHundred \\
\zeromq & \zeromqOverallPatchCovEntriesZero & \zeromqOverallPatchCovZero & \zeromqOverallPatchCovEntriesTen & \zeromqOverallPatchCovTen & \zeromqOverallPatchCovEntriesHundred & \zeromqOverallPatchCovHundred \\
\bottomrule
\end{tabular}
\label{tbl:patch-coverage-buckets}
\end{table}

Table~\ref{tbl:patch-coverage-buckets} presents the same patch
coverage statistics, but with the patches bucketed by their size into
three categories: less than 10 ELOC, between 11 and 100 ELOC, and
greater than 100 ELOC.  For all benchmarks, patches are distributed
similarly across buckets, with the majority of patches having $le$10
ELOC and only a few exceeding 100 ELOC. Except for \beanstalkd and \vim, the
average coverage of patches with $\le$10 ELOC is higher than for those with
\textgreater100 ELOC, but the coverage of the middle-size category varies.

Finally, the first column in Table~\ref{tbl:latent} shows the overall
patch coverage, \ie the percentage of covered ELOC across all patches.  For
\beanstalkd, \binutils, \git, \memcached and \vim, it is within five percentage
points from the overall program coverage, while for the other benchmarks it is
substantially lower---for example, the average overall program coverage in
\redis is \redisCoverageAverage, while the overall patch coverage is only
\redisOverallPatchCoverage.


%% \noindent
%% \binutils: \binutilsCoverageAverage vs \binutilsOverallPatchCoverage \\
%% \git: \gitCoverageAverage vs \gitOverallPatchCoverage \\
%% \lighttpdtwo: \lighttpdtwoCoverageAverage vs \lighttpdtwoOverallPatchCoverage\\
%% \redis: \redisCoverageAverage vs \redisOverallPatchCoverage\\
%% \zeromq: \zeromqCoverageAverage vs \zeromqOverallPatchCoverage\\
%% \memcached: \memcachedCoverageAverage vs \memcachedOverallPatchCoverage \\

% What fraction of patch code is tested within a few revisions after it is added, \ie what is the {\em latent patch coverage}?

In some projects, tests exercising the patch are added only after the
code has been submitted, or the patch is only enabled (\eg by changing
the value of a configuration parameter) after related patches or tests
have been added.  To account for this development style, we also
recorded the number of ELOC in each patch which are only covered in
the next few revisions (we considered up to ten subsequent revisions).
We refer to the ratio between the number of such ELOC and the total
patch ELOC as \textit{latent patch coverage}.

We counted these lines by keeping a sliding window of uncovered
patch lines from the past ten revisions and checking whether the
current revision covers them.  When a patch modifies a
source file, all entries from the sliding window associated with lines
from that file are remapped if needed, using the line mapping algorithm
discussed in Section~\ref{sec:design}.

\begin{table}[t]
\centering
\caption{Overall latent patch coverage: the fraction of the lines of code in all patches that are only executed by the regression suite in the next 1, 5 or 10 revisions. The overall patch coverage is listed for comparison.}
\begin{tabular}{lrrrr}
\toprule
\textsc{Application} & \textsc{Overall} & \textsc{+1} & \textsc{+5} & \textsc{+10}  \\
\midrule
\beanstalkd    & \beanstalkdOverallPatchCoverage  & \beanstalkdLatentOne  & \beanstalkdLatentFive  &  \beanstalkdLatentTen \\
\binutils    & \binutilsOverallPatchCoverage  & \binutilsLatentOne  & \binutilsLatentFive  &  \binutilsLatentTen \\
\git         & \gitOverallPatchCoverage       & \gitLatentOne       & \gitLatentFive       &  \gitLatentTen \\
\lighttpd    & \lighttpdOverallPatchCoverage  & \lighttpdLatentOne  & \lighttpdLatentFive  &  \lighttpdLatentTen \\
\lighttpdtwo    & \lighttpdtwoOverallPatchCoverage  & \lighttpdtwoLatentOne  & \lighttpdtwoLatentFive  &  \lighttpdtwoLatentTen \\
\memcached   & \memcachedOverallPatchCoverage & \memcachedLatentOne & \memcachedLatentFive &  \memcachedLatentTen \\
\redis       & \redisOverallPatchCoverage     & \redisLatentOne     & \redisLatentFive     &  \redisLatentTen \\
\vim      & \vimOverallPatchCoverage    & \vimLatentOne    & \vimLatentFive    &  \vimLatentTen \\
\zeromq      & \zeromqOverallPatchCoverage    & \zeromqLatentOne    & \zeromqLatentFive    &  \zeromqLatentTen \\
\bottomrule
\end{tabular}
\label{tbl:latent}
\end{table}

Table~\ref{tbl:latent} shows the overall latent patch coverage \ie the
fraction of patch lines that are covered in the next few revisions
after the patch is introduced. We report the results for three sliding
window sizes: one, five and ten revisions. The latent patch
coverage is significantly smaller compared to the overall patch
coverage, accounting at most for \redisLatentTen in \redis, where,
as previously pointed out, the developers almost never add code and
tests in the same revision.

As conjectured, we found two main causes of latent patch coverage:
tests being added only after the patch was written (this was the case
in \lighttpdtwo, where 12 revisions which only add tests cover an
additional 74 ELOC) and patch code being enabled later on. In fact,
the majority of latent patch coverage in \lighttpdtwo---337 lines---is
obtained by 6 revisions which change no test files.  Upon manual
inspection, we found that the code involved was initially unused, and
only later revisions added calls to it.

Latent patch coverage is important to consider in various coverage
analyses. The delay of several revisions until obtaining the patch
coverage can be an artefact of the development methodology, in which
case it should be assimilated into the normal patch coverage. Furthermore,
our results show that in most of the systems analysed, latent patch
coverage is small but non-negligible.

\subsection{Bug analysis}
\label{sec:bugs}

% Is the coverage of buggy code less than average?

%To answer these RQs, we collected bug data according to the
For this part of our study, we collected bug data according to the
methodology presented in Section~\ref{sec:design} and we limited our
analysis to the four systems which lend themselves to automatic
identification of bug fixes based on commit messages:
\lighttpd, \memcached, \redis and \zeromq.  The other five systems
use non-specific commit messages for bug fixes, requiring an extensive
manual analysis or more complex algorithms such as machine learning
and natural language processing to understand the contents of a
specific revision~\cite{categorization:esem10}.  We ignored revisions
which do not affect executable files, such as fixes to the build
infrastructure or the documentation and then manually confirmed
that the remaining revisions are indeed bug
fixes~\cite{bug-feature:icse13} and further removed fixes which modify
only non-executable lines (e.g. variable declarations). We thus
obtained \lighttpdFixes fixes in \lighttpd, \memcachedFixes fixes in \memcached
and \redisFixes fixes each in \redis and \zeromq.

%% \begin{table*}[t]
%% \caption{Bug fix--coverage correlation analysis. Only fixes which contain executable lines are considered.}
%% \begin{center}
%% \begin{tabular}{|l|r||r|r||r|r|}
%% \cline{3-4}\cline{5-6}
%% \multicolumn{2}{c}{}    & \multicolumn{2}{|c||}{\bf Coverage (median)} & \multicolumn{2}{|c|}{\bf \#Fully Covered} \\ \hline
%% \bf App      & \bf \#Fixes & \bf Overall  & \bf Fix & \bf Overall & \bf Fix      \\ \hline
%% \memcached   & \memcachedFixes & \memcachedPatchCovMedian  & \memcachedFixLineCoverageMedian  & \memcachedFullyCoveredPercent  & \memcachedFixesFullyLineCoveredPercent  \\ \hline
%% \redis       & \redisFixes     & \redisPatchCovMedian      & \redisFixLineCoverageMedian      & \redisFullyCoveredPercent      & \redisFixesFullyLineCoveredPercent  \\ \hline
%% \zeromq      & \zeromqFixes    & \zeromqPatchCovMedian     & \zeromqFixLineCoverageMedian     & \zeromqFullyCoveredPercent     & \zeromqFixesFullyLineCoveredPercent \\ \hline
%% \end{tabular}
%% \end{center}
%% \label{tbl:bugs}
%% \end{table*}

\begin{table}[t]
\centering
\caption{The median coverage and the number of revisions achieving 100\% 
coverage for the revisions containing bug fixes.  The overall metrics
are included for comparison.}
\begin{tabular}{lrrrr}
\toprule
\multicolumn{1}{c}{}    & \multicolumn{2}{c}{\sc Coverage (med)} & \multicolumn{2}{c}{\sc Fully Covered} \\
\cmidrule{2-3}\cmidrule{4-5}
\textsc{Application} & \textsc{Overall} & \textsc{Fix} & \textsc{Overall} & \textsc{Fix}      \\
\midrule
\lighttpd    & \lighttpdPatchCovMedian   & \lighttpdFixLineCoverageMedian   & \lighttpdFullyCoveredPercent   & \lighttpdFixesFullyLineCoveredPercent  \\
\memcached   & \memcachedPatchCovMedian  & \memcachedFixLineCoverageMedian  & \memcachedFullyCoveredPercent  & \memcachedFixesFullyLineCoveredPercent  \\
\redis       & \redisPatchCovMedian      & \redisFixLineCoverageMedian      & \redisFullyCoveredPercent      & \redisFixesFullyLineCoveredPercent  \\
\zeromq      & \zeromqPatchCovMedian     & \zeromqFixLineCoverageMedian     & \zeromqFullyCoveredPercent     & \zeromqFixesFullyLineCoveredPercent \\
\bottomrule
\end{tabular}
\label{tbl:fixes}
\end{table}

%% \begin{table}[t]
%% \caption{The overall coverage of buggy code, identified according to the methods presented.  The overall patch coverage is included for comparison.}
%% \begin{center}
%% \begin{tabular}{|l|r|r|}
%% \hline
%% \bf App      & \bf Overall                     &    \bf Buggy       \\\hline % & \bf  Line map.       \\ \hline
%% \memcached   & \memcachedOverallPatchCoverage  & \memcachedBugLineCoverage \\\hline %& \memcachedOriginsLineCoverage \\ \hline
%% \redis       & \redisOverallPatchCoverage      & \redisBugLineCoverage     \\\hline %& \redisOriginsLineCoverage     \\ \hline
%% \zeromq      & \zeromqOverallPatchCoverage     & \zeromqBugLineCoverage   \\\hline % & \zeromqOriginsLineCoverage    \\ \hline
%% \end{tabular}
%% \end{center}
%% \label{tbl:bugs}
%% \end{table}

We measured the patch coverage of these revisions and report the median values
in Table~\ref{tbl:fixes}, together with the corresponding overall metric, for
comparison.  For \lighttpd, \memcached and \redis, the coverage for fixes is
higher than that for other types of patches.  For \redis, the median value
jumps from \redisPatchCovMedian to \redisFixLineCoverageMedian, while for
\lighttpd, the median value increases from \lighttpdPatchCovMedian to
\lighttpdFixLineCoverageMedian; for \memcached the difference is less
pronounced.  On the other hand, the fixes in \zeromq are covered less than on
average.  The fraction of fixes which have 100\% coverage follows the same
trend.

%% We also determined that 23 of the fixes included a regression test
%% in \memcached, 0 in \redis and 3 in \zeromq, which was particularly
%% surprising for \redis, suggesting that the coverage improvement
%% experienced is likely incidental.


%% used two methods for determining the code
%% responsible for the bug, starting from the bug fixes introduced above.
%% One way to answer this question is to identify the revision where a
%% bug was introduced and determine the code coverage. However, this
%% approach presents several challenges: (a) identifying the revision
%% where a bug was introduced, starting from the fix, is difficult and
%% can require semantic analysis of the code; (b) a bug may result from
%% the interaction of two or more revision and (c) a revision may
%% introduce both buggy and correct code, and differentiating between
%% them is difficult. An alternative solution starts from the observation
%% that bug-fixing revisions are usually only addressing the bug, without
%% touching unrelated code.

To try to understand whether buggy code is less thoroughly tested than
the rest of the code, we started from the observation that bug-fixing
revisions are usually only addressing the bug, without touching
unrelated code.  Because of this, we can identify the code responsible
for the bugs by looking at the code which is removed or modified by bug-fixing
revisions and compute its coverage in the revision before the fix.  The
coverage for this code is \lighttpdBugLineCoverage for \lighttpd---slightly
lower than the overall patch coverage, \memcachedBugLineCoverage for
\memcached---roughly the same as the overall patch coverage,
\redisBugLineCoverage for \redis---much larger than the overall patch coverage,
and \zeromqBugLineCoverage for \zeromq---significantly lower.

%% We report in Table~\ref{tbl:bugs} the overall coverage for
%% this code: as it can be observed, the coverage is roughly the same as
%% the overall patch coverage for \memcached, more than double
%% for \redis, and significantly lower for \zeromq.

%% Very interestingly, the buggy code median coverage in \memcached
%% was \memcachedBugLineCoverageMedian, bigger than the average 
%% \memcached coverage, and \memcachedBugsFullyLineCovered out
%% of \memcachedBugs bugs were fully covered, yet not triggered. 

%% The second method improves on the first by using the line mapping
%% algorithm to track the lines are removed or modified by the bug-fixing
%% revisions to \textit{origins}, \ie the revisions and locations where
%% they were introduced. (Note that different lines in a given fix may
%% map back to different revisions.)  One issues raised by this method is
%% that only a fraction of the origins (30\%--53\%) lie within the time
%% span considered.  The last column in Table~\ref{tbl:bugs} reports the
%% overall coverage for the in-range origins.  For \redis and \memcached,
%% the coverage is significantly lower than that obtained without line
%% mapping information, indicating that the coverage of those lines has
%% improved over time.  However, the improvement is likely to be
%% incidental, \ie not specifically intending to test those lines.  An
%% interesting aspect to investigate in future work is whether there is
%% indeed a better correlation between \textit{intentional coverage}, \ie
%% tests that are together or immediately after the patch in order to
%% test it, rather than coverage more generally.  %For \redis, 

%don't indicate a correlation between buggy code and coverage, and it fact it couldn
While these numbers cannot be used to infer the correlation between
the level of coverage and the occurrence of bugs---the sample is too
small, and the bugs collected are biased by the way they are
reported---they suggest the limitations of line coverage as a testing
metric, with bugs still being introduced even by patches which are
fully covered by the regression test suite. Therefore, even well-tested
code may contain bugs, which can manifest themselves after prolonged
operation in the production environment, and runtime technologies like
NVX can be useful in those scenarios.

%Therefore, even for
%well-tested code, tools which thoroughly check each program statement
%for bugs using techniques such as symbolic execution can be useful in
%practice---for instance, our tool ZESTI~\cite{zesti} was specifically
%designed to enhance existing regression tests to check for corner-case
%scenarios.

%% Combining bug data and coverage data allows us to determine how many
%% bugs are in code which was already tested, i.e. executed without
%% triggering the bug.  This may happen because the bug is only activated
%% in corner-case scenarios.  Systems with a high number of bugs present
%% in tested code may benefit from symbolic execution-based testing tools
%% such as ZESTI~\cite{zesti}, which transparently instrument the
%% existing tests to check for corner-case scenarios.  On the other hand,
%% systems where the bugs are present in untested code can benefit from
%% test generation tools such as KATCH~\cite{katch}.
